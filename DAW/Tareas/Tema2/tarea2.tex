\input{preambulo.tex}

%----------------------------------------------------------------------------------------
%	TÍTULO Y DATOS DEL ALUMNO
%----------------------------------------------------------------------------------------

\title{
\vspace{10ex}
\normalfont \normalsize
\huge \textbf{Tarea 2: Configuración y Administración de Servidores de Aplicaciones}
}
\author{Francisco Javier Sueza Rodríguez}
\date{\normalsize\today}

%----------------------------------------------------------------------------------------
%                                     DOCUMENTO
%----------------------------------------------------------------------------------------
\begin{document}


\maketitle

\thispagestyle{empty}

\vspace{68ex}

\begin{center}
    \begin{tabular}{l l}
        \textbf{Centro}: & IES Aguadulce \\
        \textbf{Ciclo Formativo}: & Desarrollo Aplicaciones Web (Distancia)\\
        \textbf{Asignatura}: & Despliegue de Aplicaciones Web\\
        \textbf{Tema}: & Tema 2 - Configuración y Administración de Servidores de Aplicaciones\\
    \end{tabular}
\end{center}

\newpage

\tableofcontents

\newpage


\section{Ejercicio 1}

\subsection{Enunciado}

\begin{itemize}
    \item \textbf{Apartado 1.A}

    Describe los módulos y enumera los módulos más importantes de Apache y de Tomcat. Y además, describe los principales archivos de configuración de Apache.

    \item \textbf{Apartado 1.B}

    Utiliza la \textbf{instalación} del servidor \textbf{Apache Tomcat} que se realizó en la tarea online 1 y realiza el siguiente apartado:

    \begin{itemize}
        \item \textbf{Configura} Tomcat, para que puedas acceder vía web con un \textbf{usuario} al \textbf{gestor de aplicaciones} de Tomcat.
        \item \textbf{Crea un usuario} que permita acceder a  la interfaz html  y permita el acceso a la interfaz de texto sin formato llamado \textbf{usuarioXXX} donde las X sean las 3 últimas cifras de tu nº de DNI, y password \textbf{DAW.23}.
    \end{itemize}
\end{itemize}

Para este apartado \textbf{debes realizar}:
\begin{itemize}
    \item Documento de texto con las capturas de pantalla necesarias, que sean legibles, indicativas de que has realizado lo que se pide, que demuestren que todo ha funcionado correctamente, y en las que aparezca tu perfil de la plataforma de enseñanza.
\end{itemize}

\subsection{Solución}




% Bibliography

%\newpage
%\bibliography{citas}
%\bibliographystyle{unsrt}

\end{document}