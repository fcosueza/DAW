\newglossaryentry{Unicode}{
    name={Unicode},
    description={El Estándar Unicode es un estándar de codificación de caracteres diseñado para facilitar el tratamiento informático, transmisión y visualización de textos de múltiples lenguajes y disciplinas técnicas además de textos clásicos de lenguas muertas.}
}

\newglossaryentry{ANSI}{
    name={ANSI},
    description={El Instituto Nacional Estadounidense de Estándares, más conocido como ANSI (por sus siglas en inglés: American National Standards Institute), es una organización sin fines de lucro que supervisa el desarrollo de estándares para productos, servicios, procesos y sistemas en los Estados Unidos.}
}

\newglossaryentry{biunivoca}{
    name={biunívoca},
    description={Una correspondencia biunívoca, o correspondencia uno-a-uno, es simplemente una correspondencia unívoca cuya correspondencia inversa también es unívoca. En otras palabras, la relación biunívoca se establece cuando para cada elemento del primer conjunto que se corresponde con solo un elemento del segundo conjunto, tal elemento del segundo conjunto se corresponde con solo aquel elemento del primer conjunto.}
}

\newglossaryentry{CODASY}{
    name={CODASY},
    description={Conference on Data System Languages}
}

\newglossaryentry{IDMS}{
    name={IDMS},
    description={Integrated Data Management System}
}

\newglossaryentry{SQL99}{
    name={SQL99},
    description={Es la definición estandar del lenguaje de consulta de bases de datos, también denominado SQL-3. Fue creado en 1999.}
}

\newglossaryentry{OLAP}{
    name={OLAP},
    description={OLAP es el acrónimo en inglés de procesamiento analítico en línea (On-Line Analytical Processing). Su objetivo es agilizar la consulta de grandes cantidades de datos.}
}


\newglossaryentry{redundancia}{
    name={redundancia},
    description={Repetición de un mismo dato dentro de una base de datos.}
}

\newglossaryentry{inconsistencia}{
    name={inconsistencia},
    description={Se produce inconsistencia en los datos cuando datos iguales hacen referencia a distintas cosas. Es decir, cuando distintas copias de los mismos datos no coinciden.}
}

\newglossaryentry{integridad}{
    name={integridad},
    description={Consiste en la veracidad de los datos almacenados con respecto a la información esperada.}
}

\newglossaryentry{constantes}{
    name={constantes},
    description={Cantidad que tiene un valor fijo en un calculo, proceso, etc...}
}

\newglossaryentry{expresiones}{
    name={expresiones},
    description={Combinación de constantes, variables o funciones, que es interpretada (evaluada) de acuerdo a las normas particulares de precedencia y asociación para un lenguaje de programación en particular. Como en matemáticas, la expresión es su "valor evaluado", es decir, la expresión es una representación de ese valor.}
}

\newglossaryentry{funciones}{
    name={funciones},
    description={Es un grupo de instrucciones con un objetivo en particular y que se ejecuta al ser llamada desde otra función o procedimiento. Una función puede llamarse múltiples veces e incluso llamarse a sí misma (función recurrente). Las funciones pueden recibir datos desde afuera al ser llamadas a través de los parámetros y deben entregar un resultado. Se diferencian de los procedimientos porque estos no devuelven un resultado.}
}





