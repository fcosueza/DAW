\input{preambulo.tex}

%----------------------------------------------------------------------------------------
%	TÍTULO Y DATOS DEL ALUMNO
%----------------------------------------------------------------------------------------

\title{
\vspace{10ex}
\normalfont \normalsize
\huge \textbf{Tarea 1: La Empresa y su Clasificación}
}
\author{Francisco Javier Sueza Rodríguez}
\date{\normalsize\today}

%----------------------------------------------------------------------------------------
%                                     DOCUMENTO
%----------------------------------------------------------------------------------------
\begin{document}

\maketitle

\thispagestyle{empty}

\vspace{75ex}

\begin{center}
    \begin{tabular}{l l}
        \textbf{Centro}: & IES Aguadulce \\
        \textbf{Ciclo Formativo}: & Desarrollo Aplicaciones Web (Distancia)\\
        \textbf{Asignatura}: & Empresa e Iniciativa Emprendedora\\
        \textbf{Tema}: & Tema 1 -  La Empresa y su Clasificación\\
    \end{tabular}
\end{center}

\newpage

\section{Actividad 1: Elige una Empresa}

\subsection{Enunciado}
Imaginaos, es la empresa que eligió Claudia para ejemplificar los contenidos de la unidad 1, ahora te proponemos que seas tú el que elija una empresa en la que apliques lo que hemos estudiado. Los motivos de esta elección pueden ser los que quieras: trabajas en ella, te gustaría formar parte de su plantilla o haberla creado, está relacionada con el ciclo formativo que cursas, es de tu localidad o has leído sobre ella y te gusta su organización, porque es muy rentable, por la forma de contratar y de crear equipos, por su originalidad y creatividad, porque es socialmente responsable…

Responde a estos tres apartados en no más de 10 líneas.

\begin{enumerate}[label=\roman*.]
    \item El nombre de la empresa y el motivo por el que la has elegido.
    \item Actividad que realiza: ¿qué necesidades satisface en el mercado? ¿Produce bienes y/o presta servicios?
    \item En esta tarea te pedimos que elijas tú una empresa y hagas lo mismo que Claudia.
\end{enumerate}

\subsection{Solución}
\begin{enumerate}[label=\roman*.]
    \item La empresa que he elegido es \textbf{Urban Green Club}, una Startup que se dedica a la implantación y mejora de huertos urbanos, motivo por el cual \textbf{la he elegido}, ya que me parece que es una actividad bastante interesante.

    \item La empresa se dedica al \textbf{asesoramiento y creación de huertos urbanos} y la concienciación medioambiental. En este aspecto es una empresa que \textbf{presta servicios} para cubrir la \textbf{necesidad} de la gente de estar en contacto con la naturaleza en medios urbanos y de cultivar su propia fruta y verdura.

    \item Entre sus clientes se encuentra tanto \textbf{ayuntamientos} como \textbf{particulares}, siendo una de las \textbf{principales características} de ambos tener cierta concienciación sobre la sostenibilidad y la necesitad de crear espacios verdes en las grandes ciudades que permitan mantener cierto contacto con la naturaleza.
\end{enumerate}

\section{Actividad 2: Indica sus Elementos}

\subsection{Enunciado}
Identifica los elementos de la empresa que has elegido en la actividad 1 siguiendo la clasificación establecida en la unidad: Elementos humanos, elementos materiales y organización.

\subsection{Solución}
Los principales \textbf{elementos} que podemos encontrar en la empresa \textbf{Urban Green Club} son los siguientes:

\begin{itemize}
    \item \textbf{Elementos Humano}: en esta empresa los elementos humanos están compuesto por la \textbf{CEO} o la directora ejecutiva, y por el resto de empleados. En este caso hay \textbf{empleados dedicados} al \textbf{marketing} de la empresa, al \textbf{asesoramiento técnico} sobre la implantación de los huertos, a la \textbf{gestión de proyectos}.

    \item \textbf{Elementos Materiales}: ya que la empresa se dedica a la creación y mejora de huertos urbanos deberá tener materiales para dicho fin, como diferentes tipos de abonos, plantas, fertilizantes, elementos para el riego, herramientas para realizar la plantación, etc.. Además, deberán tener equipos informáticos para buscar información, llevar a cabo labores de marketing, establecer contacto con clientes, etc. También deberá tener un local para desarrollar su actividad, en este caso, está en el Centro de Iniciativas Empresariales de Granada.

    \item \textbf{Elementos de Organización}: hay diferentes elementos de organización en esta empresa, ya que hay que \textbf{organizar las laboras de marketing}, algo fundamental para dar a conocer. Además, los \textbf{proyectos de huerto urbano} necesitan una \textbf{buena organización} para que sean productivos y no se desperdicien elementos materiales, especialmente en esta empresa que tiene la sostenibilidad por bandera. Por lo que habrá que organizar la adquisición de materiales para los huertos así como usarlos de la forma más eficiente posible adaptándolos a los requisitos de los clientes.
\end{itemize}

\section{Actividad 3: Analiza sus Funciones}
\subsection{Enunciado}
En la unidad hay un ejemplo que puede servirte de ayuda para la realización de este apartado de la tarea: ¡Pincha en el dibujo de la fábrica de muebles!.
Te pedimos que determines y describas brevemente las funciones que puede tener la empresa de tu ejemplo. Esta actividad no debe ocupar más de 10 líneas.

Orientaciones: Te recomendamos para ello que sigas los siguientes pasos:

\begin{enumerate}
    \item Identifica posibles tareas que haya que realizar en el negocio.
    \item Agrupa las tareas que sean afines: grupos de tareas.
    \item Identifica con qué función se corresponde cada grupo de tareas.
\end{enumerate}

Recuerda que en los comienzos de una empresa o dependiendo del tipo que sea, no se desarrollen todas las funciones mencionadas en la unidad o unas son más importantes que otras.

\subsection{Solución}
En esta empresa, aunque es una Startup de reciente creación, las funciones están bastante bien delimitadas. La \textbf{función comercial} consistirá en diseñar y administrar la página web de la empresa así como las diferentes acciones publicitarias. La \textbf{función de producción} consistirá en el diseño e implantación de los huertos urbanos adaptados a las exigencias del clientes y realizados de forma sostenible. Para llevar a la cabo la implantación de los huertos de forma eficiente deberá de haber un control de los gastos y beneficios, concretados en al \textbf{función financiare}, que ademas deberá de intentar captar inversión exterior para su rápido crecimiento. La \textbf{función social} la cumplirán los empleados en la diferentes áreas, como marketing, expertos en agronomía, etc.  Por último, la \textbf{función directiva} consistirá en organizar los equipos implicados en todas las funciones anteriores para que trabajen conjuntamente de forma eficiente.

\section{Actividad 4: Clasifica tu Empresa}

\subsection{Enunciado}
Clasifica la empresa que pusiste en la actividad 1 atendiendo a los criterios de clasificación establecidos en el tema:

\begin{itemize}
    \item Sector económico en el que se incluye:
    \item Actividad económica que realiza:
    \item Dimensión o efectivos de personal:
    \item Ámbito geográfico en el que ejerce su actividad:
    \item Titularidad del negocio:
\end{itemize}

\subsection{Solución}
La clasificación de Urban Green Club es la siguiente:

\begin{itemize}
    \item \textbf{Sector económico en el que se incluye}:  es una empresa que se dedica al \textbf{sector terciario}.
    \item \textbf{Actividad económica que realiza}: la actividad económica que realiza según las Clasificación Nacional de Actividades Económicas (CNAE) es la de \textbf{Apoyo a las Actividades de Agricultura} (CNAE 0161) incluida dentro de las empresas de servicios.
    \item \textbf{Dimensión o efectivos de personal}: según el número de trabajadores, es una \textbf{microempresa}, ya que no tiene más de 10 trabajadores.
    \item \textbf{Ámbito geográfico en el que ejerce su actividad}: es una empresa \textbf{local}, ya que solo trabaja en Granada y en pueblos de los alrededores.
    \item \textbf{Titularidad del negocio}: el capital de la empresa es \textbf{privado}, por lo que es una \textbf{empresa privada}.
\end{itemize}

\section{Actividad 5: ¡Eres el Responsable de la Ética de la Empresa!}

\subsection{Enunciado}
Imagina que fueras el empresario o empresaria de la empresa que has seleccionado en la pregunta 1, indica una acción concreta que llevarías a cabo en materia de Responsabilidad Social y que formaría parte de la cultura de tu empresa. Te pedimos que nos digas en no más de 10 líneas:

\begin{enumerate}
    \item El grupo de interés o stackeholders al que va dirigida.
    \item Una descripción de la acción.
    \item Algunos recursos que necesitas para llevarla a cabo: contactos, colaboraciones, dinero, materiales, etc.
    \item Los indicadores que puedes utilizar para valorar si la medida es eficaz.
\end{enumerate}

\subsection{Solución}

La actividad que propondría sería la \textbf{creación de talleres de concienciación sobre las sostenibilidad}. Esta medida iría enfocada tanto a \textbf{la comunidad} como al \textbf{medio ambiente} (\textbf{stakeholders}) y consistiría en impartir talleres gratuitos donde se dieran charlas sobre la necesidad de la sostenibilidad y se enseñara a la gente a crear pequeños huertos en sus edificios, aumentando las zonas verdes en la ciudad o el barrio de esta forma. Se \textbf{necesitaría} un local para impartir los talleres así como un profesor para hacerlo. También sería necesario adquirir material de jardinería y sería interesante la colaboración de alguna personalidad local en alguno de los talleres. Para comprobar la eficacia y el grado de satisfacción de los asistentes se podría realizar \textbf{encuestas} al final de los talleres así como habilitar un \textbf{buzón de sugerencias}, tanto online como físico.

\section{Actividad 6: Compartes de Empresa en el Foro}

\subsection{Enunciado}
“La empresa que he elegido”, este es el nombre de un hilo del foro en el que compartirás con nuestro grupo de clase la empresa elegida y el trabajo realizado sobre ella. Intenta responder de forma atractiva, clara y breve a las siguientes cuestiones sin que ninguno de los apartados tenga más de dos renglones:

\begin{itemize}
    \item La empresa elegida se llama.......
    \item Se dedica a ........
    \item Sus principales clientes son....
    \item La he clasificado, siguiendo los criterios de la unidad, de la siguiente forma:  (Si se trata de una startup indica por qué)
    \item Una actividad vinculada a la responsabilidad social de la empresa es la siguiente (puede ser positiva o negativa):
    \item El motivo por el que la he elegido es .....
    \item Para más información puedes consultar su web o.....:
\end{itemize}

Recuerda: Ninguno de los apartados de esta intervención debe de ocupar más de dos renglones. Intentamos dar a conocer nuestra elección, informar, no aburrir a nuestros compañeros o hacer una tesis sobre la misma.

Transcribe obligatoriamente en este documento tu intervención en el foro mediante un “pantallazo” o captura de pantalla para su corrección en esta tarea.

\subsection{Solución}
En la siguiente figura, se incluye la captura con el mensaje posteado en el foro de la unidad.

\begin{figure}[H]
    \centering
    \includegraphics[scale=0.30]{captura-foro.png}
    \caption{Captura del hilo ``La empresa que he elegido''}
\end{figure}

\section{La Empresa y el Entorno}

\subsection{Enunciado}
En el foro de esta unidad encontrarás un hilo que se llama: “Las empresas de mi localidad”.

Participa en el mismo e indica:

\begin{enumerate}
    \item Cuál es tu localidad
    \item Pon un ejemplo de cómo las empresas, con sus buenas o malas prácticas, repercuten en tu entorno.
    \item Relaciona esa situación con los Objetivos de Desarrollo Sostenible (ODS) de la Organización de Naciones Unidas.
\end{enumerate}

Incluye obligatoriamente en este apartado una captura de pantalla de tu intervención para facilitar su corrección.

\textbf{Orientaciones}: investiga en Internet, consulta en los medios de comunicación locales, pregunta en el ayuntamiento, observa…. Recuerda que debe tratarse de un problema de tu localidad.

\subsection{Solución}
En la siguiente figura se incluye la captura solicitada sobre mi intervención en el foto de esta unidad.

\begin{figure}[H]
    \centering
    \includegraphics[scale=0.30]{captura-foro-2.png}
    \caption{Captura del hilo ``Las empresas de mi localidad''}
\end{figure}

\section{Actividad 8: Las Importancia de los Foros}
\subsection{Enunciado}
Los temas que se proponen para el debate de esta unidad consideramos que son importantes. Los puedes encontrar en la presentación de la unidad en el foro y en los diferentes hilos que se han abierto.

Lo que te pedimos en este apartado de la tarea es que elijas uno de ellos y expreses tu opinión. A continuación incluye la captura de pantalla en el documento de la tarea como respuesta.

\subsection{Solución}
A continuación dejo la captura de mi intervención en el foro. En concreto, ha sido en el ha sido en el hilo ``\textbf{Compartimos buenas prácticas empresariales...}'', donde he aportado el Código de Buena Conducta sobre Proveedores de Google.

\begin{figure}[H]
    \centering
    \includegraphics[scale=0.30]{captura-foro-3.png}
    \caption{Captura del hilo ``Compartimos buenas prácticas empresariales..''}
\end{figure}

\section{Actividad 9}
\subsection{Enunciado}
Indica en este apartado de la tarea:

\begin{itemize}
    \item Una franquicia relacionada con el ciclo que estás cursando o en la que te gustaría trabajar
    \item Una ONG que conozcas o en la que colabores y que esté relacionada con tus estudios.
    \item La Asociación de Empresas más representativa en tu localidad y algunos de los servicios que ofrece a empresas de nueva creación.
    \item Una startup que te guste y el motivo por el que la has elegido (Te invitamos a que la compartas en el foro, si te parece oportuno).
\end{itemize}

\subsection{Solución}
Una \textbf{franquicia}, es el permiso que otorga una empresa (franquiciador) sobre el uso de su nombre, imagen, producto o actividad comercial a un particular (franquiciado), a cambio de unas prestaciones económicas.

Una \textbf{ONG} no podría considerarse una empresa, ya que el fin último de una empresa es el \textbf{beneficio económico}, mientras que las ONG son entidades suelen ser \textbf{entidades sin ánimo de lucro} cuyo último fin es la ayuda social y mejora de condiciones del entorno.

La \textbf{principal diferencia} entre una \textbf{asociación de empresario} y un \textbf{sindicado} es que mientras los primeros velan por los derecho de los \textbf{empresarios}, los \textbf{sindicados} velan por los derechos de los trabajadores.

Una \textbf{PYME} no tiene porque se una \textbf{Startup}, ya que lo que caracteriza a una Startup es el crecimiento rápido y la búsqueda de inversión para la consecución de ese crecimiento, además de estar ligadas a las TIC. Mientras que las PYMES tienen crecimiento más lento y requieren una inversión inicial mayor, además de no tener porque estar ligadas a las TIC.

A continuación se responde al resto de preguntas en la siguiente lista:

\begin{itemize}
    \item Una \textbf{franquicia} relacionada con la informática y en la que me gustaría trabajar es \href{https://www.dynos.es/}{Dynos}, una franquicia de tiendas de informática.

    \item Una \textbf{ONG} que conozco relacionada con mis estudios es \href{https://abierta.org/}{Abierta}, una asociación que se encarga de reacondicionar ordenares viejos y entregarlos a personas o entidades que los necesiten.

    \item Una \textbf{asociación de empresarios} de mi localidad es \href{https://www.ajeandalucia.org/granada/}{AJE Andalucia Granada} ó la \textbf{Asociación de Jóvenes Empresarios}. Entre sus servicios podemos encontrar:
    \begin{itemize}
        \item \textbf{Financiación} mediante subvenciones y líneas de financiación para las empresas.
        \item \textbf{Asesoramiento} y apoyo para la gestión de tu empresa.
        \item \textbf{Representación} mediante el traslado de reivindicación y necesidades a las Administraciones e Instituciones Públicas.
        \item \textbf{Promoción} de la empresa mediante el uso de diferentes canales de comunicación
        \item etc...
    \end{itemize}

    \item Una \textbf{Startup} que me llama la atención es \href{https://www.reviverdes.com/}{Reviverdes}, una empresa que se dedica al diseño y la decoración sostenible. La he elegido porque me gusta sus principios y su compromiso con la sostenibilidad.
\end{itemize}

\section{¿Ubicarías tu empresa en un vivero?}

\subsection{Enunciado}
Investiga:

\begin{itemize}
    \item cuál es el vivero o incubadora de empresas más cercano a tu localidad,
    \item indica el organismo al que pertenece y
    \item un enlace para acceder a la información sobre el vivero.
\end{itemize}

Personalmente \textbf{si ubicaría} mi empresa en un vivero, ya que considero que es una \textbf{muy buena ayuda} para comenzar, además de ofrecer la posibilidad de entrar en contacto con otras empresas relacionadas con el sector y que pueden derivar en colaboraciones que pueden resultar altamente beneficiosas.

En Granada podemos encontrar diversas incubadoras de empresas, pero una de las más grandes es el \textbf{Centro de Empresas CIE Diputación}, que como su propio nombre indica, pertenece a la Diputación de Granada y ofrece varios servicios como alojamiento, salas de reuniones y formación, entre otras cosas.

En el siguiente enlace podemos encontrar más información sobre este vivero de empresas gestionado por la diputación: \href{https://www.granadaempresas.es/centro-empresas-cie-diputacion/}{Centro de Empreas CIE Diputación}.

\section{Evalúa tu Trabajo}
Personalmente creo que el trabajo puede estar \textbf{sobre un 8}, aunque es bastante completo quizá no he elegido adecuadamente la empresa ya que en algunos puntos, como en el de clasificación sobre su actividad económica, me ha resultado complicado clasificar y no se si finalmente lo he hecho correctamente.

La verdad que la tarea me ha parecido muy buena para afianzar los conocimientos que hemos visto durante este tema y se ha centrado, desde mi punto de vista, en los temas más importantes vistos.


% Bibliography

%\newpage
%\bibliography{citas}
%\bibliographystyle{unsrt}

\end{document}