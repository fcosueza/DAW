% % % % % % % % % % % % % % % % % % % % % % % % % % % % % % % % % % % % % % % % % % % %
%                                                                                     %
% Short Sectioned Assignment LaTeX Template Version 1.0 (5/5/12)                      %
% This template has been downloaded from: http://www.LaTeXTemplates.com               %
%                                                                                     %
% Original author:  Frits Wenneker (http://www.howtotex.com)                          %
%                                                                                     %
% Modified by: Fco Javier Sueza Rodríguez (fcosueza@disroot.org)                      %
%                                                                                     %
% Changes:                                                                            %
%	    - Custom Chapters, Sections and Subsections (titlesec package)                %
%           - Document type scrbook (oneside)                                         %
%           - Use babel-lang-spanish package and marvosym                             %
%           - Use hyperref, enumitem, tcolorbox and glossaries packages               %
%           - Use Time New Roman (mathptmx), Helvetic and Courier fonts               %
%                                                                                     %
% License: CC BY-NC-SA 3.0 (http://creativecommons.org/licenses/by-nc-sa/3.0/)        %
%                                                                                     %
% % % % % % % % % % % % % % % % % % % % % % % % % % % % % % % % % % % % % % % % % % % %

%-----------------------------------------------%
%	              Packages                  %
%-----------------------------------------------%

\documentclass[paper=a4, fontsize=11pt, oneside]{scrbook}

% ---- Text Input/Output ----- %

\usepackage[T1]{fontenc}
\usepackage[utf8]{inputenc}
\usepackage{mathptmx}
\usepackage[scaled=.92]{helvet}
\usepackage{courier}
\usepackage[indent=12pt]{parskip}

\usepackage{geometry}
\geometry{verbose,tmargin=3cm,bmargin=3cm,lmargin=2.6cm,rmargin=2.6cm}

% ---- Language ----- %

\usepackage[spanish]{babel}
\usepackage{marvosym}

% ---- Another packages ---- %

\usepackage{amsmath,amsfonts,amsthm}
\usepackage{graphics,graphicx}
\usepackage{titlesec}
\usepackage{fancyhdr}
\usepackage{tcolorbox}
\usepackage{hyperref}
\usepackage{enumitem}
\usepackage[automake]{glossaries}

%--------------------------------------------------------------------%
%                      Customizing Document                          %
%--------------------------------------------------------------------%


% ----------- Custom Chapters, Sections and Subsections -------------- %

\titleformat{\chapter}[display]
			{\bfseries\Huge}
			{Tema \ \thechapter} {0.5ex}
			{\vspace{1ex}\centering}

\titleformat{\section}[hang]
			{\bfseries\Large}
			{\thesection}{0.5em}{}

\titleformat{\subsection}[hang]
			{\bfseries\large}
			{\thesubsection}{0.5em}{}

\titleformat{\subsubsection}[hang]
			{\bfseries\large}
			{\thesubsubsection}{0.5em}{}

\hypersetup{
    colorlinks=true,
    linkcolor=black,
    urlcolor=magenta
}

% ------------------- Custom heaaders and footers ------------------- %

\pagestyle{fancyplain}

\fancyhead[]{}
\fancyfoot[L]{}
\fancyfoot[C]{}
\fancyfoot[R]{\thepage}

\renewcommand{\headrulewidth}{0pt} % Remove header underlines
\renewcommand{\footrulewidth}{0pt} % Remove footer underlines

\setlength{\headheight}{13.6pt} % Customize the height of the header

% --------- Numbering equations, figures and tables ----------------- %

\numberwithin{equation}{section} % Number equations within sections
\numberwithin{figure}{section} % Number figures within sections
\numberwithin{table}{section} % Number tables within sections

% ------------------------ New Commands ----------------------------- %

\newcommand{\horrule}[1]{\rule{\linewidth}{#1}} % Create horizontal rule command


%----------------------------------------------------------------------------------------
%	TÍTULO Y DATOS DEL ALUMNO
%----------------------------------------------------------------------------------------

\title{
\normalfont \normalsize
\huge \textbf{Instalación y Configuración de Odoo}
}
\author{Francisco Javier Sueza Rodríguez}
\date{\normalsize\today}

%----------------------------------------------------------------------------------------
%                                     DOCUMENTO
%----------------------------------------------------------------------------------------
\begin{document}

\maketitle

\vspace{2ex}

\begin{center}
    \begin{tabular}{l l}
        \textbf{Centro}: & IES Aguadulce \\
        \textbf{Ciclo Formativo}: & Desarrollo Aplicaciones Web (Distancia)\\
        \textbf{Asignatura}: & Entornos de Desarrollo\\
        \textbf{Tema}: & Tema 1 - Desarrollo de Software y Entornos de Desarrollo \\
    \end{tabular}
\end{center}

\vspace{10ex}

\section{Descripción}
Nos piden desarrollar una aplicación web que permita visualizar películas a través del navegador.

Por lo tanto el cliente quiere que el software que le vamos a desarrollar esté optimizado para cualquier navegador, sobre todo Edge, Firefox, Chrome y Safari. Necesitan que la web sea responsive para que se adapte a todo tipo de dispositivos móviles y de escritorio.

Los usuarios pagarán una suscripción que podrá ser semanal, mensual, anual o por día, por lo que el sistema registrará la información de los usuarios y el tipo de suscripción que tienen activa. Por lo tanto, al tratar datos de carácter personal, el sistema debe cumplir las disposiciones recogidas en la Ley Orgánica de Datos Personales. La aplicación debe proporcionar las herramientas necesarias para que los usuarios paguen (por lo que debe usar una pasarela segura para la realización de los pagos).

El sistema debe permitir a los usuarios buscar y consultar la información sobre las películas. El resultado de la búsqueda debe aparecer en pantalla rápido sin demorarse más de 2 segundos. Debe almacenar información sobre las películas que se pueden visualizar. Se tendrán en cuenta por tanto las leyes referentes a los derechos de autor.  El sistema debe permitir a un usuario ver todas las películas que quiera durante el periodo de tiempo que tenga contratado en su suscripción.

Nuestro cliente no sabe si añadirá algo más adelante porque es la primera vez que va a trabajar con un sistema así, pero lo que sí tiene claro es que quiere ir viendo como va quedando la web, por lo que le proponemos ir mostrándole el proyecto conforme se vaya desarrollando cada una de las funcionalidades de la aplicación y está conforme con ello.

\section{Solución}
Teniendo en cuenta el caso práctico
% Bibliography

%\newpage
%\bibliography{citas}
%\bibliographystyle{unsrt}

\end{document}