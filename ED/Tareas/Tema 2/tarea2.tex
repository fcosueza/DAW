\input{preambulo.tex}

%----------------------------------------------------------------------------------------
%	TÍTULO Y DATOS DEL ALUMNO
%----------------------------------------------------------------------------------------

\title{
    \vspace{10ex}
    \normalfont \normalsize
    \huge \textbf{Actividades de la Unidad 2}
}
\author{Francisco Javier Sueza Rodríguez}
\date{\normalsize\today}

%----------------------------------------------------------------------------------------
%                                     DOCUMENTO
%----------------------------------------------------------------------------------------
\begin{document}

\maketitle

\thispagestyle{empty}

\vspace{75ex}

\begin{center}
    \begin{tabular}{l l}
        \textbf{Centro}: & IES Aguadulce \\
        \textbf{Ciclo Formativo}: & Desarrollo Aplicaciones Web (Distancia)\\
        \textbf{Asignatura}: & Entornos de Desarrollo\\
       \textbf{Tema}: & Tema 1 - Desarrollo de Software y Entornos de Desarrollo\\
    \end{tabular}
\end{center}

\newpage

\tableofcontents

\vspace{15ex}

\hrule

\vspace{10ex}

\listoffigures

\newpage

\section{Caso Práctico}
La empresa SOFTED ha recibido un nuevo encargo de software

En esta ocasión, todo el software que se desarrolle debería estar integrado en algún entorno de desarrollo libre. Ada ha elegido trabajar con NetBeans y utilizar como sistema operativo Windows o Linux.

Una vez planteado el análisis de requerimientos y el diseño de la aplicación (tal y como hicimos en la unidad anterior), se requiere tener un buen entorno para el diseño y ejecución de los programas.

- Venga, no te agobies, vamos a echar un café y nos ponemos manos a la obra,...- insiste Juan.

\section{Actividades}


% Bibliography

\newpage
\bibliography{citas}
\bibliographystyle{unsrt}

\end{document}