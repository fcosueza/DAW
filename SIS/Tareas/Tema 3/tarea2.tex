\input{preambulo.tex}

%----------------------------------------------------------------------------------------
%	TÍTULO Y DATOS DEL ALUMNO
%----------------------------------------------------------------------------------------

\title{
\vspace{10ex}
\normalfont \normalsize
\Huge \textbf{Tarea 3: Análisis y Diseño de Redes}
}
\author{Francisco Javier Sueza Rodríguez}
\date{\normalsize\today}

%----------------------------------------------------------------------------------------
%                                     DOCUMENTO
%----------------------------------------------------------------------------------------
\begin{document}

\maketitle

\thispagestyle{empty}

\vspace{68ex}

\begin{center}
    \begin{tabular}{l l}
        \textbf{Centro}: & IES Aguadulce \\
        \textbf{Ciclo Formativo}: & Desarrollo Aplicaciones Web (Distancia)\\
        \textbf{Asignatura}: & Sistemas Informáticos\\
        \textbf{Tema}: & Tema 3 -  Redes de Ordenadores\\
    \end{tabular}
\end{center}

\newpage

\tableofcontents

\vspace{15ex}

\hrule

\vspace{10ex}

\listoffigures

\newpage

\section{Caso Práctico}
Antonio y Juan han sido nombrados responsables del área de sistemas y redes de la nueva empresa AguadulSoft.

La semana pasada tuvieron una reunión con Ada en la que se les comunicó que tendrían que encargarse de proyectos de implantación de redes en oficinas de clientes. Su primer trabajo será un proyecto pequeño para una biblioteca/centro lúdico de una pequeña población cercana. Antes de lanzarse a dicha tarea van a repasar algunos conceptos básicos de redes.

\section{Actividades}

\subsection{Actividad 1: Medios de TransPara esta actividad debes realizar dos tablas con información obtenida en Internet o en los contenidos de la unidad sobre los siguientes medios de transmisión:misión}

\subsubsection{Parte A: Medios Guiados}
Recientemente, el estándar cableado Ethernet de 2.5 Gbps sobre cable de par trenzado está ganando popularidad. Buscando información en Internet, haz una tabla que incluya lo siguiente:







% Bibliography

\newpage
\bibliography{citas}
\bibliographystyle{unsrt}

\end{document}