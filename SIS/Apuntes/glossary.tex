\newglossaryentry{programa}{
    name={programa},
    description={Un programa informático es un conjunto de instrucciones que el indican al ordenador las operaciones que debe realizar para lograr un fin determinado}.
}

\newglossaryentry{registros}{
    name={registros},
    description={Son una pequeña porción de memoria compuesta de biestables que permiten guardar valores en binario. Su tamaño viene dado por el número de bits que pueden almacenar; por ejemplo, puede ser un registro de 8 bits" o un de 32 bits.}.
}

\newglossaryentry{volatil}{
    name={volátil},
    description={Se conoce como memoria volátil aquella que pierde la información cuando se interrumpe el flujo eléctrico}.
}

\newglossaryentry{interfaces}{
    name={interfaces},
    description={Una interfaz hardware es una conexión física, entre dos aparatos o sistemas independientes, que funciona a través de un protocolo común a ambos. Existen diferentes interfaces estándares como el USB o el SATA, etc., definidas con unas especificaciones técnicas concretas, que deben ser comunes en todos los dispositivos con la misma interfaz, para que puedan conectarse y comunicarse entre ellos}.
}


