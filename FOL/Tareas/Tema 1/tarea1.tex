% % % % % % % % % % % % % % % % % % % % % % % % % % % % % % % % % % % % % % % % % % % %
%                                                                                     %
% Short Sectioned Assignment LaTeX Template Version 1.0 (5/5/12)                      %
% This template has been downloaded from: http://www.LaTeXTemplates.com               %
%                                                                                     %
% Original author:  Frits Wenneker (http://www.howtotex.com)                          %
%                                                                                     %
% Modified by: Fco Javier Sueza Rodríguez (fcosueza@disroot.org)                      %
%                                                                                     %
% Changes:                                                                            %
%	    - Custom Chapters, Sections and Subsections (titlesec package)                %
%           - Document type scrbook (oneside)                                         %
%           - Use babel-lang-spanish package and marvosym                             %
%           - Use hyperref, enumitem, tcolorbox and glossaries packages               %
%           - Use Time New Roman (mathptmx), Helvetic and Courier fonts               %
%                                                                                     %
% License: CC BY-NC-SA 3.0 (http://creativecommons.org/licenses/by-nc-sa/3.0/)        %
%                                                                                     %
% % % % % % % % % % % % % % % % % % % % % % % % % % % % % % % % % % % % % % % % % % % %

%-----------------------------------------------%
%	              Packages                  %
%-----------------------------------------------%

\documentclass[paper=a4, fontsize=11pt, oneside]{scrbook}

% ---- Text Input/Output ----- %

\usepackage[T1]{fontenc}
\usepackage[utf8]{inputenc}
\usepackage{mathptmx}
\usepackage[scaled=.92]{helvet}
\usepackage{courier}
\usepackage[indent=12pt]{parskip}

\usepackage{geometry}
\geometry{verbose,tmargin=3cm,bmargin=3cm,lmargin=2.6cm,rmargin=2.6cm}

% ---- Language ----- %

\usepackage[spanish]{babel}
\usepackage{marvosym}

% ---- Another packages ---- %

\usepackage{amsmath,amsfonts,amsthm}
\usepackage{graphics,graphicx}
\usepackage{titlesec}
\usepackage{fancyhdr}
\usepackage{tcolorbox}
\usepackage{hyperref}
\usepackage{enumitem}
\usepackage[automake]{glossaries}

%--------------------------------------------------------------------%
%                      Customizing Document                          %
%--------------------------------------------------------------------%


% ----------- Custom Chapters, Sections and Subsections -------------- %

\titleformat{\chapter}[display]
			{\bfseries\Huge}
			{Tema \ \thechapter} {0.5ex}
			{\vspace{1ex}\centering}

\titleformat{\section}[hang]
			{\bfseries\Large}
			{\thesection}{0.5em}{}

\titleformat{\subsection}[hang]
			{\bfseries\large}
			{\thesubsection}{0.5em}{}

\titleformat{\subsubsection}[hang]
			{\bfseries\large}
			{\thesubsubsection}{0.5em}{}

\hypersetup{
    colorlinks=true,
    linkcolor=black,
    urlcolor=magenta
}

% ------------------- Custom heaaders and footers ------------------- %

\pagestyle{fancyplain}

\fancyhead[]{}
\fancyfoot[L]{}
\fancyfoot[C]{}
\fancyfoot[R]{\thepage}

\renewcommand{\headrulewidth}{0pt} % Remove header underlines
\renewcommand{\footrulewidth}{0pt} % Remove footer underlines

\setlength{\headheight}{13.6pt} % Customize the height of the header

% --------- Numbering equations, figures and tables ----------------- %

\numberwithin{equation}{section} % Number equations within sections
\numberwithin{figure}{section} % Number figures within sections
\numberwithin{table}{section} % Number tables within sections

% ------------------------ New Commands ----------------------------- %

\newcommand{\horrule}[1]{\rule{\linewidth}{#1}} % Create horizontal rule command


%----------------------------------------------------------------------------------------
%	TÍTULO Y DATOS DEL ALUMNO
%----------------------------------------------------------------------------------------

\title{
\vspace{10ex}
\normalfont \normalsize
\huge \textbf{Actividades de la Unidad 8}
}
\author{Francisco Javier Sueza Rodríguez}
\date{\normalsize\today}

%----------------------------------------------------------------------------------------
%                                     DOCUMENTO
%----------------------------------------------------------------------------------------
\begin{document}

\maketitle

\thispagestyle{empty}

\vspace{75ex}

\begin{center}
    \begin{tabular}{l l}
        \textbf{Centro}: & IES Aguadulce \\
        \textbf{Ciclo Formativo}: & Desarrollo Aplicaciones Web (Distancia)\\
        \textbf{Asignatura}: & Formación y Orientación Laboral\\
        \textbf{Tema}: & Tema 1 -  La Relación Laboral Individual\\
    \end{tabular}
\end{center}

\newpage

\section{Actividad 1}
\subsection{Enunciado}

\begin{enumerate}[label=\alph*]
    \item Expón un ejemplo de relación laboral y cita las características que la definen como tal.
    \item Expón un ejemplo de relación laboral excluída y analiza el motivo para ser considerada así
    \item Respecto al periodo de prueba contesta a:
    \begin{enumerate}[label=\arabic*]
        \item Forma que requiere la misma.
        \item Indica qué tipo de contrato de los estudiados no la contempla en ningún caso.
        \item Finalidad que tiene el mismo.
    \end{enumerate}
    \item Responde a las siguientes cuestiones de forma razonada e indicando el principio que se le aplica.
        \begin{enumerate}[label=\arabic*]
        \item Un contrato de trabajo recoge un periodo de vacaciones de 38 días por año trabajado, el nuevo convenio es más favorable en todo que el anterior pero establece un periodo de vacaciones de 30 días anuales
        \item ¿Puede un trabajador cambiar parte del periodo de descanso por más salario?
    \end{enumerate}
\end{enumerate}

\subsection{Respuesta}

\begin{enumerate}[label=\alph*]
    \item
    \item Expón un ejemplo de relación laboral excluída y analiza el motivo para ser considerada así
    \item Respecto al periodo de prueba contesta a:
    \begin{enumerate}[label=\arabic*]
        \item Forma que requiere la misma.
        \item Indica qué tipo de contrato de los estudiados no la contempla en ningún caso.
        \item Finalidad que tiene el mismo.
    \end{enumerate}
    \item Responde a las siguientes cuestiones de forma razonada e indicando el principio que se le aplica.
    \begin{enumerate}[label=\arabic*]
        \item Un contrato de trabajo recoge un periodo de vacaciones de 38 días por año trabajado, el nuevo convenio es más favorable en todo que el anterior pero establece un periodo de vacaciones de 30 días anuales
        \item ¿Puede un trabajador cambiar parte del periodo de descanso por más salario?
    \end{enumerate}
    \end{enumerate}








% Bibliography

\newpage
\bibliography{citas}
\bibliographystyle{unsrt}

\end{document}