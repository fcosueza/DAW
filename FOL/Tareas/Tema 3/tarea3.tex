\input{preambulo.tex}

%----------------------------------------------------------------------------------------
%	TÍTULO Y DATOS DEL ALUMNO
%----------------------------------------------------------------------------------------

\title{
\vspace{10ex}
\normalfont \normalsize
\huge \textbf{Actividades de la Unidad 3: La Seguridad Social}
}
\author{Francisco Javier Sueza Rodríguez}
\date{\normalsize\today}

%----------------------------------------------------------------------------------------
%                                     DOCUMENTO
%----------------------------------------------------------------------------------------
\begin{document}

\maketitle

\thispagestyle{empty}

\vspace{75ex}

\begin{center}
    \begin{tabular}{l l}
        \textbf{Centro}: & IES Aguadulce \\
        \textbf{Ciclo Formativo}: & Desarrollo Aplicaciones Web (Distancia)\\
        \textbf{Asignatura}: & Formación y Orientación Laboral\\
        \textbf{Tema}: & Tema 3 -  La Seguridad Social\\
    \end{tabular}
\end{center}

\newpage

\tableofcontents

\newpage
\section{Caso Práctico}
Carlos explica a Blanca que como cualquier trabajador en España, debe estar afiliada a la Seguridad Social, es la forma de que participe en el sistema cotizando como trabajadora en activo y de ese modo todos los trabajadores en España, sean del sector que sean y realicen cualquier actividad económica, podamos estar cubiertos, no sólo en asistencia sanitaria, sino también en cuestiones de desempleo y protección.

\section{Enunciado}

\subsection{Actividad 1}
Una vez trabajada y visto el tema que nos ocupa, te sugiero que elijas una de estas dos opciones:

\begin{enumerate}[label=\alph*)]
    \item Aporta alguna idea o sugerencia personal que consideras que serían convenientes introducir o modificar en la Seguridad Social para mejorar el sistema y justifícala.
    \item Haz una valoración personal sobre la Seguridad Social.
\end{enumerate}

\subsection{Actividad 2}
Contesta las siguientes cuestiones que se plantean:

\begin{enumerate}[label=\alph*)]
    \item Define qué es una contingencia común y ejemplifica.
    \item Define qué es una contingencia profesional y ejemplifica.
    \item Cita los diferentes tipos de prestaciones que nos ofrece la Seguridad Social.
    \item Cita las modalidades de prestaciones de la Seguridad Social y cada modalidad ejemplifícala.
\end{enumerate}

\subsection{Actividad 3}
Cumplimenta el siguiente cuadro indicando el régimen  en el que quedan incluidos las personas que aparecen indicadas (puedes marcar con una X). Si en alguno de los ejemplos quieres añadir o comentar algo puedes hacerlo tras el cuadro.

\begin{figure}[H]
    \centering
    \includegraphics[scale=0.50]{tabla.png}
    \caption{Tabla a rellenar con el régimen de la SS}
\end{figure}

\subsection{Actividad 4}
Cita, señala o contesta brevemente:

\begin{enumerate}
    \item Obligaciones formales del empresario con la Seguridad Social.
    \item Sujeto responsable del pago de cuotas (patronal y obrera).
    \item Durante la huelga, ya visto en el tema anterior, ¿Qué sucederá con la obligación económica del pago a la Seguridad Social?
    \item Realiza las capturas de la página web de la Seguridad Social en el que  podemos descargar los modelos referentes al alta y baja de una persona trabajadora.
\end{enumerate}

\subsection{Actividad 5}
Teniendo en cuenta la nómina de la tarea de la unidad anterior (UT01). Determina:

\begin{enumerate}[label=\alph*)]
    \item Las bases de cotización del supuesto explicando cómo se obtienen.
    \item Las cuotas correspondientes al trabajador y el cálculo de las mismas.
    \item La cuotas correspondientes al empresariado y el cálculo de las mismas.
\end{enumerate}

Si en la tarea anterior realizaste la nómina, sólo sería calcular como novedad las deducciones aplicándoles los tipos de cotización (si es que no llegaste a calcularlo). Si tuviste algún error es momento ahora para corregirlo. Si no llegaste a realizarla, ahora puedes hacerlo.

\subsection{Actividad 6}
\begin{enumerate}[label=\alph*)]
    \item Explica los requisitos básicos y necesarios para poder acceder a una protección contributiva.
    \item Supuesto práctico:

    Todas las personas aquí expuestas creen que tienen derecho a algún tipo de prestación económica de la seguridad social en la modalidad contributiva. Indícales en principio cuál sería y los requisitos para acceder a ella, así como la cuantía, duración y aspectos más relevantes.

    \begin{enumerate}
        \item María José está actualmente embarazada de 7 meses y ya no está en condiciones de trabajar por el ritmo que implica su trabajo. Está  afiliada, dada de alta y trabajando en su empresa desde hace 5 años. Su edad es de 35 años.de baja por maternidad  está afiliada y en alta
        \item Marta, ante la incertidumbre que tiene en su empresa, ha decidido acogerse a una excedencia voluntaria para prepararse unas oposiciones al cuerpo de bomberos.
        \item Ángel, está de baja por una gripe y lleva tres años trabajando en una empresa dado de alta.
        \item Manuel necesita saber para el año 2023 en qué momento nos encontramos del sistema transitorio de pensiones.
    \end{enumerate}
\end{enumerate}

\subsection{Actividad 7}
En estos supuestos prácticos identifica, justifica y expón las posibles situaciones legales de desempleo.

\begin{enumerate}
    \item Pablo ha sido despedido por la empresa aplicándole un despido disciplinario calificado por el juez como procedente
    \item Silvia estando en periodo de prueba ha renunciado voluntariamente en su nueva empresa pero en la anterior estuvo siete años cotizando.
    \item Miriam empresaria ha rescindido  en periodo de prueba el contrato a Bernabé. Tiene acumulado de paro en los últimos seis años 359 días.
\end{enumerate}

\subsection{Actividad 8}
Oscar se queda en paro en el mes de mayo. Su Base de Cotización de los últimos seis meses ha sido de 1800 € mensuales, y ha cotizado 1240 días en los últimos seis años.

\begin{enumerate}
    \item Calcula la prestación de desempleo a la que tiene derecho y expón el procedimiento seguido para obtenerla.
    \item Haz una captura de la entidad que gestiona dicha prestación.
    \item Cita los requisitos para tener derecho al cobro de la prestación en la modalidad contributiva.
\end{enumerate}

\section{Solución}

\subsection{Actividad 1}
En nuestro caso, hemos elegido la \textbf{opción a} y vamos a aportar alguna idea que sería conveniente introducir en la Seguridad Social.

En mi opinión una buena adición al Sistema de la Seguridad social sería \textbf{ampliar} la \textbf{cobertura dental} que ofrece la Seguridad Social, ya que ahora mismo esta muy limitado, prácticamente cubriendo solo las extracciones dentales y poco más.

Esto se \textbf{justificaría} ya que la salud dental es algo muy importante, ya no solo para la salud de los ciudadanos, sino también para su autoestima, lo que puede influir en sus interacciones sociales, incluida en la búsqueda de empleo. Actualmente, los precios de muchos procedimientos dentales son prohibitivos para familias que tengan poca renta, y aunque algunos pueden considerarse meramente estéticos, otros son necesarios para una buena salud dental, la cual puede afectar a la salud general a varias escalas.

Por ello, la ampliación de la cobertura, por lo menos a los procedimientos relacionados con la salud, como pueden ser endodoncias, empastes, tratamientos de infecciones como la periodontitis, etc.., sería una buena inclusión en la Seguridad Social y tendrían un  impacto muy positivo en muchos ciudadanos.

\subsection{Actividad 2}

\begin{enumerate}[label=\alph*)]
    \item \textbf{Contingencias Comunes}: son las causadas por alguna enfermedad común o un accidente no laboral,  \textbf{por ejemplo}, una contingencias común sería \textbf{coger alguna enfermedad vírica}, como la Gripe A.

    \item \textbf{Contingencia Profesional}: son aquellas causas por riesgos enfermedad o accidente laboral, \textbf{por ejemplo}, si un operario se \textbf{lesiona una mano} al quedar ésta atrapada en una cinta transportadora.

    \item Las \textbf{principales prestaciones} que nos ofrece la seguridad social son las siguientes:
    \begin{itemize}
        \item Incapacidad Temporal.
        \item Prestación por Nacimiento y Cuidado de Menor.
        \item Suspensión por Riesgo en el Embarazo o Lactancia Natural.
        \item Incapacidad Permanente.
        \item La Jubilación.
        \item Prestación por Muerte y Supervivencia.
        \item Prestación por Desempleo.
    \end{itemize}

    \item La Seguridad Social tiene \textbf{dos modalidades} de prestaciones que son la siguientes:

    \begin{itemize}
        \item \textbf{Prestación Contributiva}: en este tipo de prestaciones no es importante la situación económica del beneficiario, sino que se tienen en cuenta otros requisitos a la hora de poder acceder a ellas, a saber:

        \begin{itemize}
            \item El beneficiario debe estar afiliado a la seguridad social y dado de alta o alta asimilada.
            \item Se deben tener cubiertos los periodos de cotización previos o períodos de carencia, en caso de que sean exigibles, teniendo cada prestación unos límites bien definidos por la ley.
        \end{itemize}

        Este tipo de prestaciones tiene un cuantía económica variables que irá en función de los períodos cotizados y la base reguladora que haya que aplicar dependiendo de la prestación en concreto.

        \textbf{Un ejemplo} de este tipo de prestaciones es \textbf{la jubilación}, donde deberemos de tener una \textbf{cantidad determinada de años cotizados} para poder acceder a ella, y dependiendo de éstos y de la base reguladora que haya que aplicar, los importes a recibir pueden variar enormemente.

        \item \textbf{Prestaciones No Contributivas}: al contrario que en las anteriores, en las prestaciones no contributivas no es necesario tener cubierto un \textbf{período de carencia}, pero si hay están condicionadas a no tener unos ingresos suficientes y a la residencia en el territorio español. La \textbf{cuantía} de este tipo de prestaciones suele ser una cuantía fija para todos los beneficiarios.

        \textbf{Un ejemplo} de estas prestaciones es la \textbf{pensión no contributiva} de \textbf{jubilación}, la cual se concede a trabajadores que no han llegado a cubrir el período de cotización mínimo para acceder a una pensión de jubilación contributiva.
    \end{itemize}
\end{enumerate}

\subsection{Actividad 3}

    \begin{figure}[ht]

    \vspace{3ex}
    \centering

    \setlength{\tabcolsep}{10pt}
    \renewcommand{\arraystretch}{1.4}

    \begin{tabular}{| l | c | c |}
        \hline
         {} & \textbf{Régimen General} & \textbf{Régimen Especial} \\ \hline
        \centering Estudiante &   & X \\
        \hline
        \centering Autónoma &   & X \\
        \hline
        \centering Maestro de Colegio Público &  X &  \\
        \hline
        \centering Técnico Informático* &  X &  \\
        \hline
        \centering Agricultora &   & X \\
        \hline
    \end{tabular}
\end{figure}

* El técnico informático dependerá de la actividad que desarrolle, se han incluido en el Régimen General, pero si el desarrollara su actividad como autónomo, algo que también es común en la profesión, se incluiría dentro del Régimen Especial, en concreto, dentro del Régimen Especial de Autónomos.

\subsection{Actividad 4}

\begin{enumerate}
    \item Las \textbf{obligaciones formales del empresario} son las siguientes:
    \begin{itemize}
        \item \textbf{Inscripción}: deben inscribir su empresa en la TSGSS.
        \item \textbf{Afiliación}: se trata de la incorporación del trabajador a la Seguridad Social.
        \item \textbf{El Alta}: es el acto de incluir al trabajador en el Régimen General o Especial de la seguridad social.
        \item \textbf{La Baja}: se produce cuando el trabajador cesa el trabajo en su empresa y se tiene que comunicar en los 3 días siguientes.
    \end{itemize}

    \item El \textbf{sujeto responsable} de pagar tanto la cuota patronal como la obrera es el \textbf{empresario}.

    \item Durante \textbf{una huelga}, así como durante un \textbf{cierre patronal}, la obligación económica se \textbf{suspenderá}, siempre y cuando la empresa haya presentado los partes de baja con \textbf{3 días de antelación}. Si no se han presentado los partes de baja, la obligación no se suspenderá.

    \item A continuación se muestra una captura con la página de la Seguridad Social donde se puede descargar el modelo \textbf{TA.2/S}, que es el utilizado tanto para el Alta como la Baja del trabajador, así como para comunicar la variación de datos de este.

    \begin{figure}[H]
        \centering
        \includegraphics[scale=0.28]{modelo-ts2.png}
        \caption{Descarga modelo TA.2/S en la Web de la SS}
    \end{figure}
\end{enumerate}

\subsection{Actividad 5}
En nuestro caso, ya tenemos todo calculado de la tarea del tema 1, incluido el cálculo de las deducciones.
\begin{enumerate}[label=\alph*)]
    \item En primer lugar vamos a calcular las bases de cotización. Para ellos, hay que calcular el primero el salar
\end{enumerate}

% Bibliography

\newpage
\bibliography{citas}
\bibliographystyle{unsrt}

\end{document}