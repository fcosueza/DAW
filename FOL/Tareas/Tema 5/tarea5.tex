% % % % % % % % % % % % % % % % % % % % % % % % % % % % % % % % % % % % % % % % % % % %
%                                                                                     %
% Short Sectioned Assignment LaTeX Template Version 1.0 (5/5/12)                      %
% This template has been downloaded from: http://www.LaTeXTemplates.com               %
%                                                                                     %
% Original author:  Frits Wenneker (http://www.howtotex.com)                          %
%                                                                                     %
% Modified by: Fco Javier Sueza Rodríguez (fcosueza@disroot.org)                      %
%                                                                                     %
% Changes:                                                                            %
%	    - Custom Chapters, Sections and Subsections (titlesec package)                %
%           - Document type scrbook (oneside)                                         %
%           - Use babel-lang-spanish package and marvosym                             %
%           - Use hyperref, enumitem, tcolorbox and glossaries packages               %
%           - Use Time New Roman (mathptmx), Helvetic and Courier fonts               %
%                                                                                     %
% License: CC BY-NC-SA 3.0 (http://creativecommons.org/licenses/by-nc-sa/3.0/)        %
%                                                                                     %
% % % % % % % % % % % % % % % % % % % % % % % % % % % % % % % % % % % % % % % % % % % %

%-----------------------------------------------%
%	              Packages                  %
%-----------------------------------------------%

\documentclass[paper=a4, fontsize=11pt, oneside]{scrbook}

% ---- Text Input/Output ----- %

\usepackage[T1]{fontenc}
\usepackage[utf8]{inputenc}
\usepackage{mathptmx}
\usepackage[scaled=.92]{helvet}
\usepackage{courier}
\usepackage[indent=12pt]{parskip}

\usepackage{geometry}
\geometry{verbose,tmargin=3cm,bmargin=3cm,lmargin=2.6cm,rmargin=2.6cm}

% ---- Language ----- %

\usepackage[spanish]{babel}
\usepackage{marvosym}

% ---- Another packages ---- %

\usepackage{amsmath,amsfonts,amsthm}
\usepackage{graphics,graphicx}
\usepackage{titlesec}
\usepackage{fancyhdr}
\usepackage{tcolorbox}
\usepackage{hyperref}
\usepackage{enumitem}
\usepackage[automake]{glossaries}

%--------------------------------------------------------------------%
%                      Customizing Document                          %
%--------------------------------------------------------------------%


% ----------- Custom Chapters, Sections and Subsections -------------- %

\titleformat{\chapter}[display]
			{\bfseries\Huge}
			{Tema \ \thechapter} {0.5ex}
			{\vspace{1ex}\centering}

\titleformat{\section}[hang]
			{\bfseries\Large}
			{\thesection}{0.5em}{}

\titleformat{\subsection}[hang]
			{\bfseries\large}
			{\thesubsection}{0.5em}{}

\titleformat{\subsubsection}[hang]
			{\bfseries\large}
			{\thesubsubsection}{0.5em}{}

\hypersetup{
    colorlinks=true,
    linkcolor=black,
    urlcolor=magenta
}

% ------------------- Custom heaaders and footers ------------------- %

\pagestyle{fancyplain}

\fancyhead[]{}
\fancyfoot[L]{}
\fancyfoot[C]{}
\fancyfoot[R]{\thepage}

\renewcommand{\headrulewidth}{0pt} % Remove header underlines
\renewcommand{\footrulewidth}{0pt} % Remove footer underlines

\setlength{\headheight}{13.6pt} % Customize the height of the header

% --------- Numbering equations, figures and tables ----------------- %

\numberwithin{equation}{section} % Number equations within sections
\numberwithin{figure}{section} % Number figures within sections
\numberwithin{table}{section} % Number tables within sections

% ------------------------ New Commands ----------------------------- %

\newcommand{\horrule}[1]{\rule{\linewidth}{#1}} % Create horizontal rule command


%----------------------------------------------------------------------------------------
%	TÍTULO Y DATOS DEL ALUMNO
%----------------------------------------------------------------------------------------

\title{
\vspace{10ex}
\normalfont \normalsize
\huge \textbf{Actividades de la Unidad 4: Evaluación de Riesgos Profesionales}
}
\author{Francisco Javier Sueza Rodríguez}
\date{\normalsize\today}

%----------------------------------------------------------------------------------------
%                                     DOCUMENTO
%----------------------------------------------------------------------------------------
\begin{document}

\maketitle

\thispagestyle{empty}

\vspace{65ex}

\begin{center}
    \begin{tabular}{l l}
        \textbf{Centro}: & IES Aguadulce \\
        \textbf{Ciclo Formativo}: & Desarrollo Aplicaciones Web (Distancia)\\
        \textbf{Asignatura}: & Formación y Orientación Laboral\\
        \textbf{Tema}: & Tema 4 -  Evaluación de Riesgos Profesionales\\
    \end{tabular}
\end{center}

\newpage

\tableofcontents

\newpage
\section{Caso Práctico}
Jesús trabaja en la empresa  ``TecladoSA'', se dedica al diseño de páginas web y al desarrollo de pequeñas aplicaciones para PYMES y cuenta con una plantilla de cinco personas entre las que se incluye Lola, la empresaria. Lola se dedica a las relaciones públicas, las actividades comerciales y de captación de clientes. Los otros cuatro puestos están ocupados por programadores, Jesús es uno de ellos.

Jesús está preocupado, Carlos tiene la tensión un poco alta y se la ve con mucho estrés, cada vez tiene más trabajo y un hijo pequeño y pasa muchas horas conduciendo. Su compañero Damián está un poco grueso y se queja de que le duelen las piernas y tiene problemas de circulación. A Jaime no le gusta su trabajo, tampoco la forma en la que Lola lo organiza y continuamente está buscando ofertas de empleo pero su situación económica y su carácter le impiden tomar la decisión de marcharse, les dice que duerme poco y está cada vez más irascible. Jaime tiene una hernia discal y está yendo a rehabilitación.

Para la iluminación de la oficina se utilizan tubos fluorescentes porque la luz natural se refleja en las pantallas de los ordenadores. Algunos compañeros se quejan de que al estar tanto tiempo sentados pasan frío y el calefactor de aceite del que disponen no tiene capacidad para calentar una sala tan grande.

\section{Actividades}

\subsection{Actividad 1}

\subsubsection{Enunciado}
Indica de qué forma o maneras podemos intervenir para que la cultura preventiva se convierta en objetivo clave en toda actividad profesional.

\subsubsection{Solución}
La cultura preventiva es algo muy importante y que debería de tener prioridad en cualquier empresa. Algunas de las medidas que se podrían tomar para incentivarla pueden ser:

\begin{itemize}
    \item \textbf{Campañas de concienciación}: realizar campañas informativas de concienciación tanto para los empresarios como para los trabajadores.

    \item \textbf{Aumento de las sanciones}: un aumento en las sanciones por el incumplimiento de la legislación sobre riesgos laborales también podría incentivar la implementación por parte las empresas de una buena cultura preventiva.

    \item \textbf{Aumento de las  inspecciones laborales}: seria otro punto que ayudaría a controlar que se cumple la legislación en materia de prevención de riesgos laboral.

    \item \textbf{Subvenciones}: un programa de ayudas y subvenciones para que las empresas que necesiten realizar una inversión considerable para adaptarse a la normativa puedan mitigar el coste de esa inversión inicial.
\end{itemize}

\subsection{Actividad 2}

\subsubsection{Enunciado}
Reflexiona de manera personal y justifica si la salud tiene que ver algo con el trabajo y las condiciones en las que se desarrolla.

\subsubsection{Solución}
El \textbf{trabajo} es una parte fundamental de la vida de cualquier persona a la que dedicamos un buen número de horas y que nos sirve no solo como fuente de ingresos para cubrir nuestras necesidades básicas sino que también nos ayuda a realizarnos como persona. Por ello, desarrollar nuestra actividad laboral en una condiciones adecuadas es de suma importancia.

El hecho de desarrollar nuestra actividad laboral en \textbf{malas condiciones} puede \textbf{afectar a nuestra salud} a varios niveles, tanto \textbf{físicos} como \textbf{psicológicos}. Por un lado, puede causarnos problemas físicos y enfermedades relacionadas con el puesto de trabajo concreto, que pueden ir desde simples contracturas musculares hasta incluso la muerte en trabajos donde haya riesgos laborales muy graves. Por otro lado, desarrollar nuestra actividad en malas condiciones puede afectarnos psicológicamente, causando depresión, ansiedad, insatisfacción personal, etc.., las cuales afectarán a nuestra calidad de vida y a nuestras relaciones con otras personas, incluidos nuestros familiares.

Por lo tanto, es fundamental que los trabajadores puedan realizar su trabajo en una ambiente idóneo. Esto no solo beneficia al trabajador, ya que un \textbf{trabajador contento} también va a ser una \textbf{trabajador más productivo}.

\subsection{Actividad 3}

\subsubsection{Enunciado}

\begin{enumerate}[label=\alph*)]
    \item Define y justifica qué es un riesgo laboral. Además cita los cuatro grandes grupos que trabajamos.
    \item Los daños derivados de factores de riesgos son de dos tipos: accidentes laborales y enfermedades profesionales. Te pedimos:
    \begin{enumerate}
        \item Diferencias entre ambos tipos de daños.
        \item Ejemplifica cada uno de ellos en una situación real o ficticia.
        \item Explica qué es el accidente in itinere
    \end{enumerate}
\end{enumerate}

\subsubsection{Solución}

\begin{enumerate}[label=\alph*)]
    \item Se \textbf{define} un \textbf{riesgo laboral} como la probabilidad que tiene una trabajador de sufrir un determinado daño, ya sea debido a un accidente o enfermedad, en su trabajo. Podemos clasificar los riesgos laborales en \textbf{4 grupos} según su naturaleza, siendo estos:
    \begin{itemize}
        \item \textbf{Riesgos derivados de las condiciones de seguridad}: son aquellas condiciones materiales que pueden dar lugar a accidentes de trabajo, como la maquinaria y el equipo que se emplea.
        \item \textbf{Riesgos derivados de las condiciones ambientales}: son factores del medio ambiente que se pueden presentar en el ámbito de trabajo de forma normal o modificados por las diferentes técnicas de producción, como por ejemplo diferentes tipos de contaminantes físicos, químicos o biológicos.
        \item \textbf{Riesgos derivados de las condiciones ergonómicas}: son aquellos derivados de la carga física de trabajo, mantenimiento de malas posturas en el trabajo, etc..,
        \item \textbf{Riesgos derivados de las condiciones psicosociales}: son riesgos relacionados con los factores psicosociales del puesto de trabajo, como la amplitud de tareas, altas exigencias, altas responsabilidades, etc...
    \end{itemize}
    \item Los daños derivados de factores de riesgos son de dos tipos: accidentes laborales y enfermedades profesionales. Te pedimos:
    \begin{enumerate}
        \item Tanto los \textbf{accidentes laborales} como las \textbf{enfermedades laborales} son situación en las que se le produce un daño a la salud del trabajado.

        La principal \textbf{diferencia} entre ellos es que mientras un \textbf{accidente laboral} es una lesión por un suceso imprevisto, presentándose de forma súbita, la \textbf{enfermedad laboral} es una enfermedad contraría en el puesto de trabajo donde el daño, por normal general, se va produciendo de forma paulatina en el tiempo.
        \item \textbf{Accidente Laboral}: un ejemplo de accidente laboral sería la descarga eléctrica recibida por un electricista al ir a comprobar la tensión de una instalación eléctrica.
        \item \textbf{Enfermedad Laboral}: un ejemplo de enfermedad laboral sería por ejemplo la fibrosis pulmonar adquirida por una trabajador de una fábrica donde se haya trabajado o se trabaje con amianto.
        \item El \textbf{accidente \textit{in itinere}} es un accidente que sufre un trabajador en el trayecto, tanto de ida como de vuelta, entre su lugar de trabajo y su lugar de residencia.
    \end{enumerate}
\end{enumerate}

\subsection{Actividad 4}

\subsubsection{Enunciado}

En esta actividad pretendemos el estudio de aquellos riesgos que más puedan tener que ver con tu perfil profesional. Contesta y argumenta los siguientes apartados:

\begin{enumerate}[label=\alph*)]
    \item \textbf{Condiciones de seguridad}: Un riesgo de este tipo a tu elección y comentar las circunstancias en las que se produce o lo que lo produce.
    \item \textbf{Condiciones ambientales}: Un riesgo de este tipo a tu elección y comentar las circunstancias en las que se produce o lo que lo produce.
    \item \textbf{Condiciones ergonómicas}: Un riesgo de este tipo a tu elección y comentar las circunstancias en las que se produce o lo que lo produce.
    \item \textbf{Condiciones psicosociales}: Un riesgo de este tipo a tu elección y comentar las circunstancias en las que se produce o lo que lo produce.
\end{enumerate}

\subsubsection{Solución}

\begin{enumerate}[label=\alph*)]
    \item \textbf{Condiciones de seguridad}: un riesgo de este grupo en nuestro sector es el \textbf{riesgo eléctrico}. Un \textbf{ejemplo} de esto sería la \textbf{descarga recibida} al manipular el hardware de un ordenador que ha tenido un mal funcionamiento, no habiéndose tomado las precauciones debidas.
    \item \textbf{Condiciones ambientales}: un riesgo de este grupo serían los \textbf{riesgos por condiciones termohigrométricas}, como por ejemplo, \textbf{sufrir temperaturas altas} debido a un mal acondicionamiento del ambiente del lugar del trabajo, ya que los ordenadores suelen genera una gran cantidad de calor.
    \item \textbf{Condiciones ergonómicas}: un riesgo de este grupo serían los \textbf{riesgos derivados por posturas} en el entorno de trabajo, como \textbf{por ejemplo}, la mala postura en la silla de un programador, pudiendo esta derivar en problemas de espalda.
    \item \textbf{Condiciones psicosociales}: un riesgo de este tipo podría ser el \textbf{riesgo de estrés labora}, como \textbf{por ejemplo} el \textbf{síndrome de burout}, derivado de un estrés continuo por las exigencia de tiempo y cara laboral derivando en un agotamiento físico y mental que se cronifica en el tiempo pudiendo afectar la personalidad y autoestima del trabajador.
\end{enumerate}

\subsection{Actividad 5}

\subsubsection{Enunciado}
De la evaluación de riesgo contesta a las siguientes cuestiones de forma clara y justificada.

\begin{enumerate}[label=\alph*)]
    \item Finalidad de la evaluación.
    \item Justifica el tiempo y el momento en el que se tiene que llevar a cabo.
    \item Responsabilidad de llevarla a efecto.
    \item Cita los colectivos especiales que hay que considerar al hacer la evaluación
\end{enumerate}

\subsubsection{Solución}

\begin{enumerate}[label=\alph*)]
    \item La principal \textbf{finalidad} de la evaluación de riesgos laborales es el análisis de los posibles riesgos laborales presentes en el lugar de trabajo y la gravedad de los mismos, con el fin de priorizar y poder establecer medidas correctoras y preventivas que los eliminen o reduzcan.
    \item La \textbf{evaluación de riesgos laborales} debe realizarse cuando se esta montando el lugar de trabajo y antes de que los trabajadores ocupen dicho lugar, ya que de otro modo se estaría exponiendo a las trabajadores de los diferentes riesgos laborales que puedan acontecer en el lugar de trabajo de forma innecesaria. Así, se puede realizar un plan de prevención, teniendo en cuenta la maquinaria y herramientas que van a usar los trabajadores, incluso pudiendo elegir otro tipo de maquinaria que genere menos riesgos para estos.
    \item La \textbf{responsabilidad} de la evaluación de riesgos laborales corre a cargo de empresario. Si este no esta formado lo suficiente en este aspecto deberá contratar un experto para que se haga un análisis lo mas preciso posible.
    \item Los \textbf{colectivos especiales} que hay que considerar en el momento de la evaluación son:
    \begin{itemize}
        \item Aquellos especialmente sensibles a determinados riesgos.
        \item Embarazadas o en período de lactancia.
        \item Menores de 18 años.
        \item Trabajadores con contratos temporales.
    \end{itemize}
\end{enumerate}

\subsection{Actividad 6}

\subsubsection{Enunciado}
Dado tu perfil profesional sugiere medidas preventivas que debes tomar para preservar tu salud y que no llegue a resentirse.

\subsubsection{Solución}
Teniendo en cuenta las características el trabajo desarrollado por un \textbf{técnico informático}, las principales medidas de prevención que debemos tomar son las siguientes:

\begin{itemize}
    \item Comprobar que los \textbf{equipos informáticos} están siempre \textbf{apagados} cuando se va a manipular el hardware, usando destornilladores dieléctricos y descargando la placa base de \textbf{electricidad estática}.
    \item Coger la postura adecuada, \textbf{flexionando las piernas}, cuando se van a \textbf{levantar pesos} para evitar tirones y problemas de espalda.
    \item Mantener el lugar de trabajo \textbf{limpio y ordenado} para evitar golpes  y caídas.
    \item Cuando se trabaja delante de un ordenador, \textbf{tomarse un breve descanso} cada cierto tiempo para descansar la vista, aprovechando también para estirar la espalda y hacer algún ejercicio que pueda liberar la tensión acumulada.
    \item \textbf{Ajustar} adecuadamente \textbf{la silla} para evitar problemas musculares y tener una posición correcta delante del ordenador.
    \item Realizar \textbf{ejercicio y actividades sociales} en nuestras horas de descanso para desconectar y evitar el riesgo de burnout.
\end{itemize}

\subsection{Actividad 7}

\subsubsection{Enunciado}
Si el riesgo se llega a materializar puede dar lugar a daños en el trabajador. Según la clasificación de las condiciones y el riesgo elegido en la actividad 4, determina y concreta tipos de daños  para cada uno de los riesgos elegidos en dicha actividad.

\subsubsection{Solución}

\begin{enumerate}[label=\alph*)]
    \item \textbf{Condiciones de seguridad} (Riesgo Eléctrico): en este punto mencionamos el riesgo eléctrico. Los daños causas por este tipo de riesgo pueden ser diversos, desde \textbf{daños leves}, como una simple contracción muscular momentánea, hasta \textbf{daños muy graves}, como la muerte, si la descarga eléctrica tiene la intensidad suficiente.

    \item \textbf{Condiciones ambientales} (Riesgos por Condiciones Termohigrométricas): en este caso, uno de los \textbf{daños} que podrían causas los riesgos por condiciones termohigrométricas sería, por ejemplo, \textbf{un golpe de calor}, debido a la exposición durante un tiempo prolongado a altas temperaturas si el local no está adecuadamente climatizado.

    \item \textbf{Condiciones ergonómicas} (Riesgos Derivados por Malas Posturas): con los riesgos derivados por malas posturas podemos producirnos \textbf{diferentes daños}, algunos más severos que otros, siendo algunos ejemplos \textbf{dolores de espalda}, \textbf{síndrome del tunel carpiano}, \textbf{problemas cervicales}, etc...

    \item \textbf{Condiciones psicosociales} (Riesgo Laboral por Estrés): los daños derivados de los riesgos derivados del estrés son muchos y variados, como \textbf{dolores de cabeza}, \textbf{problemas de sueño}, \textbf{tensión o dolor muscular}, \textbf{fatiga}, etc...
\end{enumerate}





% Bibliography

\newpage
\bibliography{citas}
\bibliographystyle{unsrt}

\end{document}