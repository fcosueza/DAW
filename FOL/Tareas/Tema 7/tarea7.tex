\input{preambulo.tex}

%----------------------------------------------------------------------------------------
%	TÍTULO Y DATOS DEL ALUMNO
%----------------------------------------------------------------------------------------

\title{
\vspace{10ex}
\normalfont \normalsize
\huge \textbf{Actividades de la Unidad 7: Equipos de Trabajo y Gestión del Conflicto}
}
\author{Francisco Javier Sueza Rodríguez}
\date{\normalsize\today}

%----------------------------------------------------------------------------------------
%                                     DOCUMENTO
%----------------------------------------------------------------------------------------
\begin{document}

\maketitle

\thispagestyle{empty}

\vspace{65ex}

\begin{center}
    \begin{tabular}{l l}
        \textbf{Centro}: & IES Aguadulce \\
        \textbf{Ciclo Formativo}: & Desarrollo Aplicaciones Web (Distancia)\\
        \textbf{Asignatura}: & Formación y Orientación Laboral\\
        \textbf{Tema}: & Tema 7: Equipos de Trabajo y Gestión del Conflicto\\
    \end{tabular}
\end{center}

\newpage

\tableofcontents

\newpage
\section{Caso Práctico}
Lola, la empresaria y dueña de ``TECLASA'' ya vimos que tiene intención y de hecho ya ha empezado a poner al día a la empresa en todos los aspectos. Ese esfuerzo y empeño pasa por el convencimiento de que todo esto es posible si trabajan en equipo y aúnan esfuerzos y voluntad. Está dispuesta a ello y se pone en marcha, contrata una consultoría de recursos humanos para que le ayuden a llevarlo a cabo.

Dos semanas más tarde de comenzar con este proyecto los resultados están empezando a notarse. Ya los trabajadores más que definirlos como un grupo, empiezan a ser equipo. Esto es  el inicio de  un triunfo de todos, por todos y para todos!!

\section{Actividades}

\subsection{Actividad 1}

\subsubsection{Enunciado}
\begin{enumerate}
    \item Seguro que has trabajado en alguna ocasión en grupo. Por ello te pedimos:
    \begin{enumerate}
        \item Expón dos ventajas sobre esta forma de trabajar.
        \item Expón dos inconvenientes.
    \end{enumerate}
    \item Coméntanos un ejemplo personal donde hayas trabajado en equipo y hayas desarrollado las funciones características en el equipo basado en las “5C”, descríbelas.
\end{enumerate}

\subsubsection{Solución}
\begin{enumerate}
    \item Seguro que has trabajado en alguna ocasión en grupo. Por ello te pedimos:
    \begin{enumerate}
        \item Las \textbf{dos ventajas} del trabajo en equipo son:
        \begin{itemize}
            \item El en \textbf{trabajo en equipo} se pueden llegar a soluciones más adecuadas a un problema, ya que hay \textbf{diferentes puntos de vista}, lo que proporciona una visión más amplia de las soluciones.
            \item En un \textbf{equipo de trabajo} existe mejor comunicación entre los integrantes de la que podría haber si estuvieran trabajando de forma individual.
        \end{itemize}

        \item Los \textbf{dos inconvenientes} del trabajo en equipo son:
        \begin{itemize}
            \item La \textbf{aparición de conflictos} por los diferentes tipos de personalidad de los integrantes del equipo, que pueden crear mal ambiente y ralentizas el trabajo.
            \item La \textbf{falta de coordinación} puede generar mucho caos y que el trabajo no se complete en el tiempo estimado.
        \end{itemize}
    \end{enumerate}

    \item La última \textbf{experiencia de trabajo} en equipo fue en la facultad, cuando estudiaba el Grado de Informática, en concreto en la asignatura de ``Desarrollo Basado en Agentes''. La experiencia fue muy positiva y obtuvimos un buen resultado en el desarrollo del trabajo propuesto. Respecto a las ``5C'', las funciones características se desarrollaron de la siguiente forma:

    \begin{enumerate}
        \item \textbf{Comunicación}: la comunicación con el equipo fue bastante fluida, ya que que todo el mundo aportaba soluciones y escuchaba las del resto. Está era tanto de tipo presencial, cuando nos reuníamos en la facultad, como mediante videoconferencia, las cuales realizábamos los fines de semana.

        \item \textbf{Coordinación}: la coordinación fue bastante eficiente, ya que usamos herramientas de desarrollo ágil, en concreto \textbf{SCRUM}, que facilita mucho el proceso de coordinación. Sumado a esto, se usaron herramientas de programación colaborativa como \textbf{Github}, que también ayudaron a la coordinación fuera muy efectiva.

        \item \textbf{Complementariedad}: en este aspecto nos complementamos bastante bien, ya que había gente de diferentes especialidades que tenían diferentes enfoques de como afrontar el mismo problema, lo que ayudo a llegar a soluciones óptimas.

        \item \textbf{Confianza}: aunque realmente no nos conocíamos mucho, la verdad que la gente del grupo era ``muy sana'', y no fue muy complicado establecer relaciones de confianza (incluso de amistad) entre los integrantes del grupo.

        \item \textbf{Compromiso}: por suerte, y diferencia de otras experiencias previas, todos los integrantes del grupo se comprometieron plenamente con el desarrollo de la tarea.
    \end{enumerate}
\end{enumerate}

\subsection{Actividad 2}

\subsubsection{Enunciado}
Tras leer el proceso de creación de un equipo y sus diferentes etapas, comenta si los enunciados que proponemos a continuación son verdaderos o falsos.

\begin{enumerate}[label=\alph*.]
    \item Las etapas por las que atraviesan un equipo y en orden son; Inicio, primeras dificultades, madurez, acoplamiento y agotamiento.
    \item La fase de Acoplamiento es cuando el equipo está acoplado, controla el trabajo y sus miembros han aprendido a trabajar juntos . El equipo entra en una fase muy productiva.
    \item La existencia de un equipo siempre está justificada. Siempre será un tiempo y esfuerzo bien empleado.
    \item La designación de sus miembros estará en función de la tarea asignada. Hay que buscar a personas con capacidades y experiencia suficiente para cubrir adecuadamente las distintas facetas del trabajo encomendado.
    \item En los equipos de trabajo el número de sus integrantes no es importante. Estará justificado el número de componentes cualquiera que fuera.
    \item El equipo debe estar completamente integrado en la organización y es muy conveniente fomentar el espíritu de equipo.
    \item Los cometidos y los objetivos del grupo a alcanzar vienen ya supuestos, no es necesario definirlos.
    \item Hay que seleccionar personas con capacidad para trabajar en equipo evitando individualistas. Es preferible además que tengan personalidades diferentes ya que ello enriquece al equipo.
\end{enumerate}

\subsubsection{Solución}

\begin{enumerate}[label=\alph*.]
    \item Este enunciado \textbf{es falso}, ya que las fases no están en el orden adecuado. La \textbf{fase de acoplamiento} va antes de la \textbf{fase de madurez}
    \item Este enunciado \textbf{es falso}, ya que lo se que esta describiendo es la \textbf{fase de madurez}, no la de acoplamiento
    \item Es enunciado \textbf{es falso}, ya que solo deben formarse equipos cuando haya razones de peso, sino sería un perdida de tiempo y recursos.
    \item Es enunciado \textbf{es verdadero}, ya que es importante que los miembros tengan la experiencia suficiente para desarrollar la actividad asignada dentro del equipo.
    \item Es enunciado \textbf{es falso}, ya que el número de integrantes si es importante y tiene que ser el necesario para que el equipo realice su labor de las forma mas eficaz y eficiente posible.
    \item Es enunciado \textbf{es verdadero}, el equipo debe estar integrado perfectamente en la empresa y su posición dentro de esta bien determinada.
    \item Este enunciado \textbf{es falso}, los cometidos y objetivos del equipo deben ser definidos con claridad, ya que de estos dependerán las decisiones que se tomen respecto a su formación durante el proceso de formación, como el número de integrantes, etc..
    \item Este enunciado \textbf{es verdadero}, la diversidad de personalidad contribuye a un enriquecimiento del equipo.
\end{enumerate}

\subsection{Actividad 3}

\subsubsection{Enunciado}
\begin{enumerate}
    \item Después de trabajar la asertividad, ¿crees o te consideras asertiv@? Justifica la respuesta.
    \item Piensa en alguien que consideres un buen/a comunicador/a, y justifica según lo trabajado en la unidad tan buen calificativo. En el Anexo III "Recomendaciones para hablar en público" tienes algunas o muchas de ellas.
\end{enumerate}

\subsubsection{Solución}
\begin{enumerate}
    \item Personalmente me \textbf{considero una persona asertiva}, ya que siempre, desde el respeto, intento expresar mi opinión o deseos de una forma abierta y directa intentado no herir los sentimientos de los demás, aunque si es verdad que en ciertos temas, como política o religión, puede haber momentos en los que me ``excite'' demasiado y pueda llegar a hacer algún comentario no demasiado oportuno, aunque intento que esto no suceda.
    \item Para mí, un buen comunidad es, por ejemplo, \textbf{Iñaki Gabilondo}, ya que es profesional que cumple muchas de las características que debe tener un buen comunicador, como:

    \begin{enumerate}
        \item Conoce el material del que habla y se siente cómodo hablando de ello.
        \item Es un comunicador muy natural.
        \item Sabe concentrar el mensaje en lo que los oyentes esperan escuchar.
        \item Evita muletillas y pausas innecesarias.
        \item Su voz tiene el poder de generar emociones.
    \end{enumerate}

    Estás, son, entre otras muchas características, las que considero que hacen de \textbf{Iñaki Gabilondo} un buen comunicador.
\end{enumerate}

\subsection{Actividad 4}

\subsubsection{Enunciado}
\begin{enumerate}
    \item ¿Qué cualidades destacas en ti para trabajar en grupo y qué es lo que más te costaría? Identifícate con un rol positivo y uno negativo como integrante de un equipo de trabajo y desarrolla y justifica la elección.
    \item De los estilos de dirección, elige uno acorde con tu personalidad y justifica la elección.
\end{enumerate}

\subsubsection{Solución}
\begin{enumerate}
    \item Los \textbf{roles} con los que me puedo identificar dentro de un equipo son:
    \begin{itemize}
        \item \textbf{Rol Positivo: Investigador}: por norma general tiendo a incentivar la búsqueda de información continua sobre los problemas que se estén tratando en el equipo.

        \item \textbf{Rol Negativo: Gracioso}: aunque me cueste reconocerlo es cierto que a veces puedo hacer gracias o bromas en momentos no muy adecuados.
    \end{itemize}
    \item Respecto de los \textbf{estilos de dirección} con el que más me identifico es con el \textbf{democrático}, ya que considero que se debe escuchar a todos los integrantes del equipo e incentivar el desarrollo de su potencial.
\end{enumerate}

\subsection{Actividad 5}

\subsubsection{Enunciado}
Según tu visión, ¿cuál crees que es la causa o fuente por las que surgen más conflictos laborales? Nómbrala y razona tus respuestas.

\subsubsection{Solución}
Pienso que la \textbf{principal causa de conflicto} puede ser por la \textbf{diferencia de opiniones y posiciones}, ya que para resolver este tipo de conflictos una cualidad muy importante es la \textbf{asertividad}, y creo que es no es, por desgracia, una cualidad muy extendida entre la mayoría de la población.

\subsection{Actividad 6}

\subsubsection{Enunciado}
Una vez trabajado el conflicto, nombra las fases por las que discurre.

\subsubsection{Solución}
Las \textbf{fase} por las que pasa un \textbf{conflicto} son las siguientes:

\begin{enumerate}
    \item \textbf{Fase Inicial}: existen una situación conflictiva latente.
    \item \textbf{Fase de Aceptación}: el conflicto ``explota'', generando situaciones como hostilidades, deterioro de la comunicación, toma de posiciones, etc...
    \item \textbf{Fase de Tratamiento del Conflicto}: en esta fase se debe actuar intentado dominar la situación usando las tácticas adecuadas por parte del jefe de equipo, las cuales deben llevar a la resolución positiva del conflicto.
    \item \textbf{Fase de Análisis y Evaluación}: en este última fase de evalúa las consecuencias que ha tenido el conflicto en el equipo.
\end{enumerate}

\subsection{Actividad 7}

\subsubsection{Enunciado}
Respecto a las tácticas de solución de conflicto, haz una enumeración ordenada, de más a menos activas, de las tácticas de las que se puede valer un/a jefe/a

\subsubsection{Solución}
Las \textbf{tácticas de solución} de conflictos, ordenadas de más a menos activas, son las siguientes:

\begin{enumerate}
    \item \textbf{Confrontación}
    \item \textbf{Supresión}
    \item \textbf{Compromiso}
    \item \textbf{Suavización}
    \item \textbf{Evitación}
\end{enumerate}




% Bibliography

\newpage
\bibliography{citas}
\bibliographystyle{unsrt}

\end{document}