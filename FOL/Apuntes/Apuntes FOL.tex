% % % % % % % % % % % % % % % % % % % % % % % % % % % % % % % % % % % % % % % % % % % %
%                                                                                     %
% Short Sectioned Assignment LaTeX Template Version 1.0 (5/5/12)                      %
% This template has been downloaded from: http://www.LaTeXTemplates.com               %
%                                                                                     %
% Original author:  Frits Wenneker (http://www.howtotex.com)                          %
%                                                                                     %
% Modified by: Fco Javier Sueza Rodríguez (fcosueza@disroot.org)                      %
%                                                                                     %
% Changes:                                                                            %
%	    - Custom Chapters, Sections and Subsections (titlesec package)                %
%           - Document type scrbook (oneside)                                         %
%           - Use babel-lang-spanish package and marvosym                             %
%           - Use hyperref, enumitem, tcolorbox and glossaries packages               %
%           - Use Time New Roman (mathptmx), Helvetic and Courier fonts               %
%                                                                                     %
% License: CC BY-NC-SA 3.0 (http://creativecommons.org/licenses/by-nc-sa/3.0/)        %
%                                                                                     %
% % % % % % % % % % % % % % % % % % % % % % % % % % % % % % % % % % % % % % % % % % % %

%-----------------------------------------------%
%	              Packages                  %
%-----------------------------------------------%

\documentclass[paper=a4, fontsize=11pt, oneside]{scrbook}

% ---- Text Input/Output ----- %

\usepackage[T1]{fontenc}
\usepackage[utf8]{inputenc}
\usepackage{mathptmx}
\usepackage[scaled=.92]{helvet}
\usepackage{courier}
\usepackage[indent=12pt]{parskip}

\usepackage{geometry}
\geometry{verbose,tmargin=3cm,bmargin=3cm,lmargin=2.6cm,rmargin=2.6cm}

% ---- Language ----- %

\usepackage[spanish]{babel}
\usepackage{marvosym}

% ---- Another packages ---- %

\usepackage{amsmath,amsfonts,amsthm}
\usepackage{graphics,graphicx}
\usepackage{titlesec}
\usepackage{fancyhdr}
\usepackage{tcolorbox}
\usepackage{hyperref}
\usepackage{enumitem}
\usepackage[automake]{glossaries}

%--------------------------------------------------------------------%
%                      Customizing Document                          %
%--------------------------------------------------------------------%


% ----------- Custom Chapters, Sections and Subsections -------------- %

\titleformat{\chapter}[display]
			{\bfseries\Huge}
			{Tema \ \thechapter} {0.5ex}
			{\vspace{1ex}\centering}

\titleformat{\section}[hang]
			{\bfseries\Large}
			{\thesection}{0.5em}{}

\titleformat{\subsection}[hang]
			{\bfseries\large}
			{\thesubsection}{0.5em}{}

\titleformat{\subsubsection}[hang]
			{\bfseries\large}
			{\thesubsubsection}{0.5em}{}

\hypersetup{
    colorlinks=true,
    linkcolor=black,
    urlcolor=magenta
}

% ------------------- Custom heaaders and footers ------------------- %

\pagestyle{fancyplain}

\fancyhead[]{}
\fancyfoot[L]{}
\fancyfoot[C]{}
\fancyfoot[R]{\thepage}

\renewcommand{\headrulewidth}{0pt} % Remove header underlines
\renewcommand{\footrulewidth}{0pt} % Remove footer underlines

\setlength{\headheight}{13.6pt} % Customize the height of the header

% --------- Numbering equations, figures and tables ----------------- %

\numberwithin{equation}{section} % Number equations within sections
\numberwithin{figure}{section} % Number figures within sections
\numberwithin{table}{section} % Number tables within sections

% ------------------------ New Commands ----------------------------- %

\newcommand{\horrule}[1]{\rule{\linewidth}{#1}} % Create horizontal rule command


%----------------------------------------------------------------------------------------
%	TÍTULO Y DATOS DEL ALUMNO
%----------------------------------------------------------------------------------------

\title{
\normalfont \normalsize
\textsc{{\bfseries Curso 2022-2023} \\ Ciclo Superior de Desarrollo de Aplicaciones Web \\ IES Aguadulce} \\ [25pt]
\horrule{0.5pt} \\[0.4cm]
\huge Formación y Orientación Laboral \\
\horrule{0.5pt} \\[0.4cm]
}

\author{Francisco Javier Sueza Rodríguez}
\date{\normalsize\today}

%----------------------------------------------------------------------------------------
%                                     DOCUMENTO
%----------------------------------------------------------------------------------------

\makeglossaries
\loadglsentries{glossary}

\begin{document}

\maketitle

\newpage

\tableofcontents

%\listoffigures

%\listoftables

\newpage

\chapter{Búsqueda de Empleo}
En este tema vamos a ver en que consiste la búsqueda de empleo y que herramientas y técnicas tenemos a nuestra disposición para hacer que este proceso tenga mas probabilidades de éxito. En este aspecto hay que poner de relieve la importancia de la {\bfseries auto-orientación}, definida como:

<<El proceso por el cuál una persona se dota de los instrumentos y la formación necesaria para elaborar alternativas profesionales, evaluando y eligiendo aquella que se considere mejor para nuestra carrera profesional>>\cite{apuntes01}

Además, estudiaremos las salidas laborales de las titulaciones de DAW, DAM y ASIR, el perfil y la carrera profesional de estos ciclos formativos y la importancia de la formación continua dentro del desarrollo profesional.

\section{Los Ciclos Formativos de DAW, DAM y ASIR}
Los Ciclos Formativos de {\bfseries DAW} (Desarrollo de Aplicaciones Web), {\bfseries DAM} (Desarrollo de Aplicaciones Multiplataforma) y {\bfseries ASIR} (Administración de Sistemas Informáticos en Red), pertenecientes de la familia de Informática y Telecomunicaciones, son Ciclos Formativos de Grado Superior, enmarcados dentro de la enseñanza superior.

Estos títulos capacitan para el desempeño de una profesión tanto por cuenta propia como por cuenta ajena, en el sector público o en el privado, en el ámbito TIC para el que esta pensado cada título. Cada uno de estos ciclos esta especializado en un área concreta, a saber:

 \begin{itemize}
 	\item {\bfseries DAM}: centrado en el área de desarrollo de aplicaciones multiplataforma en diferentes ámbitos, como gestión empresarial, ocio, dispositivos móviles,..etc
 	\item {\bfseries DAW}: que capacita para desempeñar un trabajo en el área de desarrollo en entornos Web (intranet, internet y/o extranet)
 	\item {\bfseries ASIR}: este ciclo se centra en la gestión y administración de datos y la infraestructura de red de las empresas.
 \end{itemize}

\subsection{El Título}
 Los Ciclos Formativos de Grado Superior están enmarcados dentro de la Educación Superior del sistema educativo español y por lo tanto regulados por la {\bfseries Ley Orgánica de Educación}, promulgada en 2006 y modificada en 2020.

 Más concretamente, el título de {\bfseries ASIR} fue aprobado por el {\bfseries Real Decreto 1629/2009}, publicado en el BOE el miércoles 18 de noviembre de 2009. El título de {\bfseries DAM} fue aprobado en el {\bfseries Real Decreto 450/2010} y publicado en el BOE el jueves 20 de mayo. Por último, el título de {\bfseries DAW} fue aprobado por el {\bfseries Real Decreto 686/2010} y publicado en el BOE el sábado 12 de junio de 2010.



 Los elementos básicos que identifican a estos títulos son los siguientes:

 \begin{enumerate}
    \item {\bfseries Denominación}: {\bfseries Desarrollo de Aplicaciones Multiplataforma}, {\bfseries Desarrollo de Aplicaciones Web} y {\bfseries Administración de Sistemas Informáticos en Red}
    \item {\bfseries Nivel}: Formación Profesional de Grado Superior
    \item {\bfseries Duración}; 2000 horas
    \item {\bfseries Familia Profesional}: Informática y Telecomunicaciones
    \item {\bfseries Referente Europeo}: CINE-5b (Clasificación Internacional Normalizada de Educación)
\end{enumerate}

 Estos títulos tiene validez en todo el territorio nacional, independientemente de las diferencias en su desarrollo entre las diferentes comunidades autónomas.

 Los ciclos formativos están compuestos de asignaturas, que en la formación profesional de denominan {\bfseries módulos profesionales}. Los diferentes módulos que conforman los títulos de DAM, ASIR y DAM son los siguientes:

\begin{center}
 \begin{table}[ht]
    {\renewcommand{\arraystretch}{1.5}
        \begin{tabular}[c]{ |l|l| }
            \hline
            \multicolumn{2}{|c|}{{\bfseries Desarrollo de Aplicaciones Web}} \\ \hline
            Sistemas Informáticos & Desarrollo Web en Entorno Servidor \\ \hline
            Bases de Datos & Despliegue de Aplicaciones Web \\ \hline
            Programación & Empresa e Iniciativa Emprendedora \\ \hline
            Lenguajes de Marcas y Sistemas de Gestión de la Información & Proyecto de Desarrollo Web\\ \hline
            Entornos de Desarrollo & Formación y Orientación Laboral \\ \hline
            Desarrollo en el Entorno Cliente & Formación en Centros de trabajo \\ \hline
            Desarrollo de Interfaces WEB &  \\ \hline
    \end{tabular}}
 \end{table}

 \begin{table}[ht]
    {\renewcommand{\arraystretch}{1.5}
        \begin{tabular}[c]{ |l|l| }
            \hline
            \multicolumn{2}{|c|}{{\bfseries Desarrollo de Aplicaciones Multiplataforma}} \\ \hline
            Sistemas Informáticos &  Entornos de Desarrollo\\ \hline
            Bases de Datos & Desarrollo de Interfaces \\ \hline
            Programación & Sistemas de Gestión Empresarial \\ \hline
            Lenguajes de Marcas y Sistemas de Gestión de la Información & Proyecto de Desarrollo Multiplataforma \\ \hline
            Programación de Servicios y Procesos & Empresa e Iniciativa Emprendedora \\ \hline
            Acceso a Datos & Formación y Orientación Laboral \\ \hline
            Programación Multimedia y Dispositivos Móviles & Formación en Centro de Trabajo \\ \hline
    \end{tabular}}
 \end{table}


 \begin{table}[ht]
    {\renewcommand{\arraystretch}{1.5}
        \begin{tabular}[c]{ |l|l| }
            \hline
            \multicolumn{2}{|c|}{{\bfseries Administración de Sistemas Informáticos en Red}} \\ \hline
            Implantación de Sistemas Operativos &  Implantación de Aplicaciones Web\\ \hline
            Administración de Sistemas Gestores de Bases de Datos & Planificación y administración de Redes \\ \hline
            Fundamentos de Hardware & Seguridad y Alta Disponibilidad \\ \hline
            Lenguajes de Marcas y Sistemas de Gestión de la Información & Formación y Orientación Laboral \\ \hline
            Gestión de Bases de Datos & Proyecto de Administración de Sistemas \\ \hline
            Acceso a Datos & Empresa e Iniciativa emprendedora \\ \hline
            Servicios de Red e Internet & Formación en Centro de Trabajo \\ \hline
    \end{tabular}}
 \end{table}
\end{center}

Para información sobre la normativa que regula los títulos de de Formación Profesional y en concreto los visto en esta sección podemos consultar las siguientes enlaces:

\begin{itemize}
    \item \href{https://www.boe.es/buscar/doc.php?id=BOE-A-2006-7899}{Ley Orgánica de Educación}
    \item \href{https://www.boe.es/boe/dias/2010/05/20/pdfs/BOE-A-2010-8067.pdf}{Real Decreto 686/2010} - {\bfseries DAM}
    \item \href{https://www.boe.es/boe/dias/2010/06/12/pdfs/BOE-A-2010-9269.pdf}{Real Decreto 450/2010} - {\bfseries DAW}
    \item \href{https://www.boe.es/boe/dias/2009/11/18/pdfs/BOE-A-2009-18355.pdf}{Real Decreto 1629/2009} - {\bfseries ASIR}
\end{itemize}

\subsection{El futuro de tus estudios}
En la sociedad actual, cada vez es mas necesario para las empresas el acceso y la organización de la información. Para llevar esto a cabo, se necesitan aplicaciones que permitan gestionar dicha información de manera íntegra. Así mismo, el uso cada vez mas extendido de dispositivos electrónicos como PDAs, móviles, tablets y el acceso a internet de la mayoría de la población demanda la creación de aplicaciones especificas.

El perfil de {\bfseries desarrollador de software} (multiplataforma o web) se encarga de la creación de estas aplicaciones, proporcionando una mayor integración de los sistema de gestión e intercambio de información en las diferentes plataformas así como asegurando la integridad, consistencia y accesibilidad de los datos empleados. También es trabajo del desarrollador tener en cuenta parámetros como la usabilidad, que facilitan la interacción del usuario con la aplicaciones, y la adaptación a nuevas técnicas y entornos de desarrollo específicos para la aplicaciones que se quieren desarrollar.

En el caso del {\bfseries administrador de sistemas informáticos en red}, su labor consiste en una mayor integración, en la pequeña y mediana empresa, de la integración de los sistema de gestión de la información	así como la intervención en sistemas informáticos destinados a la producción y el apoyo al resto de departamentos de una organización.

Por lo tanto, todo parece indicar que el futuro de la profesión esta garantizado, ya que cada vez se necesitan mas profesionales con perfiles técnicos y específicos debido a que los requerimientos cada vez de vuelven mas complejos y se precisan personas mejor preparadas.

Si quieres ampliar la información, puedes consultar los siguientes enlaces:

\begin{itemize}
    \item \href{https://administracionelectronica.gob.es/pae_Home?_nfpb=true&_pageLabel=P1200733131296129097704&langPae=es#faq1}{Centro de Transferencia de Tecnología}
    \item \href{https://educacionadistancia.juntadeandalucia.es/formacionprofesional/mod/scorm/player.php?a=6198&scoid=178876&currentorg=eXe68448_2zip5d5bab14269131fb5c2&mode=&attempt=1}{Cursos de Especialización FP}
\end{itemize}

\begin{tcolorbox}[sharp corners, colback=green!20, colframe=magenta!90, title={\bfseries {\Large Recuerda que...}}]
    La profesión evoluciona hacia:
    \begin{itemize}
        \item Una mayor integración de los sistemas de gestión en intercambio de información basados en diferentes plataformas y tecnologías.
        \item Una mayor demanda de asistencia técnica a través de la tele-operación, asistencia técnica remota y asistencia <<on line>>.
        \item La adaptación de los desarrolladores a nuevas técnicas y entornos de desarrollo por el aumento en el consumo de teléfonos, PDA y dispositivos móviles.
        \item El apoyo a otros departamentos dentro de la empresa asegurando la funcionalidad y rentabilidad del sistema informático.x
    \end{itemize}
\end{tcolorbox}

\subsection{Los técnicos superiores en los centros de trabajo}
Los Técnicos Superiores en DAW, DAM y ASIR desempeñan su labor tanta en la empresa pública como en la privada, aunque lo más común es que sea en el sector privado donde desarrollan su actividad.

Tanto los titulados en {\bfseries DAM} como en {\bfseries DAW} desempeñan su trabajo en el área de desarrollo de aplicaciones en diversos ámbitos, estando el primero mas enfocado al desarrollo en diferentes plataformas incluyendo los dispositivos móviles mientras en segundo se centra más en la plataforma web. Respecto a los titulados en {\bfseries ASIR}, su labor consiste en el aseguramiento de la funcionalidad y rentabilidad del sistema informático, asistencia local y remota,..etc.

Aunque no es lo más frecuente, los Técnicos de DAM, DAW y ASIR pueden desempeñar sus funciones en las administración pública estatal, autonómica o loca. Aunque no siempre es necesario tener aprobada una oposición, esta es la vía más común.

Según un estudio publicado por Infoempleo, los profesionales que han cursado estudios de grado superior de FP son mas demandados que los que han cursado grado medio, con un porcentaje de 57\% y 47\% respectivamente. Estos datos de inserción son muy positivos e indican que los perfiles se adecuan a los actuales requerimientos del mercado laboral.\cite{educaweb}

\subsection{El nivel académico}
Los títulos de Técnico Superior (DAW, DAM y ASIR) forman parte del Sistema Educativo Español promulgado por la Ley Orgánica de Educación y se enmarcan dentro de las Enseñanzas Superiores y de la Formación Profesional de Grado Superior. Por lo tanto, se trata de unos {\bfseries estudios superiores}, solo por detrás de los estudios universitarios.

Para acceder a un Grado Superior es necesario cumplir alguna de los siguientes requisitos:
\begin{enumerate}[label={\alph*.}]
    \item Estar en posesión del título de Bachiller
    \item Poseer el título de Técnico de grado medio y haber superado un curso de formación específico para el acceso a ciclos de grado superior en centros públicos o privados autorizados por la administración educativa
    \item Haber superado una prueba de acceso. En este supuesto, se requerirá tener diecinueve años, cumplidos en el año de realización de la prueba o dieciocho si se acredita estar en posesión de un título de Técnico relacionado con aquél al que se desea acceder
    \item Haber superado la Prueba de Acceso de la Universidad para mayores de 25, se requerirá tener cumplidos 25 años en el momento de realización de la prueba.
\end{enumerate}

Para más información sobre los requisitos de acceso, cupos y equivalencias, se puede consultar la pagina de la \href{https://www.juntadeandalucia.es/educacion/portals/web/formacion-profesional-andaluza/quiero-formarme/ensenanzas/fp-grado-superior/requisitos}{Consejería de Educación} de la Junta de Andalucía.

Es conveniente reseñar que dado el nivel superior de estudios, el nivel {\bfseries \gls{actitudinal}}, {\bfseries \gls{procedimental}} y {\bfseries \gls{conceptual}} requerido corresponderá con ello, y todo el proceso de enseñanza de ve influenciado por esta característica.

Cabe también destacar que el nivel académico superior posibilita un {\bfseries sistema de convalidaciones} de módulos del Ciclo Formativo con asignaturas o materias de de enseñanza universitarias, basado en un sistema de créditos ECTS y en los acuerdos entre las Consejerías de Educación y las Universidades.

En el apartado 3 de la Disposición adicional primera, "Colaboración entre la formación profesional superior y la enseñanza universitaria", la \href{https://www.boe.es/boe/dias/2011/03/12/pdfs/BOE-A-2011-4551.pdf}{Ley Orgánica 4/2011}, de 11 de marzo, complementaria de la Ley de Economía Sostenible, podrás comprobar la normativa más actual en relación a las convalidaciones de módulos y materias universitarias.

\subsection{El nivel profesional}
Los Ciclos Formativos de Formación profesional tienen como objetivo desarrollar las competencias generales correspondientes a la {\bfseries \gls{cualificacion}} o cualificaciones objetos de estudio en dicho título.

Desde un punto de vista profesional, la cualificación es un conjunto de \gls{competencias} profesionales (conocimientos y capacidades) que dan respuesta a ocupaciones y puestos de trabajo con valor en el mercado laboral, y que pueden adquirirse mediante la formación o la experiencia laboral. Las cualificaciones profesionales son el referente de los títulos de Formación Profesional.

El conjunto de cualificaciones profesionales vienen recogidas en el Catálogo Nacional de Cualificaciones Profesionales ({\bfseries CNCP}) que determina el Instituto Nacional de Cualificaciones ({\bfseries INCUAL}) a través de un proceso de análisis productivo.

Las cualificaciones se clasifican en 5 niveles profesionales en función de la complejidad y destreza en los conocimientos y capacidades en las competencias que se alcanzan. En concreto, los {\bfseries Técnicos Superiores} se enmarcan dentro del {\bfseries nivel 3} de esta clasificación que se define de la siguientes forma:

<<Competencia en un conjunto de actividades profesionales que requieren el dominio de diversas técnicas y puede ser ejecutado de forma autónoma. Comporta responsabilidad de coordinación y supervisión de trabajo técnico y especializado. Exige la comprensión de los fundamentos técnicos y científicos de las actividades y la evaluación de los factores del proceso y de sus repercusiones económicas.>>

Toda esta teoría se aplica tanto al sector público como al privado.

En el {\bfseries sector público} se establece una clasificación de los funcionarios en base a la LEY 7/2007, de 12 de abril, del Estatuto Básico del Empleado Público, donde se reconocen los siguientes grupos:

\begin{itemize}
    \item {\bfseries Grupo A}: Graduados Universitarios
    \item {\bfseries Grupo B}: Técnicos Superiores de la FP
    \item {\bfseries Grupo C1}: Técnicos de FP y Bachilleres
    \item {\bfseries Grupo C2}; Titulados en ESO
\end{itemize}

Tanto en el {\bfseries sector privado} como sí el trabajador de la administración pública no es personal funcionario, la clasificación se rige por los {\bfseries \gls{convenios colectivos}}, que se tratarán con mas profundidad en otros temas.

\section{Salidas Profesionales}








% Glossary

\glsaddall
\printglossaries

% Bibliography

\newpage
\addcontentsline{toc}{chapter}{Bibliografía}
\bibliography{citas}
\bibliographystyle{unsrt}

\end{document}