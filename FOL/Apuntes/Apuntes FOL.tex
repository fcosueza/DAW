% % % % % % % % % % % % % % % % % % % % % % % % % % % % % % % % % % % % % % % % % % % %
%                                                                                     %
% Short Sectioned Assignment LaTeX Template Version 1.0 (5/5/12)                      %
% This template has been downloaded from: http://www.LaTeXTemplates.com               %
%                                                                                     %
% Original author:  Frits Wenneker (http://www.howtotex.com)                          %
%                                                                                     %
% Modified by: Fco Javier Sueza Rodríguez (fcosueza@disroot.org)                      %
%                                                                                     %
% Changes:                                                                            %
%	    - Custom Chapters, Sections and Subsections (titlesec package)                %
%           - Document type scrbook (oneside)                                         %
%           - Use babel-lang-spanish package and marvosym                             %
%           - Use hyperref, enumitem, tcolorbox and glossaries packages               %
%           - Use Time New Roman (mathptmx), Helvetic and Courier fonts               %
%                                                                                     %
% License: CC BY-NC-SA 3.0 (http://creativecommons.org/licenses/by-nc-sa/3.0/)        %
%                                                                                     %
% % % % % % % % % % % % % % % % % % % % % % % % % % % % % % % % % % % % % % % % % % % %

%-----------------------------------------------%
%	              Packages                  %
%-----------------------------------------------%

\documentclass[paper=a4, fontsize=11pt, oneside]{scrbook}

% ---- Text Input/Output ----- %

\usepackage[T1]{fontenc}
\usepackage[utf8]{inputenc}
\usepackage{mathptmx}
\usepackage[scaled=.92]{helvet}
\usepackage{courier}
\usepackage[indent=12pt]{parskip}

\usepackage{geometry}
\geometry{verbose,tmargin=3cm,bmargin=3cm,lmargin=2.6cm,rmargin=2.6cm}

% ---- Language ----- %

\usepackage[spanish]{babel}
\usepackage{marvosym}

% ---- Another packages ---- %

\usepackage{amsmath,amsfonts,amsthm}
\usepackage{graphics,graphicx}
\usepackage{titlesec}
\usepackage{fancyhdr}
\usepackage{tcolorbox}
\usepackage{hyperref}
\usepackage{enumitem}
\usepackage[automake]{glossaries}

%--------------------------------------------------------------------%
%                      Customizing Document                          %
%--------------------------------------------------------------------%


% ----------- Custom Chapters, Sections and Subsections -------------- %

\titleformat{\chapter}[display]
			{\bfseries\Huge}
			{Tema \ \thechapter} {0.5ex}
			{\vspace{1ex}\centering}

\titleformat{\section}[hang]
			{\bfseries\Large}
			{\thesection}{0.5em}{}

\titleformat{\subsection}[hang]
			{\bfseries\large}
			{\thesubsection}{0.5em}{}

\titleformat{\subsubsection}[hang]
			{\bfseries\large}
			{\thesubsubsection}{0.5em}{}

\hypersetup{
    colorlinks=true,
    linkcolor=black,
    urlcolor=magenta
}

% ------------------- Custom heaaders and footers ------------------- %

\pagestyle{fancyplain}

\fancyhead[]{}
\fancyfoot[L]{}
\fancyfoot[C]{}
\fancyfoot[R]{\thepage}

\renewcommand{\headrulewidth}{0pt} % Remove header underlines
\renewcommand{\footrulewidth}{0pt} % Remove footer underlines

\setlength{\headheight}{13.6pt} % Customize the height of the header

% --------- Numbering equations, figures and tables ----------------- %

\numberwithin{equation}{section} % Number equations within sections
\numberwithin{figure}{section} % Number figures within sections
\numberwithin{table}{section} % Number tables within sections

% ------------------------ New Commands ----------------------------- %

\newcommand{\horrule}[1]{\rule{\linewidth}{#1}} % Create horizontal rule command


%----------------------------------------------------------------------------------------
%	TÍTULO Y DATOS DEL ALUMNO
%----------------------------------------------------------------------------------------

\title{
\normalfont \normalsize
\textsc{{\bfseries Curso 2022-2023} \\ Ciclo Superior de Desarrollo de Aplicaciones Web \\ IES Aguadulce} \\ [25pt]
\horrule{0.5pt} \\[0.4cm]
\huge Formación y Orientación Laboral \\
\horrule{0.5pt} \\[0.4cm]
}

\author{Francisco Javier Sueza Rodríguez}
\date{\normalsize\today}

%----------------------------------------------------------------------------------------
%                                     DOCUMENTO
%----------------------------------------------------------------------------------------

\begin{document}

\maketitle

\newpage

\tableofcontents

%\listoffigures

%\listoftables

\newpage

\chapter{Búsqueda de Empleo}
En este tema vamos a ver en que consiste la búsqueda de empleo y que herramientas y técnicas tenemos a nuestra disposición para hacer que este proceso tenga mas probabilidades de éxito. En este aspecto hay que poner de relieve la importancia de la {\bfseries auto-orientación}, definida como:

<<El proceso por el cuál una persona se dota de los instrumentos y la formación necesaria para elaborar alternativas profesionales, evaluando y eligiendo aquella que se considere mejor para nuestra carrera profesional>>\cite{apuntes01}

Además, estudiaremos las salidas laborales de las titulaciones de DAW, DAM y ASIR, el perfil y la carrera profesional de estos ciclos formativos y la importancia de la formación continua dentro del desarrollo profesional.

\section{Ciclos Formativos de DAW, DAM y ASIR}
Los Ciclos Formativos de {\bfseries DAW} (Desarrollo de Aplicaciones Web), {\bfseries DAM} (Desarrollo de Aplicaciones Multiplataforma) y {\bfseries ASIR} (Administración de Sistemas Informáticos en Red), pertenecientes de la familia de Informática y Telecomunicaciones, son Ciclos Formativos de Grado Superior, enmarcados dentro de la enseñanza superior.

Estos títulos capacitan para el desempeño de una profesión tanto por cuenta propia como por cuenta ajena, en el sector público o en el privado, en el ámbito TIC para el que esta pensado cada título. Cada uno de estos ciclos esta especializado en un área concreta, a saber:

 \begin{itemize}
 	\item {\bfseries DAM}: centrado en el área de desarrollo de aplicaciones multiplataforma en diferentes ámbitos, como gestión empresarial, ocio, dispositivos móviles,..etc
 	\item {\bfseries DAW}: que capacita para desempeñar un trabajo en el área de desarrollo en entornos Web (intranet, internet y/o extranet)
 	\item {\bfseries ASIR}: este ciclo se centra en la gestión y administración de datos y la infraestructura de red de las empresas.
 \end{itemize}

\section{El Título}
 Los Ciclos Formativos de Grado Superior están enmarcados dentro de la Educación Superior del sistema educativo español y por lo tanto regulados por la {\bfseries Ley Orgánica de Educación}, promulgada en 2006 y modificada en 2020.

 Más concretamente, el título de {\bfseries ASIR} fue aprobado por el {\bfseries Real Decreto 1629/2009}, publicado en el BOE el miércoles 18 de noviembre de 2009. El título de {\bfseries DAM} fue aprobado en el {\bfseries Real Decreto 450/2010} y publicado en el BOE el jueves 20 de mayo. Por último, el título de {\bfseries DAW} fue aprobado por el {\bfseries Real Decreto 686/2010} y publicado en el BOE el sábado 12 de junio de 2010.



 Los elementos básicos que identifican a estos títulos son los siguientes:

 \begin{enumerate}
    \item {\bfseries Denominación}: {\bfseries Desarrollo de Aplicaciones Multiplataforma}, {\bfseries Desarrollo de Aplicaciones Web} y {\bfseries Administración de Sistemas Informáticos en Red}
    \item {\bfseries Nivel}: Formación Profesional de Grado Superior
    \item {\bfseries Duración}; 2000 horas
    \item {\bfseries Familia Profesional}: Informática y Telecomunicaciones
    \item {\bfseries Referente Europeo}: CINE-5b (Clasificación Internacional Normalizada de Educación)
\end{enumerate}

 Estos títulos tiene validez en todo el territorio nacional, independientemente de las diferencias en su desarrollo entre las diferentes comunidades autónomas.

 Los ciclos formativos están compuestos de asignaturas, que en la formación profesional de denominan {\bfseries módulos profesionales}. Los diferentes módulos que conforman los títulos de DAM, ASIR y DAM son los siguientes:

\begin{center}
 \begin{table}[ht]
    {\renewcommand{\arraystretch}{1.5}
        \begin{tabular}[c]{ |l|l| }
            \hline
            \multicolumn{2}{|c|}{{\bfseries Desarrollo de Aplicaciones Web}} \\ \hline
            Sistemas Informáticos & Desarrollo Web en Entorno Servidor \\ \hline
            Bases de Datos & Despliegue de Aplicaciones Web \\ \hline
            Programación & Empresa e Iniciativa Emprendedora \\ \hline
            Lenguajes de Marcas y Sistemas de Gestión de la Información & Proyecto de Desarrollo \\ \hline
            Entornos de Desarrollo & Formación y Orientación Laboral \\ \hline
            Desarrollo en el Entorno Cliente & Formación en Centros de trabajo \\ \hline
            Desarrollo de Interfaces WEB &  \\ \hline
    \end{tabular}}
 \end{table}

 \begin{table}[ht]
    {\renewcommand{\arraystretch}{1.5}
        \begin{tabular}[c]{ |l|l| }
            \hline
            \multicolumn{2}{|c|}{{\bfseries Desarrollo de Aplicaciones Multiplataforma}} \\ \hline
            Sistemas Informáticos &  Entornos de Desarrollo\\ \hline
            Bases de Datos & Desarrollo de Interfaces \\ \hline
            Programación & Sistemas de Gestión Empresarial \\ \hline
            Lenguajes de Marcas y Sistemas de Gestión de la Información & Proyecto de Desarrollo \\ \hline
            Programación de Servicios y Procesos & Empresa e Iniciativa Emprendedora \\ \hline
            Acceso a Datos & Formación y Orientación Laboral \\ \hline
            Programación Multimedia y Dispositivos Móviles & Formación en Centro de Trabajo \\ \hline
    \end{tabular}}
 \end{table}


 \begin{table}[ht]
    {\renewcommand{\arraystretch}{1.5}
        \begin{tabular}[c]{ |l|l| }
            \hline
            \multicolumn{2}{|c|}{{\bfseries Administración de Sistemas Informáticos en Red}} \\ \hline
            Implantación de Sistemas Operativos &  Implantación de Aplicaciones Web\\ \hline
            Administración de Sistemas Gestores de Bases de Datos & Planificación y administración de Redes \\ \hline
            Fundamentos de Hardware & Seguridad y Alta Disponibilidad \\ \hline
            Lenguajes de Marcas y Sistemas de Gestión de la Información & Formación y Orientación Laboral \\ \hline
            Gestión de Bases de Datos & Proyecto de Administración de Sistemas \\ \hline
            Acceso a Datos & Empresa e Iniciativa emprendedora \\ \hline
            Servicios de Red e Internet & Formación en Centro de Trabajo \\ \hline
    \end{tabular}}
 \end{table}
\end{center}

Para información sobre la normativa que regula los títulos de de Formación Profesional y en concreto los visto en esta sección podemos consultar las siguientes enlaces:

\begin{itemize}
    \item \href{https://www.boe.es/buscar/doc.php?id=BOE-A-2006-7899}{Ley Orgánica de Educación}
    \item \href{https://www.boe.es/boe/dias/2010/05/20/pdfs/BOE-A-2010-8067.pdf}{Real Decreto 686/2010} - {\bfseries DAM}
    \item \href{https://www.boe.es/boe/dias/2010/06/12/pdfs/BOE-A-2010-9269.pdf}{Real Decreto 450/2010} - {\bfseries DAW}
    \item \href{https://www.boe.es/boe/dias/2009/11/18/pdfs/BOE-A-2009-18355.pdf}{Real Decreto 1629/2009} - {\bfseries ASIR}
\end{itemize}

\section{El futuro de tus estudios}










\bibliography{citas}
\bibliographystyle{plain}

\end{document}