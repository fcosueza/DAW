\input{preambulo.tex}

%----------------------------------------------------------------------------------------
%	TÍTULO Y DATOS DEL ALUMNO
%----------------------------------------------------------------------------------------

\title{	
\normalfont \normalsize 
\textsc{{\bfseries Curso 2022-2023} \\ Ciclo Superior de Desarrollo de Aplicaciones Web \\ IES Aguadulce} \\ [25pt] 
\horrule{0.5pt} \\[0.4cm] 
\huge Formación y Orientación Laboral \\ 
\horrule{0.5pt} \\[0.4cm] 
}

\author{Francisco Javier Sueza Rodríguez} 
\date{\normalsize\today} 

%----------------------------------------------------------------------------------------
%                                     DOCUMENTO
%----------------------------------------------------------------------------------------

\begin{document}

\maketitle

\newpage 

\tableofcontents 

%\listoffigures

%\listoftables

\newpage

\chapter{Búsqueda de Empleo}
En este tema vamos a ver en que consiste la búsqueda de empleo y que herramientas y técnicas tenemos a nuestra disposición para hacer que este proceso tenga mas probabilidades de éxito. En este aspecto hay que poner de relieve la importancia de la {\bfseries auto-orientación}, definida como:

 <<El proceso por el cuál una persona se dota de los instrumentos y la formación necesaria para elaborar alternativas profesionales, evaluando y eligiendo aquella que se considere mejor para nuestra carrera profesional>>\cite{apuntes01}
 
 Además, estudiaremos las salidas laborales de las titulaciones de DAW, DAM y ASIR, el perfil y la carrera profesional de estos ciclos formativos y la importancia de la formación continua dentro del desarrollo profesional.  
 
 \section{Ciclos Formativos de DAW, DAM y ASIR}
 Los Ciclos Formativos de {\bfseries DAW} (Desarrollo de Aplicaciones Web), {\bfseries DAM} (Desarrollo de Aplicaciones Multiplataforma) y {\bfseries ASIR} (Administración de Sistemas Informáticos en RED), pertenecientes de la familia de Informática y Telecomunicaciones
	



 

 
 
 
\bibliography{citas}  
\bibliographystyle{plain} 

\end{document}