\input{preambulo.tex}

%----------------------------------------------------------------------------------------
%	TÍTULO Y DATOS DEL ALUMNO
%----------------------------------------------------------------------------------------

\title{	
\normalfont \normalsize 
\textsc{{\bfseries Curso 2022-2023} \\ Ciclo Superior de Desarrollo de Aplicaciones Web \\ IES Aguadulce} \\ [25pt] 
\horrule{0.5pt} \\[0.4cm] 
\huge Formación y Orientación Laboral \\ 
\horrule{0.5pt} \\[0.4cm] 
}

\author{Francisco Javier Sueza Rodríguez} 
\date{\normalsize\today} 

%----------------------------------------------------------------------------------------
%                                     DOCUMENTO
%----------------------------------------------------------------------------------------

\begin{document}

\maketitle

\newpage 

\tableofcontents 

%\listoffigures

%\listoftables

\newpage

\chapter{Búsqueda de Empleo}
En este tema vamos a ver en que consiste la búsqueda de empleo y que herramientas y técnicas tenemos a nuestra disposición para hacer que este proceso tenga mas probabilidades de éxito. En este aspecto hay que poner de relieve la importancia de la {\bfseries auto-orientación}, definida como:

 <<El proceso por el cuál una persona se dota de los instrumentos y la formación necesaria para elaborar alternativas profesionales, evaluando y eligiendo aquella que se considere mejor para nuestra carrera profesional>>\cite{apuntes01}
 
 Además, estudiaremos las salidas laborales de las titulaciones de DAW, DAM y ASIR, el perfil y la carrera profesional de estos ciclos formativos y la importancia de la formación continua dentro del desarrollo profesional.  
 
 \section{Ciclos Formativos de DAW, DAM y ASIR}
 Los Ciclos Formativos de {\bfseries DAW} (Desarrollo de Aplicaciones Web), {\bfseries DAM} (Desarrollo de Aplicaciones Multiplataforma) y {\bfseries ASIR} (Administración de Sistemas Informáticos en RED), pertenecientes de la familia de Informática y Telecomunicaciones, son Ciclos Formativos de Grado Superior, enmarcados dentro de la enseñanza superior.
 
 Estos títulos capacitan para el desempeño de una profesión tanto por cuenta propia como por cuenta ajena, en el sector público o en el privado, en el ámbito TIC para el que esta pensado cada título. Cada uno de estos ciclos esta especializado en un área concreta, a saber: 
 
 \begin{itemize}
 	\item {\bfseries DAM}: centrado en el área de desarrollo de aplicaciones multiplataforma en diferentes ámbitos, como gestión empresarial, ocio, dispositivos móviles,..etc
 	\item {\bfseries DAW}: que capacita para desempeñar un trabajo en el área de desarrollo en entornos Web (intranet, internet y/o extranet)
 	\item {\bfseries ASIR}: este ciclo se centra en la gestión y administración de datos y la infraestructura de red de las empresas.
 \end{itemize}
	
\section{El Título}
Los Ciclos Formativos de Grado Superior están enmarcados dentro de la Educación Superior del sistema educativo español y por lo tanto regulados por la {\bfseries Ley Orgánica de Educación}, promulgada en 2006 y modificada en 2020.

Más concretamente, el título de {\bfseries ASIR} fue aprobado por el {\bfseries Real Decreto 1629/2009}, publicado en el BOE el miércoles 18 de noviembre de 2009. El título de {\bfseries DAM} fue aprobado en el {\bfseries Real Decreto 450/2010} y publicado en el BOE el jueves 20 de mayo. Por último, el título de {\bfseries DAW} fue aprobado por el {\bfseries Real Decreto 686/2010} y publicado en el BOE el sábado 12 de junio de 2010. 

Cada Comunidad Autónoma del Estado ha desarrollado los títulos a través de currículos propios, pudiendo incluir diferencias en los módulos entre las diferentes Comunidades Autónomas. Estos currículos se pueden consultar en la página de la Conserjería de Educación de cada comunidad.

Los elementos básicos que identifican a estos títulos son los siguientes: 

\begin{enumerate}
	\item {\bfseries Denominación}: {\bfseries Desarrollo de Aplicaciones Multiplataforma}, {\bfseries Desarrollo de Aplicaciones Web} y {\bfseries Administración de Sistemas Informáticos y en Red}
	\item {\bfseries Nivel}: Formación Profesional de Grado Superior
	\item {\bfseries Duración}; 2000 horas
	\item {\bfseries Familia Profesional}: Informática y Telecomunicaciones
	\item {\bfseries Referente Europeo}: CINE-5b (Clasificación Internacional Normalizada de Educación)
\end{enumerate}

Estos títulos tiene validez en todo el territorio nacional, independientemente de las diferencias en su desarrollo entre las diferentes comunidades autónomas.


 

 
 
 
\bibliography{citas}  
\bibliographystyle{plain} 

\end{document}