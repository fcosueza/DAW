\newglossaryentry{actitudinal}{
    name={actitudinal},
    description={Relativo a la actitud. <<Se puede definir una actitud como la tendencia o predisposición aprendida, más o menos generalizada y de tono afectivo, a responder de un modo bastante persistente y característico, por lo común positiva o negativamente (a favor o en contra), con referencia a una situación>>}
}

\newglossaryentry{procedimental}{
    name={procedimental},
    description={El conocimiento procedimental es el relacionado con cosas que sabemos hacer, pero no conscientemente, como por ejemplo montar en bicicleta o hablar nuestra lengua. El conocimiento procedimental se adquiere gradualmente a través de la práctica y está relacionado con el aprendizaje de las destrezas}
}

\newglossaryentry{conceptual}{
    name={conceptual},
    description={Relativo a concepto, que es una unidad cognitiva de significado, una idea abstracta o mental que a veces se define como una <<unidad de conocimiento>>}
}

\newglossaryentry{cualificacion}{
    name={cualificación},
    description={Conjunto de competencias profesionales (conocimientos y capacidades) que permiten dar respuesta a ocupaciones y puestos de trabajo con valor en el mercado laboral, y que pueden adquirirse a través de formación o por experiencia laboral}
}

\newglossaryentry{competencias}{
    name={competencias},
    description={Conjuntos de conocimientos, habilidades, disposiciones y conductas que posee una persona, que le permiten la realización exitosa de una actividad}
}

\newglossaryentry{convenios colectivos}{
    name={convenios colectivos},
    description={Un convenio colectivo es el acuerdo suscrito entre los representantes de los trabajadores y los representantes de los empresarios para fijar las condiciones de trabajo y productividad en un ámbito laboral determinado}
}

\newglossaryentry{perfil profesional}{
    name={perfil profesional},
    description={Conjunto de capacidades y competencias que identifican la formación de una persona para asumir en condiciones óptimas las responsabilidades propias del desarrollo de funciones y tareas de una determinada profesión}
}

