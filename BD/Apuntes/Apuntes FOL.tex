% % % % % % % % % % % % % % % % % % % % % % % % % % % % % % % % % % % % % % % % % % % %
%                                                                                     %
% Short Sectioned Assignment LaTeX Template Version 1.0 (5/5/12)                      %
% This template has been downloaded from: http://www.LaTeXTemplates.com               %
%                                                                                     %
% Original author:  Frits Wenneker (http://www.howtotex.com)                          %
%                                                                                     %
% Modified by: Fco Javier Sueza Rodríguez (fcosueza@disroot.org)                      %
%                                                                                     %
% Changes:                                                                            %
%	    - Custom Chapters, Sections and Subsections (titlesec package)                %
%           - Document type scrbook (oneside)                                         %
%           - Use babel-lang-spanish package and marvosym                             %
%           - Use hyperref, enumitem, tcolorbox and glossaries packages               %
%           - Use Time New Roman (mathptmx), Helvetic and Courier fonts               %
%                                                                                     %
% License: CC BY-NC-SA 3.0 (http://creativecommons.org/licenses/by-nc-sa/3.0/)        %
%                                                                                     %
% % % % % % % % % % % % % % % % % % % % % % % % % % % % % % % % % % % % % % % % % % % %

%-----------------------------------------------%
%	              Packages                  %
%-----------------------------------------------%

\documentclass[paper=a4, fontsize=11pt, oneside]{scrbook}

% ---- Text Input/Output ----- %

\usepackage[T1]{fontenc}
\usepackage[utf8]{inputenc}
\usepackage{mathptmx}
\usepackage[scaled=.92]{helvet}
\usepackage{courier}
\usepackage[indent=12pt]{parskip}

\usepackage{geometry}
\geometry{verbose,tmargin=3cm,bmargin=3cm,lmargin=2.6cm,rmargin=2.6cm}

% ---- Language ----- %

\usepackage[spanish]{babel}
\usepackage{marvosym}

% ---- Another packages ---- %

\usepackage{amsmath,amsfonts,amsthm}
\usepackage{graphics,graphicx}
\usepackage{titlesec}
\usepackage{fancyhdr}
\usepackage{tcolorbox}
\usepackage{hyperref}
\usepackage{enumitem}
\usepackage[automake]{glossaries}

%--------------------------------------------------------------------%
%                      Customizing Document                          %
%--------------------------------------------------------------------%


% ----------- Custom Chapters, Sections and Subsections -------------- %

\titleformat{\chapter}[display]
			{\bfseries\Huge}
			{Tema \ \thechapter} {0.5ex}
			{\vspace{1ex}\centering}

\titleformat{\section}[hang]
			{\bfseries\Large}
			{\thesection}{0.5em}{}

\titleformat{\subsection}[hang]
			{\bfseries\large}
			{\thesubsection}{0.5em}{}

\titleformat{\subsubsection}[hang]
			{\bfseries\large}
			{\thesubsubsection}{0.5em}{}

\hypersetup{
    colorlinks=true,
    linkcolor=black,
    urlcolor=magenta
}

% ------------------- Custom heaaders and footers ------------------- %

\pagestyle{fancyplain}

\fancyhead[]{}
\fancyfoot[L]{}
\fancyfoot[C]{}
\fancyfoot[R]{\thepage}

\renewcommand{\headrulewidth}{0pt} % Remove header underlines
\renewcommand{\footrulewidth}{0pt} % Remove footer underlines

\setlength{\headheight}{13.6pt} % Customize the height of the header

% --------- Numbering equations, figures and tables ----------------- %

\numberwithin{equation}{section} % Number equations within sections
\numberwithin{figure}{section} % Number figures within sections
\numberwithin{table}{section} % Number tables within sections

% ------------------------ New Commands ----------------------------- %

\newcommand{\horrule}[1]{\rule{\linewidth}{#1}} % Create horizontal rule command


%----------------------------------------------------------------------------------------
%	TÍTULO Y DATOS DEL ALUMNO
%----------------------------------------------------------------------------------------

\title{
\normalfont \normalsize
\textsc{{\bfseries Curso 2022-2023} \\ Ciclo Superior de Desarrollo de Aplicaciones Web \\ IES Aguadulce} \\ [25pt]
\horrule{0.5pt} \\[0.4cm]
\huge Formación y Orientación Laboral \\
\horrule{0.5pt} \\[0.4cm]
}

\author{Francisco Javier Sueza Rodríguez}
\date{\normalsize\today}

%----------------------------------------------------------------------------------------
%                                     DOCUMENTO
%----------------------------------------------------------------------------------------

\makeglossaries
\loadglsentries{glossary.tex}

\begin{document}

\maketitle

\newpage

\tableofcontents

%\listoffigures

%\listoftables

\newpage

\chapter{Búsqueda de Empleo}
En este tema vamos a ver en que consiste la búsqueda de empleo y que herramientas y técnicas tenemos a nuestra disposición para hacer que este proceso tenga mas probabilidades de éxito. En este aspecto hay que poner de relieve la importancia de la \textbf{auto-orientación}, definida como:

<<El proceso por el cuál una persona se dota de los instrumentos y la formación necesaria para elaborar alternativas profesionales, evaluando y eligiendo aquella que se considere mejor para nuestra carrera profesional>>\cite{apuntes01}

Además, estudiaremos las salidas laborales de las titulaciones de DAW, DAM y ASIR, el perfil y la carrera profesional de estos ciclos formativos y la importancia de la formación continua dentro del desarrollo profesional.

\section{Los Ciclos Formativos de DAW, DAM y ASIR}
Los Ciclos Formativos de \textbf{DAW} (Desarrollo de Aplicaciones Web), \textbf{DAM} (Desarrollo de Aplicaciones Multiplataforma) y \textbf{ASIR} (Administración de Sistemas Informáticos en Red), pertenecientes de la familia de Informática y Telecomunicaciones, son Ciclos Formativos de Grado Superior, enmarcados dentro de la enseñanza superior.

Estos títulos capacitan para el desempeño de una profesión tanto por cuenta propia como por cuenta ajena, en el sector público o en el privado, en el ámbito TIC para el que esta pensado cada título. Cada uno de estos ciclos esta especializado en un área concreta, a saber:

 \begin{itemize}
 	\item \textbf{DAM}: centrado en el área de desarrollo de aplicaciones multiplataforma en diferentes ámbitos, como gestión empresarial, ocio, dispositivos móviles,..etc
 	\item \textbf{DAW}: que capacita para desempeñar un trabajo en el área de desarrollo en entornos Web (intranet, internet y/o extranet)
 	\item \textbf{ASIR}: este ciclo se centra en la gestión y administración de datos y la infraestructura de red de las empresas.
 \end{itemize}

\subsection{El Título}
 Los Ciclos Formativos de Grado Superior están enmarcados dentro de la Educación Superior del sistema educativo español y por lo tanto regulados por la \textbf{Ley Orgánica de Educación}, promulgada en 2006 y modificada en 2020.

 Más concretamente, el título de \textbf{ASIR} fue aprobado por el \textbf{Real Decreto 1629/2009}, publicado en el BOE el miércoles 18 de noviembre de 2009. El título de \textbf{DAM} fue aprobado en el \textbf{Real Decreto 450/2010} y publicado en el BOE el jueves 20 de mayo. Por último, el título de \textbf{DAW} fue aprobado por el \textbf{Real Decreto 686/2010} y publicado en el BOE el sábado 12 de junio de 2010.



 Los elementos básicos que identifican a estos títulos son los siguientes:

 \begin{enumerate}
    \item \textbf{Denominación}: \textbf{Desarrollo de Aplicaciones Multiplataforma}, \textbf{Desarrollo de Aplicaciones Web} y \textbf{Administración de Sistemas Informáticos en Red}
    \item \textbf{Nivel}: Formación Profesional de Grado Superior
    \item \textbf{Duración}: 2000 horas
    \item \textbf{Familia Profesional}: Informática y Telecomunicaciones
    \item \textbf{Referente Europeo}: CINE-5b (Clasificación Internacional Normalizada de Educación)
\end{enumerate}

 Estos títulos tiene validez en todo el territorio nacional, independientemente de las diferencias en su desarrollo entre las diferentes comunidades autónomas.

 Los ciclos formativos están compuestos de asignaturas, que en la formación profesional de denominan \textbf{módulos profesionales}. Los diferentes módulos que conforman los títulos de DAM, ASIR y DAM son los siguientes:

\begin{center}
 \begin{table}[ht]
    {\renewcommand{\arraystretch}{1.5}
        \begin{tabular}[c]{ |l|l| }
            \hline
            \multicolumn{2}{|c|}{\textbf{Desarrollo de Aplicaciones Web}} \\ \hline
            Sistemas Informáticos & Desarrollo Web en Entorno Servidor \\ \hline
            Bases de Datos & Despliegue de Aplicaciones Web \\ \hline
            Programación & Empresa e Iniciativa Emprendedora \\ \hline
            Lenguajes de Marcas y Sistemas de Gestión de la Información & Proyecto de Desarrollo Web\\ \hline
            Entornos de Desarrollo & Formación y Orientación Laboral \\ \hline
            Desarrollo en el Entorno Cliente & Formación en Centros de trabajo \\ \hline
            Desarrollo de Interfaces WEB &  \\ \hline
    \end{tabular}}
 \end{table}

 \begin{table}[ht]
    {\renewcommand{\arraystretch}{1.5}
        \begin{tabular}[c]{ |l|l| }
            \hline
            \multicolumn{2}{|c|}{\textbf{Desarrollo de Aplicaciones Multiplataforma}} \\ \hline
            Sistemas Informáticos &  Entornos de Desarrollo\\ \hline
            Bases de Datos & Desarrollo de Interfaces \\ \hline
            Programación & Sistemas de Gestión Empresarial \\ \hline
            Lenguajes de Marcas y Sistemas de Gestión de la Información & Proyecto de Desarrollo Multiplataforma \\ \hline
            Programación de Servicios y Procesos & Empresa e Iniciativa Emprendedora \\ \hline
            Acceso a Datos & Formación y Orientación Laboral \\ \hline
            Programación Multimedia y Dispositivos Móviles & Formación en Centro de Trabajo \\ \hline
    \end{tabular}}
 \end{table}


 \begin{table}[ht]
    {\renewcommand{\arraystretch}{1.5}
        \begin{tabular}[c]{ |l|l| }
            \hline
            \multicolumn{2}{|c|}{\textbf{Administración de Sistemas Informáticos en Red}} \\ \hline
            Implantación de Sistemas Operativos &  Implantación de Aplicaciones Web\\ \hline
            Administración de Sistemas Gestores de Bases de Datos & Planificación y administración de Redes \\ \hline
            Fundamentos de Hardware & Seguridad y Alta Disponibilidad \\ \hline
            Lenguajes de Marcas y Sistemas de Gestión de la Información & Formación y Orientación Laboral \\ \hline
            Gestión de Bases de Datos & Proyecto de Administración de Sistemas \\ \hline
            Acceso a Datos & Empresa e Iniciativa emprendedora \\ \hline
            Servicios de Red e Internet & Formación en Centro de Trabajo \\ \hline
    \end{tabular}}
 \end{table}
\end{center}

Para información sobre la normativa que regula los títulos de de Formación Profesional y en concreto los visto en esta sección podemos consultar las siguientes enlaces:

\begin{itemize}
    \item \href{https://www.boe.es/buscar/doc.php?id=BOE-A-2006-7899}{Ley Orgánica de Educación}
    \item \href{https://www.boe.es/boe/dias/2010/05/20/pdfs/BOE-A-2010-8067.pdf}{Real Decreto 686/2010} - \textbf{DAM}
    \item \href{https://www.boe.es/boe/dias/2010/06/12/pdfs/BOE-A-2010-9269.pdf}{Real Decreto 450/2010} - \textbf{DAW}
    \item \href{https://www.boe.es/boe/dias/2009/11/18/pdfs/BOE-A-2009-18355.pdf}{Real Decreto 1629/2009} - \textbf{ASIR}
\end{itemize}

\subsection{El futuro de tus estudios}
En la sociedad actual, cada vez es mas necesario para las empresas el acceso y la organización de la información. Para llevar esto a cabo, se necesitan aplicaciones que permitan gestionar dicha información de manera íntegra. Así mismo, el uso cada vez mas extendido de dispositivos electrónicos como PDAs, móviles, tablets y el acceso a internet de la mayoría de la población demanda la creación de aplicaciones especificas.

El perfil de \textbf{desarrollador de software} (multiplataforma o web) se encarga de la creación de estas aplicaciones, proporcionando una mayor integración de los sistema de gestión e intercambio de información en las diferentes plataformas así como asegurando la integridad, consistencia y accesibilidad de los datos empleados. También es trabajo del desarrollador tener en cuenta parámetros como la usabilidad, que facilitan la interacción del usuario con la aplicaciones, y la adaptación a nuevas técnicas y entornos de desarrollo específicos para la aplicaciones que se quieren desarrollar.

En el caso del \textbf{administrador de sistemas informáticos en red}, su labor consiste en una mayor integración, en la pequeña y mediana empresa, de la integración de los sistema de gestión de la información	así como la intervención en sistemas informáticos destinados a la producción y el apoyo al resto de departamentos de una organización.

Por lo tanto, todo parece indicar que el futuro de la profesión esta garantizado, ya que cada vez se necesitan mas profesionales con perfiles técnicos y específicos debido a que los requerimientos cada vez de vuelven mas complejos y se precisan personas mejor preparadas.

Si quieres ampliar la información, puedes consultar los siguientes enlaces:

\begin{itemize}
    \item \href{https://administracionelectronica.gob.es/pae_Home?_nfpb=true&_pageLabel=P1200733131296129097704&langPae=es#faq1}{Centro de Transferencia de Tecnología}
    \item \href{https://educacionadistancia.juntadeandalucia.es/formacionprofesional/mod/scorm/player.php?a=6198&scoid=178876&currentorg=eXe68448_2zip5d5bab14269131fb5c2&mode=&attempt=1}{Cursos de Especialización FP}
\end{itemize}

\begin{tcolorbox}[sharp corners, colback=green!20, colframe=magenta!90, title=\textbf{\Large Recuerda que...}]
    La profesión evoluciona hacia:
    \begin{itemize}
        \item Una mayor integración de los sistemas de gestión en intercambio de información basados en diferentes plataformas y tecnologías.
        \item Una mayor demanda de asistencia técnica a través de la tele-operación, asistencia técnica remota y asistencia <<on line>>.
        \item La adaptación de los desarrolladores a nuevas técnicas y entornos de desarrollo por el aumento en el consumo de teléfonos, PDA y dispositivos móviles.
        \item El apoyo a otros departamentos dentro de la empresa asegurando la funcionalidad y rentabilidad del sistema informático.x
    \end{itemize}
\end{tcolorbox}

\subsection{Los técnicos superiores en los centros de trabajo}
Los Técnicos Superiores en DAW, DAM y ASIR desempeñan su labor tanto en la empresa pública como en la privada, aunque lo más común es que sea en el sector privado donde desarrollan su actividad.

Tanto los titulados en \textbf{DAM} como en \textbf{DAW} desempeñan su trabajo en el área de desarrollo de aplicaciones en diversos ámbitos, estando el primero mas enfocado al desarrollo en diferentes plataformas incluyendo los dispositivos móviles mientras en segundo se centra más en la plataforma web. Respecto a los titulados en \textbf{ASIR}, su labor consiste en el aseguramiento de la funcionalidad y rentabilidad del sistema informático, asistencia local y remota,..etc.

Aunque no es lo más frecuente, los Técnicos de DAM, DAW y ASIR pueden desempeñar sus funciones en las administración pública estatal, autonómica o local. Aunque no siempre es necesario tener aprobada una oposición, esta es la vía más común de acceso a estos puestos.

Según un estudio publicado por Infoempleo, los profesionales que han cursado estudios de grado superior de FP son mas demandados que los que han cursado grado medio, con un porcentaje de 57\% y 47\% respectivamente. Estos datos de inserción son muy positivos e indican que los perfiles se adecuan a los actuales requerimientos del mercado laboral.\cite{educaweb}

\subsection{El nivel académico}
Los títulos de Técnico Superior (DAW, DAM y ASIR) forman parte del Sistema Educativo Español promulgado por la Ley Orgánica de Educación y se enmarcan dentro de las Enseñanzas Superiores y de la Formación Profesional de Grado Superior. Por lo tanto, se trata de unos \textbf{estudios superiores}, solo por detrás de los estudios universitarios.

Para acceder a un Grado Superior es necesario cumplir alguna de los siguientes requisitos:
\begin{enumerate}[label={\alph*.}]
    \item Estar en posesión del título de Bachiller
    \item Poseer el título de Técnico de grado medio y haber superado un curso de formación específico para el acceso a ciclos de grado superior en centros públicos o privados autorizados por la administración educativa
    \item Haber superado una prueba de acceso. En este supuesto, se requerirá tener diecinueve años, cumplidos en el año de realización de la prueba o dieciocho si se acredita estar en posesión de un título de Técnico relacionado con aquél al que se desea acceder
    \item Haber superado la Prueba de Acceso de la Universidad para mayores de 25, se requerirá tener cumplidos 25 años en el momento de realización de la prueba.
\end{enumerate}

Para más información sobre los requisitos de acceso, cupos y equivalencias, se puede consultar la pagina de la \href{https://www.juntadeandalucia.es/educacion/portals/web/formacion-profesional-andaluza/quiero-formarme/ensenanzas/fp-grado-superior/requisitos}{Consejería de Educación} de la Junta de Andalucía.

Es conveniente reseñar que dado el nivel superior de estudios, el nivel \textbf{\gls{actitudinal}}, \textbf{\gls{procedimental}} y \textbf{\gls{conceptual}} requerido corresponderá con ello, y todo el proceso de enseñanza de ve influenciado por esta característica.

Cabe también destacar que el nivel académico superior posibilita un \textbf{sistema de convalidaciones} de módulos del Ciclo Formativo con asignaturas o materias de de enseñanza universitarias, basado en un sistema de créditos ECTS y en los acuerdos entre las Consejerías de Educación y las Universidades.

En el apartado 3 de la Disposición adicional primera, <<Colaboración entre la formación profesional superior y la enseñanza universitaria>>, la \href{https://www.boe.es/boe/dias/2011/03/12/pdfs/BOE-A-2011-4551.pdf}{Ley Orgánica 4/2011}, de 11 de marzo, complementaria de la Ley de Economía Sostenible, podrás consultar la normativa más actual en relación a las convalidaciones de módulos y materias universitarias.

\subsection{El nivel profesional}
Los Ciclos Formativos de Formación profesional tienen como objetivo desarrollar las competencias generales correspondientes a la \textbf{\gls{cualificacion}} o cualificaciones objetos de estudio en dicho título.

Desde un punto de vista profesional, la cualificación es un conjunto de \textbf{\gls{competencias}} profesionales (conocimientos y capacidades) que dan respuesta a ocupaciones y puestos de trabajo con valor en el mercado laboral, y que pueden adquirirse mediante la formación o la experiencia laboral. Las cualificaciones profesionales son el referente de los títulos de Formación Profesional.

El conjunto de cualificaciones profesionales vienen recogidas en el Catálogo Nacional de Cualificaciones Profesionales (\textbf{CNCP}) que determina el Instituto Nacional de Cualificaciones (\textbf{INCUAL}) a través de un proceso de análisis productivo.

Las cualificaciones se clasifican en 5 niveles profesionales en función de la complejidad y destreza en los conocimientos y capacidades en las competencias que se alcanzan. En concreto, los \textbf{Técnicos Superiores} se enmarcan dentro del\textbf{ nivel 3} de esta clasificación que se define de la siguientes forma:

<<Competencia en un conjunto de actividades profesionales que requieren el dominio de diversas técnicas y puede ser ejecutado de forma autónoma. Comporta responsabilidad de coordinación y supervisión de trabajo técnico y especializado. Exige la comprensión de los fundamentos técnicos y científicos de las actividades y la evaluación de los factores del proceso y de sus repercusiones económicas.>>

Toda esta teoría se aplica tanto al sector público como al privado.

En el \textbf{sector público} se establece una clasificación de los funcionarios en base a la Ley 7/2007, de 12 de abril, del Estatuto Básico del Empleado Público, donde se reconocen los siguientes grupos:

\begin{itemize}
    \item \textbf{Grupo A}: Graduados Universitarios
    \item \textbf{Grupo B}: Técnicos Superiores de la FP
    \item \textbf{Grupo C1}: Técnicos de FP y Bachilleres
    \item \textbf{Grupo C2}; Titulados en ESO
\end{itemize}

Tanto en el {\bfseries sector privado} como sí el trabajador de la administración pública no es personal funcionario, la clasificación se rige por los \textbf{\gls{convenios colectivos}}, que se tratarán con mas profundidad en otros temas.

\section{Salidas Profesionales}
El sector de la informática y las nuevas tecnologías esta en constante evolución. Los trabajadores en este sector suelen tener una gran vocación y creatividad, que les permite dar soluciones satisfactorias a los diferentes problemas que se les van planteando. Además, el contacto con el cliente es cada vez mas directo lo que hace que tengan que ser buenos comunicadores con gran capacidad de dialogo y negociación.

Como futuro profesional del sector debes conocer la peculiaridades del entorno laboral en que tendrás que desenvolverte, por eso en este apartado analizaremos las características del empleo en el sector de la informática y especialmente las salidas laborales que existen.

\subsection{Características Generales del Sector de la Informática}
Todos los sectores tienen unas características propias que los definen, el sector de la informática y las nuevas tecnologías no es menos. A continuación se listan las principales características de este sector:

\begin{itemize}
    \item La informática y las nuevas tecnologías son una realidad en nuestras vidas, por ello la \textbf{demanda de profesionales} que puedan ayudarnos en su uso y mantenimiento es una \textbf{constante} tanto en el ámbito empresarial como en el particular.
    \item Es un sector donde los cambios y la innovación son constantes, por ello se necesitan \textbf{profesionales} que estén en \textbf{continuo reciclaje} y sean capaces del \textbf{autoaprendizaje}. La formación continua es clave en el sector.
    \item La \textbf{empresa privada} es la principal \textbf{fuente de empleo} para el sector. Aunque la administración pública es consumidora del mismo, no suele convocar oposiciones para cubrir estos puestos, sino que contrata empresas externas.
    \item Es un \textbf{sector} que esta en \textbf{auge} ya que la mayoría de las empresa usan, en mayor o menor medida, nuevas tecnologías.
\end{itemize}

\subsection{Salidas Profesionales en el Sector Privado}
En el sector de la informática y las nuevas tecnologías la empresa privada es al que ofrece más posibilidades de colocación. A continuación mostramos una lista con los puestos mas relevantes que pueden ocupar los diferentes técnicos:

\begin{itemize}
    \item \textbf{Técnico Superior en de Aplicaciones Multiplataforma}:
    \begin{itemize}
        \item Desarrollador de aplicaciones informáticas para la gestión empresarial y de negocio.
        \item Desarrollador de aplicaciones de propósito general.
        \item Desarrollador de aplicaciones en el ámbito del entretenimiento y la informática móvil.
    \end{itemize}
    \item \textbf{Técnico Superior en Desarrollo de Aplicaciones Web}:
    \begin{itemize}
        \item Desarrollador de aplicaciones Web.
        \item Programador Web.
        \item Programador Multimedia.
    \end{itemize}
    \item \textbf{Técnico Superior en Administración de Aplicaciones en Red}:
    \begin{itemize}
        \item Técnico en administración de sistema.
        \item Responsable de informática.
        \item Técnico de servicio de internet.
        \item Técnico de servicios de mensajería electrónica.
        \item Personal de apoyo y soporte técnico.
        \item Técnico de teleasistencia.
        \item Técnico de administración de bases de datos.
        \item Técnico de redes.
        \item Supervisor de Sistemas.
        \item Técnico de servicios de comunicaciones.
        \item Técnico de entornos web.
    \end{itemize}
 \end{itemize}

Para todos estos puestos se buscan trabajadores que tengan capacidad de trabajo en equipo ya que la mayoría de proyectos se desarrollan en equipos que debe estar perfectamente coordinado. También se buscan personas flexibles y creativas, adaptables a los cambios tecnológicos y con facilidad de aprender nuevos lenguajes de programación.

\subsection{Salidas Profesionales en el Sector Público}
El cuerpo de \textbf{Técnicos Auxiliares de Informática} de la Administración del Estado se crea con carácter de cuerpo general interministerial en 1990 adscribiéndose al Ministerio de Administraciones Públicas. Este cuerpo esta clasificado dentro del \textbf{Grupo C1}, por lo que para acceder a estas plazas se necesita tener un título de Bachiller o equivalente. Los perfiles idóneos son los Técnicos Superiores en Desarrollo Multiplataforma, Desarrollo de Aplicaciones Web y Administradores de Sistemas en RED.

Las funciones desempeñadas por los miembros del \textbf{Cuerpo de Técnicos Auxiliares de Informática} son un parte esencial en el desarrollo y mantenimiento de los sistemas informáticos automatizados de la Administración General el Estado. Entre sus funciones se encuentra el análisis y programación de aplicaciones, apoyo a los usuarios,  mantenimiento de hardware, instalación de equipos y sistemas, operación de sistemas de grandes bases de datos,.. entre otros.

Es una opción interesante que también se puede contemplar, aunque hay muchas más posibilidades de encontrar empleo en el sector privado.

\subsection{El Autoempleo}
Una alternativa al trabajo por cuenta ajena es el \textbf{autoempleo}, montar tu propia empresa, organizarte como emprendedor/a.

Antes de continuar con el desarrollo de esta sección cabe destacar que en el currículo de estos módulos profesionales se incluye el módulo \textbf{Empresa e Iniciativa Emprendedora}, destinado específicamente al desarrollo de las ideas del autoempleo, por lo que que vamos a ver en esta sección se verá ampliado y desarrollado en este módulo.

Las posibilidades de organizar una empresa en el ámbito de la informática y las TIC presenta una serie de oportunidades que podemos resumir en las siguientes:

\begin{itemize}
    \item El \textbf{capital inicial} que se precisa es, en términos relativos, bajo.
    \item En uso de las tecnologías por parte de las empresas y particulares hace que haya una \textbf{demanda importante de técnicos} que puedan resolver de forma rápida y eficiente los problemas que van surgiendo con el uso de estas tecnologías.
    \item Existe la posibilidad de acceder a \textbf{financiación pública} a través de subvenciones y ayudas de las distintas administraciones.
\end{itemize}

Las oportunidades son importantes, no obstante hay que saber valorar que no basta con que existan buenas oportunidades para que un proyecto empresarial o de autoempleo alcance el éxito. Es necesario\textbf{ poseer la cualidades} y capacidades técnicas que requiere el trabajo autónomo, tales como iniciativa, capacidad de liderazgo, resistencia a la presión, conocimientos generales de gestión de empresas, mucha capacidad de trabajo, autoorganización y autocontrol.

Respecto al apartado económico, el capital inicial va a ser necesario, aunque en este sector se tratan de cantidades que una persona de nivel económico medio, o incluso bajo, pueda llegar a alcanzar, ya que la \textbf{inversión inicial} necesaria es \textbf{baja} en comparación con otro tipo de negocios. Además, se puede recurrir a la \textbf{financiación ajena}, por medio de un banco, por ejemplo, y optar a \textbf{subvenciones al autoempleo} que puede ser una buena ayuda, especialmente al principio.

\section{El Perfil Profesional de los Técnicos de DAM, DAW y ASIR}





% Glossary

\glsaddall
\printglossaries

% Bibliography

\newpage
\addcontentsline{toc}{chapter}{Bibliografía}
\bibliography{citas}
\bibliographystyle{unsrt}

\end{document}