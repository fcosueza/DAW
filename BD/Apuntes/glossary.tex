\newglossaryentry{Microsoft Windows}{
    name={Microsoft Windows},
    description={Sistema Operativo desarrollado por Microsoft que actualmente se encuentra en la versión 11}
}

\newglossaryentry{Linux}{
    name={Linux},
    description={Sistema Operativo tipo Unix compuesto por software libre y de código abierto que sirve como base para multitud de distribuciones como Debian, Slackware, Ubuntu, Gentoo...}
}

\newglossaryentry{macOS}{
    name={macOS},
    description={Sistema Opertativo basado en Unix, más concretamente en FreeBSD, y que está desarrollado y comercializado por Apple desde 2001}
}

\newglossaryentry{ERP}{
    name={ERP},
    description={Enterprise Resource Planning o Sistema de Planificación de Recursos Empresariales, son programas que se usan para la gestión empresarial y que se hacen cargo de distintas operaciones internas, desde la producción, la distribución o incluso los recursos humanos}
}

\newglossaryentry{Desarrollo de Software}{
    name={Desarrollo de Software},
    description={Conjunto de procesos desde que nace una idea hasta que se convierte en software}
}

\newglossaryentry{CASE}{
    name={CASE},
    description={Computer Aided Software Engineerig}
}

\newglossaryentry{RAD}{
    name={RAD},
    description={Rapid Application Development}
}

\newglossaryentry{IDE}{
    name={IDE},
    description={Integrated Development Environment}
}

\newglossaryentry{UML}{
    name={UML},
    description={Unified Modeling Language}
}

\newglossaryentry{rutina}{
    name={rutina},
    description={Secuencia invariable de instrucciones que forman parte de un programa y que son reutilizables}
}

\newglossaryentry{biblioteca}{
    name={biblioteca},
    description={Conjunto de subprogramas que sirven para desarrollar componentes software o que actúan como interfaz de comunicación entre componentes software}
}

\newglossaryentry{tarjetas perforadas}{
    name={tarjetas perforadas},
    description={Tarjeta que almacenaba información que era leída por un lector específico}
}



