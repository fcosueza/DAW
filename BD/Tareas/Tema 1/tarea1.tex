\input{preambulo.tex}

%----------------------------------------------------------------------------------------
%	TÍTULO Y DATOS DEL ALUMNO
%----------------------------------------------------------------------------------------

\title{
\normalfont \normalsize
\huge \textbf{Instalación de MySQL y Oracle Database en Linux}
}
\author{Francisco Javier Sueza Rodríguez}
\date{\normalsize\today}

%----------------------------------------------------------------------------------------
%                                     DOCUMENTO
%----------------------------------------------------------------------------------------
\begin{document}

\maketitle

\vspace{2ex}

\begin{center}
    \begin{tabular}{l l}
        \textbf{Centro}: & IES Aguadulce \\
        \textbf{Ciclo Formativo}: & Desarrollo Aplicaciones Web (Distancia)\\
        \textbf{Asignatura}: & Bases de Datos\\
        \textbf{Tema}: & Tema 1 - Almacenamiento de la Información \\
    \end{tabular}
\end{center}

\vspace{10ex}

\section{Descripción}
La descripción de los dos apartados de los que consta esta actividad es la siguiente:

\begin{itemize}
    \item \textbf{Apartado 1}: Instalación de MySQL Community Server 8.0.30 (que incluye MySQL Server y MySQL Workbench)
    \begin{enumerate}
        \item Descarga en tu ordenador el producto MySQL Community Server 8.0.30 para el S.O que tengas instalado. Puedes encontrarlo en la página oficial de \href{https://dev.mysql.com/downloads/mysql/}{MySQL}.
        \item Inicia, desde la ubicación donde lo hayas descargado, el instalador del producto y completa la instalación.
        \item Ejecuta MySQL Workbench  y accede a su página principal sin lleguar a establecer ninguna conexión.
        \item Establece una conexión con el usuario administrador 'root' y utiliza la contraseña que hayas establecido durante el proceso de instalación.
        \item Una vez que te hayas autenticado con el usuario administrador, crea un usuario nuevo estableciendo el nombre de usuario y la contraseña. (Los demás valores de la configuración: roles, privilegios,....déjalos por defecto).
        \item Establece una conexión con el nuevo usuario.
    \end{enumerate}
    \item \textbf{Apartado 2}:
    \begin{enumerate}
        \item Descarga en tu ordenador el producto Oracle Database Express Edition 11g R2 que puedes encontrarlo en la página oficial de \href{https://www.oracle.com/database/technologies/xe-prior-release-downloads.html}{Oracle} o bien en estos enlaces:
        \begin{itemize}
            \item \href{https://www.filehorse.com/es/descargar-oracle-database-express/27799/}{Oracle Express Edition 11g Release (32 bits) Windows}
            \item \href{https://www.filehorse.com/es/descargar-oracle-database-express/27798/}{Oracle Express Edition 11g Release (64 bits) Windows}
            \item \href{https://www.tuinformaticafacil.com/descargas-gratis/bases-de-datos/herramientas-oracle/oracle-database-express-edition-11g-r2-para-linux-x64}{Oracle Express Edition 11g Release (64 bits) Linux}
        \end{itemize}
    \item Inicia, desde la ubicación donde lo hayas descargado, el instalador del producto y completa la instalación.
    \item Ejecuta Oracle Database Express Edition 11g R2 y accede a su página principal a través del navegador web que desees.
    \item Inicia sesión con el usuario administrador estándar de Oracle Express y utiliza la contraseña que hayas establecido durante el proceso de instalación.
    \item Una vez que te hayas autenticado con el usuario administrador, crea un espacio de trabajo con su usuario correspondiente.
    \item Entra en el espacio de trabajo con el usuario autenticado.
    \end{enumerate}
\end{itemize}

\section{Instalación de MySQL Community}
En esta sección vamos a indicar los pasos para la instalación y configuración inicial de \textbf{MySQL Community Server 8.0.30}, que incluye también \textbf{MySQL Workbench}. El sistema en el que se instalará es una \textbf{Kubuntu 20.04}, aunque los pasos aquí seguidos son válidos para prácticamente cualquier distribución de Linux.

Así, los \textbf{pasos a seguir} para la instalación y configuración inicial son los siguientes:

\begin{enumerate}
    \item En primer lugar, \textbf{debemos descargarnos} la versión para Linux de MySQL Community Server 8.0.30. En la página principal de \href{https://dev.mysql.com/downloads/mysql/}{MySQL} podemos encontrar enlaces para descargarnos las diferentes versiones de MySQL.

    En nuestro caso, podríamos seleccionar en el menu ``Select Operatig System'' de la página de MySQL, ``Ubuntu'' como sistema operativo y realizar la instalación mediante APT, pero como lo que se nos pide es que usemos el instalador, vamos a seleccionar ``Linux Generic'' y descargar el instalador, en concreto la versión para Linux de 64bits, como podemos ver en la siguiente figura.

    \begin{figure}[ht]
        \centering
        \includegraphics[scale=0.27]{descarga-mysql.png}
        \caption{Página de descarga de MySQL Community Edition}
    \end{figure}
\end{enumerate}
% Bibliography

%\newpage
%\bibliography{citas}
%\bibliographystyle{unsrt}

\end{document}