\input{preambulo.tex}

%----------------------------------------------------------------------------------------
%	TÍTULO Y DATOS DEL ALUMNO
%----------------------------------------------------------------------------------------

\title{
\vspace{10ex}
\normalfont \normalsize
\huge \textbf{Tarea 2: La Iniciativa Emprendedora y el Plan de Empresa}
}
\author{Francisco Javier Sueza Rodríguez}
\date{\normalsize\today}

%----------------------------------------------------------------------------------------
%                                     DOCUMENTO
%----------------------------------------------------------------------------------------
\begin{document}

\maketitle

\thispagestyle{empty}

\vspace{65ex}

\begin{center}
    \begin{tabular}{l l}
        \textbf{Centro}: & IES Aguadulce \\
        \textbf{Ciclo Formativo}: & Desarrollo Aplicaciones Web (Distancia)\\
        \textbf{Asignatura}: & Empresa e Iniciativa Emprendedora\\
        \textbf{Tema}: & Tema 2 -  La Iniciativa Emprendedora y el Plan de Empresa\\
    \end{tabular}
\end{center}

\newpage

\section{La Idea}
El proyecto que se va a presentar hoy es \textbf{Site Wise}, una empresa de desarrollo web especializada en asesoramiento, creación de páginas web y presencia online para PyMES y grandes empresas. Hoy en día la \textbf{presencia en internet} es vital para cualquier negocio, ya sea grande o pequeño, y la implementación de tiendas online pueden suponer un gran incremento en la visibilidad y la venta de cualquier empresa.

En \textbf{Site Wise} nos encargamos de buscar la mejor solución a las necesidades de nuestros clientes. Por un lado, con nuestro sistema de \textbf{creación de páginas web online}, los clientes pueden crear su propia página web haciendo uso de plantillas, mediante un sistema de suscripción que incluye asesoramiento en la creación y mantenimiento de la aplicación. Por otro lado para los clientes que quieren soluciones más personales, \textbf{desarrollamos aplicaciones web a medida} que cubran todas las necesidades del cliente de una forma eficaz y con un trato personal y cercano con nuestros clientes.

    \begin{figure}[H]
    \centering
    \includegraphics[scale=0.40]{logo-empresa.png}
    \caption{Logotipo de la Empresa}
\end{figure}

\section{Promotores del Proyecto}
El promotor del proyecto es \textbf{Francisco Sueza} (es decir yo), soy una persona dinámica con buena capacidad analítica a la que le gusta la resolución de problemas, analizando las diferentes soluciones y aplicando la más adecuada a la situación. Soy una persona creativa, proactiva y empática a la que le encantan los desafíos y que no escatima tiempo ni esfuerzo a la hora de cumplir las metas que me propongo, utilizando para ello las herramientas adecuadas.

\subsubsection{Datos Personales}
\begin{itemize}
    \item \textbf{Nombre y Apellidos}: Francisco Javier Sueza Rodríguez
    \item \textbf{Dirección}: Calle Arrayanes, n6, 4º A
    \item \textbf{Teléfono}: 640585461
    \item \textbf{Correo}: fcosueza@gmail.com
    \item \textbf{Formación Relacionada}:
    \begin{itemize}
        \item \textbf{2004}: Curso FPO “Programación de Aplicaciones Informáticas” en Centro de Estudios Hnos.
        Naranjo, Granada (950h).
        \item \textbf{2020}: Certificado “Diseño Web Responsive” en freeCodeCamp (300h)
        \item \textbf{2021}: Certificado “Algoritmos y estructuras de datos en Javascript” en freeCodeCamp (300h)
        \item \textbf{2022 - Actualidad}: Grado Superior de Desarrollo de Aplicaciones Web (IES Aguadulce)
    \end{itemize}
    \item \textbf{Experiencia Laboral Relacionada}:
    \begin{itemize}
        \item \textbf{2022 - Actualidad}: Desarrollador Web Freelance, desarrollando páginas web, utilizando diferentes herramientas, desde su concepción hasta su puesta en marcha.
    \end{itemize}
    \item \textbf{Habilidades Relacionadas con el Proyecto}:
    \begin{itemize}
        \item \textbf{Diseño de Interfaces}: diseño de interfaces siguiendo los estándares de accesibilidad y buenas prácticas para la experiencia de usuario.
        \item \textbf{Lenguajes de Programación}: Javascript, Java, Ruby, C++, HTML, CSS.
        \item \textbf{Librerias y FrameWorks}: React, Node, SASS, testing library, Jest.
        \item \textbf{Diseño}: Figma, Diseño Responsivo, UX.
    \end{itemize}
    \item \textbf{Motivación}: el desarrollo web es mi pasión, y estoy entusiasmado por la creación de productos de calidad. Siendo mi propio jefe puedo establecer la ruta de la empresa estableciendo buenos estándares de calidad y un proceso de desarrollo innovador para aportar nuevas ideas al sector.
\end{itemize}

\section{La Cultura de Empresa y la Responsabilidad Social}
Las principios de la empresa son dar una \textbf{trato cercano a los clientes}, adaptando el desarrollo a sus necesidades y a sus posibilidades económicas. Los equipos y el \textbf{consumo energético} debe ser responsable con el medio ambiente, primando compañías para el suministro que empleen fuentes de \textbf{energía renovables}, intentado minimizar la huella de carbono generada por la empresa.

En \textbf{futuras ampliaciones de plantilla}, se optaría por una \textbf{organización plana} donde todos los empleados cuentan, primando su \textbf{bienestar laboral}, la \textbf{conciliación familiar} y el \textbf{desarrollo profesional} de estos, creando un ambiente de trabajo lo más agradable posibles para los empleados.

\section{Vente tu Idea en el Foro}

Esta actividad, por desgracia se ha quedado sin realizar, ya que por falta de tiempo me ha sido imposible realizar el vídeo de presentación.

\section{Que prefieres: ¿Trabajar para tí o para otro?}
Personalmente creo que trabajar para uno mismo tiene más ventajas que hacerlo para otra persona, pero también tiene sus inconvenientes. Estas ventajas e inconvenientes son los siguientes:

\begin{itemize}
    \item \textbf{Ventajas}:
    \begin{itemize}
        \item \textbf{Ser tu propio jefe}: esto es una ventaja de trabajar para uno mismo, no solo porque podrás establecer los horarios de trabajo, sino porque tendrás control total sobre los objetivos de la empresa y como se llegará a esos objetivos.
        \item \textbf{Mayor Beneficio Económico}: respecto a trabajar por cuenta ajena, el beneficio económico de ser empresario o autónomo es mayor, en relación a la facturación de al empresa, lo que también puede ser un incentivo.
        \item \textbf{Realización Personal}: el proyecto creado será tuyo y formará parte de ti, por lo que si alcanza el éxito, la satisfacción persona y la realización personal serán mayores que trabajando por cuenta ajena.
    \end{itemize}

    \item \textbf{Desventajas}:
    \begin{itemize}
        \item \textbf{Dedicación}: en una empresa personal la dedicación de tiempo y esfuerzo es mayor que un trabajo por cuenta ajena, teniendo a veces que dedicarle tiempo fuera del horario de trabajo.
        \item \textbf{Responsabilidad}: la responsabilidad será total por parte del empresario o autónomo, teniendo que hacer frente a las posibles consecuencias, tanto económicas como jurídicas, que deriven de la actividad de la empresa.
        \item \textbf{Gestión}: toda la gestión estará a cargo del empresario, así como el pago de impuestos, registros de nombre y empresa, etc.., lo que supone trabajo extra en comparación al trabajo por cuenta ajena.
    \end{itemize}
\end{itemize}

\section{Nos Evaluamos}
Personalmente creo que podría haber hecho más en esta práctica, especialmente en el cálculo del tiempo para su realización ya que esto me ha impedido realizar el ejercicio 4, que era bastante importante. El resto de ejercicios me he esforzado y creo que están correctos, por lo que creo que mi evaluación estaría entre un 5 y un 6, a lo sumo.

% Bibliography

%\newpage
%\bibliography{citas}
%\bibliographystyle{unsrt}

\end{document}