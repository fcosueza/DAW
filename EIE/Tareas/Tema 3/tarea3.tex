\input{preambulo.tex}

%----------------------------------------------------------------------------------------
%	TÍTULO Y DATOS DEL ALUMNO
%----------------------------------------------------------------------------------------

\title{
\vspace{10ex}
\normalfont \normalsize
\huge \textbf{Tarea 3: El Marketing}
}
\author{Francisco Javier Sueza Rodríguez}
\date{\normalsize\today}

%----------------------------------------------------------------------------------------
%                                     DOCUMENTO
%----------------------------------------------------------------------------------------
\begin{document}

\maketitle

\thispagestyle{empty}

\vspace{65ex}

\begin{center}
    \begin{tabular}{l l}
        \textbf{Centro}: & IES Aguadulce \\
        \textbf{Ciclo Formativo}: & Desarrollo Aplicaciones Web (Distancia)\\
        \textbf{Asignatura}: & Empresa e Iniciativa Emprendedora\\
        \textbf{Tema}: & Tema 3 -  El Marketing\\
    \end{tabular}
\end{center}

\newpage

\section*{Dirección Web}
Incluyo aquí la dirección de la página web creada: \url{https://sites.google.com/view/sitewise/inicio}

\section*{Nos Valoramos}
Personalmente creo que mi trabajo puede tener entre un 7 y un 8. Le he dedicado esfuerzo y tiempo pero la verdad es que podría haber realizado un análisis más profundo en algunos apartados. \\ \\

\textbf{NOTA}: he incluido aquí este apartado porque me parecía que estaba un poco fuera de lugar en el sitio web de la empresa, espero que no suponga un problema.

% Bibliography

%\newpage
%\bibliography{citas}
%\bibliographystyle{unsrt}

\end{document}