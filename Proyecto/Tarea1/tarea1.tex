% % % % % % % % % % % % % % % % % % % % % % % % % % % % % % % % % % % % % % % % % % % %
%                                                                                     %
% Short Sectioned Assignment LaTeX Template Version 1.0 (5/5/12)                      %
% This template has been downloaded from: http://www.LaTeXTemplates.com               %
%                                                                                     %
% Original author:  Frits Wenneker (http://www.howtotex.com)                          %
%                                                                                     %
% Modified by: Fco Javier Sueza Rodríguez (fcosueza@disroot.org)                      %
%                                                                                     %
% Changes:                                                                            %
%	    - Custom Chapters, Sections and Subsections (titlesec package)                %
%           - Document type scrbook (oneside)                                         %
%           - Use babel-lang-spanish package and marvosym                             %
%           - Use hyperref, enumitem, tcolorbox and glossaries packages               %
%           - Use Time New Roman (mathptmx), Helvetic and Courier fonts               %
%                                                                                     %
% License: CC BY-NC-SA 3.0 (http://creativecommons.org/licenses/by-nc-sa/3.0/)        %
%                                                                                     %
% % % % % % % % % % % % % % % % % % % % % % % % % % % % % % % % % % % % % % % % % % % %

%-----------------------------------------------%
%	              Packages                  %
%-----------------------------------------------%

\documentclass[paper=a4, fontsize=11pt, oneside]{scrbook}

% ---- Text Input/Output ----- %

\usepackage[T1]{fontenc}
\usepackage[utf8]{inputenc}
\usepackage{mathptmx}
\usepackage[scaled=.92]{helvet}
\usepackage{courier}
\usepackage[indent=12pt]{parskip}

\usepackage{geometry}
\geometry{verbose,tmargin=3cm,bmargin=3cm,lmargin=2.6cm,rmargin=2.6cm}

% ---- Language ----- %

\usepackage[spanish]{babel}
\usepackage{marvosym}

% ---- Another packages ---- %

\usepackage{amsmath,amsfonts,amsthm}
\usepackage{graphics,graphicx}
\usepackage{titlesec}
\usepackage{fancyhdr}
\usepackage{tcolorbox}
\usepackage{hyperref}
\usepackage{enumitem}
\usepackage[automake]{glossaries}

%--------------------------------------------------------------------%
%                      Customizing Document                          %
%--------------------------------------------------------------------%


% ----------- Custom Chapters, Sections and Subsections -------------- %

\titleformat{\chapter}[display]
			{\bfseries\Huge}
			{Tema \ \thechapter} {0.5ex}
			{\vspace{1ex}\centering}

\titleformat{\section}[hang]
			{\bfseries\Large}
			{\thesection}{0.5em}{}

\titleformat{\subsection}[hang]
			{\bfseries\large}
			{\thesubsection}{0.5em}{}

\titleformat{\subsubsection}[hang]
			{\bfseries\large}
			{\thesubsubsection}{0.5em}{}

\hypersetup{
    colorlinks=true,
    linkcolor=black,
    urlcolor=magenta
}

% ------------------- Custom heaaders and footers ------------------- %

\pagestyle{fancyplain}

\fancyhead[]{}
\fancyfoot[L]{}
\fancyfoot[C]{}
\fancyfoot[R]{\thepage}

\renewcommand{\headrulewidth}{0pt} % Remove header underlines
\renewcommand{\footrulewidth}{0pt} % Remove footer underlines

\setlength{\headheight}{13.6pt} % Customize the height of the header

% --------- Numbering equations, figures and tables ----------------- %

\numberwithin{equation}{section} % Number equations within sections
\numberwithin{figure}{section} % Number figures within sections
\numberwithin{table}{section} % Number tables within sections

% ------------------------ New Commands ----------------------------- %

\newcommand{\horrule}[1]{\rule{\linewidth}{#1}} % Create horizontal rule command


%----------------------------------------------------------------------------------------
%	TÍTULO Y DATOS DEL ALUMNO
%----------------------------------------------------------------------------------------

\title{
\vspace{10ex}
\normalfont \normalsize
\huge \textbf{Informe Proyecto DAW}
}
\author{Francisco Javier Sueza Rodríguez}
\date{\normalsize\today}

%----------------------------------------------------------------------------------------
%                                     DOCUMENTO
%----------------------------------------------------------------------------------------
\begin{document}


\maketitle

\thispagestyle{empty}

\vspace{65ex}

\begin{center}
    \begin{tabular}{l l}
        \textbf{Centro}: & IES Aguadulce \\
        \textbf{Ciclo Formativo}: & Desarrollo Aplicaciones Web (Distancia)\\
        \textbf{Asignatura}: & Proyecto de Desarrollo de Aplicaciones Web\\
        \textbf{Tema}: & Tema 1 - Empresas y Programación\\
    \end{tabular}
\end{center}

\newpage

\tableofcontents

\newpage

\section{El Producto y la Empresa}
En este apartado se van a explicar los puntos relacionados con la empresa y nuestro producto, especificando la organización, que oportunidades nos brinda y realizando un breve análisis de la competencia, entre otras cosas.

\subsection{Organización de la Empresa}
La \textbf{organización de empresa} en este caso va a ser muy simple, ya que la empresa consta de solo 1 trabajador, en este caso, el alumno, que será el encargado de asumir todos los roles y responsabilidades dentro de ésta.

\subsection{Oportunidad de Negocio}
Actualmente podemos encontrar diferentes aplicaciones para gestión de comunidades de vecinos en el mercado. Estás aplicaciones, además de tener un coste que en algunos casos puede no ser asumido por los usuarios, no están tan enfocadas en la accesibilidad y la simpleza como deberían, lo que puede producir que vecinos con necesidades especiales de accesibilidad o pocos conocimientos en TIC tengan dificultades de para su uso.

Nuestra aplicación, \textbf{SolucionesVecinales}, ofrece una solución a este problema, con una aplicación accesible y simple que pueda usar todo el mundo, además de ofrecerse de forma gratuita en la plataforma web, ya que ésta nos ofrece la posibilidad de acceder al mayor número de usuarios, independientemente de la plataforma que usen, siempre que tenga conexión a la red y un navegador web.

En este aspecto, \textbf{no se espera un beneficio económico}, ya que como se ha comentado la aplicación se ofrece de forma gratuita, pero ofreciendo este servicio a la comunidad se \textbf{promociona la imagen de marca} del programador de cara a futuros clientes o empleadores, ofreciendo un valor que, dadas las circunstancias, es incluso mayor que el rédito económico que se pueda obtener. Además, el desarrollo de un proyecto de esta envergadura no solo sirve como un punto de \textbf{aprendizaje}, sino como un escaparate para \textbf{mostrar las habilidades} del programador.

\subsection{Subvenciones y Ayudas}
El trabajador se dará de alta como \textbf{autónomo}, pudiendo acogerse una \textbf{reducción de la cuota de autónomos}, por el que se obtendrá una reducción de la tarifas por contingencias comunes y profesionales, regulada por el \href{https://www.boe.es/eli/es/rdl/2022/07/26/13/con}{Real Decreto-ley 12/2022} de 26 de Julio, por el que se establece el nuevo sistema de cotización para trabajadores por cuenta propia y por el \href{https://www.boe.es/eli/es/rdl/2022/08/01/14/con}{Real Decreto-ley 14/2022} de 1 de Agosto.

Tras un año de alta como trabajador autónomo, podrá acogerse a la \textbf{Tarifa Cero}, de la Junta de Andalucía, por la  que se reembolsarán todas las cuotas del año trabajado. Esta subvención esta regulada por la \href{https://www.juntadeandalucia.es/boja/2023/125/1}{Orden 29 de Junio de 2023 de la Junta de Anlucía} donde se establece la subvención y la \href{https://www.juntadeandalucia.es/boja/2023/248/4}{Resolución del 23 de Diciembre} de la Dirección de Trabajo Autónomo y Economía Social, por la que se realiza la convocatoria para los años 2024 a 2026.

\subsection{Funcionalidades Demandadas}
Se ha realizado una estudio con diferentes aplicaciones para la gestión de comunidades de vecinos. Las aplicaciones que se han analizado han sido \textbf{Colindar}, \textbf{Fincapp}, \textbf{VecinosEnRed} y \textbf{Insulae}. Después de estudiar estás aplicaciones se han establecido las funcionalidades más demandas por lo usuarios y que se ofrecen en éstas, siendo los siguientes:

\begin{itemize}
	\item \textbf{Reserva de Espacios Comunitarios}
	\item \textbf{Estados de las Cuentas}
	\item \textbf{Gestión de Incidencias}
	\item \textbf{Acceso a la Documentación}
\end{itemize}

Estás cuatro funcionalidades las podemos encontrar en todas las aplicaciones mencionadas, por lo que cabe esperar que sean las más demandadas por los usuarios. Podemos encontrar otras en las diferentes aplicaciones pero esas son ya podrías de cada aplicación.

\subsection{Conclusiones}
Actualmente podemos encontrar \textbf{varias aplicaciones en el mercado} para la gestión de comunidades de vecinos, pero son \textbf{aplicaciones de pago} y que \textbf{no están enfocadas} en la \textbf{accesibilidad} y facilidad de uso tanto como deberían.

Nuestra aplicación, \textbf{SolucionesVecinales}, ofrece una \textbf{alternativa gratuita} que incluye \textbf{todas las funcionalidades más demandadas} por los usuarios y que ofrece una \textbf{interfaz accesible y simple de utilizar}, para que todos los usuarios puedan implicarse en la gestión de su comunidad independientemente de sus conocimientos TIC o problemas de accesibilidad.

Además, nuestra aplicación necesita una \textbf{inversión mínima} para su realización, ya que el equipo se compone de un solo programado, por lo que aunque no se espera un beneficio económico, el \textbf{beneficio publicitario y de imagen de marca} compensará el desarrollo y la puesta en funcionamiento de la aplicación.


\section{Legislación y Obligaciones}
En esta sección se van a exponer todas las obligaciones fiscales y legales que deberá cumplir la empresa, así como toda la legislación aplicable en esta materia y relativa a la protección de datos y manejo de cookies.

\subsection{Obligaciones Fiscales y Laborales}





%\bibliographystyle{unsrt}

\end{document}