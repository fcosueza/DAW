\input{preambulo.tex}

%----------------------------------------------------------------------------------------
%	TÍTULO Y DATOS DEL ALUMNO
%----------------------------------------------------------------------------------------

\title{
\vspace{10ex}
\normalfont \normalsize
\huge \textbf{Proyecto de DAW: Tarea 1}
}
\author{Francisco Javier Sueza Rodríguez}
\date{\normalsize\today}

%----------------------------------------------------------------------------------------
%                                     DOCUMENTO
%----------------------------------------------------------------------------------------
\begin{document}


\maketitle

\thispagestyle{empty}

\vspace{65ex}

\begin{center}
    \begin{tabular}{l l}
        \textbf{Centro}: & IES Aguadulce \\
        \textbf{Ciclo Formativo}: & Desarrollo Aplicaciones Web (Distancia)\\
        \textbf{Asignatura}: & Proyecto de Desarrollo de Aplicaciones Web\\
        \textbf{Tema}: & Tema 1 - Empresas y Programación\\
    \end{tabular}
\end{center}

\newpage

\tableofcontents

\newpage

\section{Organización de la Empresa}
La \textbf{organización de empresa} en este caso va a ser muy simple, ya que la empresa consta de solo 1 trabajador, en este caso, el alumno, que será el encargado de asumir todos los roles y responsabilidades dentro de ésta.

\section{Oportunidad de Negocio}
Actualmente podemos encontrar diferentes aplicaciones para gestión de comunidades de vecinos en el mercado. Estás aplicaciones, además de tener un coste que en algunos casos puede no ser asumido por los usuarios, no están tan enfocadas en la accesibilidad y la simpleza como deberían, lo que puede producir que vecinos con necesidades especiales de accesibilidad o pocos conocimientos en TIC tengan dificultades de para su uso.

Nuestra aplicación, \textbf{SolucionesVecinales}, ofrece una solución a este problema, con una aplicación accesible y simple que pueda usar todo el mundo, además de ofrecerse de forma gratuita en la plataforma web, ya que ésta nos ofrece la posibilidad de acceder al mayor número de usuarios, independientemente de la plataforma que usen, siempre que tenga conexión a la red y un navegador web.

En este aspecto, \textbf{no se espera un beneficio económico}, ya que como se ha comentado la aplicación se ofrece de forma gratuita, pero ofreciendo este servicio a la comunidad se \textbf{promociona la imagen de marca} del programador de cara a futuros clientes o empleadores, ofreciendo un valor que, dadas las circunstancias, es incluso mayor que el rédito económico que se pueda obtener. Además, el desarrollo de un proyecto de esta envergadura no solo sirve como un punto de \textbf{aprendizaje}, sino como un escaparate para \textbf{mostrar las habilidades} del programador.

\section{Subvenciones y Ayudas}



%\bibliographystyle{unsrt}

\end{document}