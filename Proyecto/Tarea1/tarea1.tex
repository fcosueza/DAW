\input{preambulo.tex}

%----------------------------------------------------------------------------------------
%	TÍTULO Y DATOS DEL ALUMNO
%----------------------------------------------------------------------------------------

\title{
\vspace{10ex}
\normalfont \normalsize
\huge \textbf{Informe Proyecto DAW}
}
\author{Francisco Javier Sueza Rodríguez}
\date{\normalsize\today}

%----------------------------------------------------------------------------------------
%                                     DOCUMENTO
%----------------------------------------------------------------------------------------
\begin{document}


\maketitle

\thispagestyle{empty}

\vspace{65ex}

\begin{center}
    \begin{tabular}{l l}
        \textbf{Centro}: & IES Aguadulce \\
        \textbf{Ciclo Formativo}: & Desarrollo Aplicaciones Web (Distancia)\\
        \textbf{Asignatura}: & Proyecto de Desarrollo de Aplicaciones Web\\
        \textbf{Tema}: & Tema 1 - Empresas de Programación\\
    \end{tabular}
\end{center}

\newpage

\tableofcontents

\newpage

\section{El Producto y la Empresa}
En este apartado se van a explicar los puntos relacionados con la empresa y nuestro producto, especificando la organización, que oportunidades nos brinda y realizando un breve análisis de la competencia, entre otras cosas.

\subsection{Organización de la Empresa}
La \textbf{organización de empresa} en este caso va a ser muy simple, ya que la empresa consta de solo 1 trabajador, en este caso, el alumno, que será el encargado de asumir todos los roles y responsabilidades dentro de ésta.

\subsection{Oportunidad de Negocio}
Actualmente podemos encontrar diferentes aplicaciones para gestión de comunidades de vecinos en el mercado. Estás aplicaciones, además de tener un coste que en algunos casos puede no ser asumido por los usuarios, no son tan accesibles como deberían. Algunas están enfocadas en los gestores de fincas, otras tienen interfaces poco intuitivas o precios elevados.

Nuestra aplicación, \textbf{SolucionesVecinales}, ofrece una solución a este problema, centrada en la accesibilidad y facilidad de uso, se ofrece de forma gratuita en la plataforma web, ya que ésta nos ofrece la posibilidad de acceder al mayor número de usuarios, independientemente de la plataforma que usen, siempre que tenga conexión a la red y un navegador web, estando optimizada para la mayoría de navegadores que podemos encontrar en el mercado.

En el \textbf{aspecto económico}, \textbf{no se espera un beneficio directo}, ya que como se ha comentado, la aplicación se ofrece de forma gratuita. Sin embargo, ofreciendo este servicio a la comunidad se \textbf{promociona la imagen de marca} del programador de cara a futuros clientes o empleadores, ofreciendo un valor que, dadas las circunstancias, es incluso mayor que el rédito económico que se pueda obtener. Además, el desarrollo de un proyecto de esta envergadura no solo sirve como un punto de \textbf{aprendizaje}, sino como un escaparate para \textbf{mostrar las habilidades} del programador.

\subsection{Subvenciones y Ayudas}
El trabajador se dará de alta como \textbf{autónomo}, pudiendo acogerse una \textbf{reducción de la cuota de autónomos}, por el que se obtendrá una reducción de la tarifas por contingencias comunes y profesionales, regulada por el \href{https://www.boe.es/eli/es/rdl/2022/07/26/13/con}{Real Decreto-ley 12/2022} de 26 de Julio, por el que se establece el nuevo sistema de cotización para trabajadores por cuenta propia y por el \href{https://www.boe.es/eli/es/rdl/2022/08/01/14/con}{Real Decreto-ley 14/2022} de 1 de Agosto.

Tras un año de alta como trabajador autónomo, podrá acogerse a la \textbf{Tarifa Cero}, de la Junta de Andalucía, por la  que se reembolsarán todas las cuotas del año trabajado. Esta subvención esta regulada por la \href{https://www.juntadeandalucia.es/boja/2023/125/1}{Orden 29 de Junio de 2023 de la Junta de Anlucía} donde se establece la subvención y la \href{https://www.juntadeandalucia.es/boja/2023/248/4}{Resolución del 23 de Diciembre} de la Dirección de Trabajo Autónomo y Economía Social, por la que se realiza la convocatoria para los años 2024 a 2026.

Para realizar estos trámites, se concertará una cita con la oficina del \href{https://www.andaluciaemprende.es/CADE/}{CADE} local, para recibir ayuda profesional de forma gratuita en el proceso de alta en el RETA.

\subsection{Productos Similares}
En este apartado se han analizado varias aplicaciones similares a la nuestra, y se han especificado sus principales funcionalidades, así como que diferencia a nuestra aplicación de ellas y que podemos aportar. Las aplicaciones analizadas han sido las siguientes:

\begin{itemize}
	\item  \textbf{Fynkus} (\href{https://www.fynkus.es/}{wwww.fynkus.es})
	\begin{itemize}
		\item \textbf{Descripción}: aplicación para la gestión de comunidades de vecinos orientada a los gestores de fincas enfocada en administración financiera de la comunidad.
		\item \textbf{Funcionalidades}: ofrece gestión de \textbf{incidencias y financiera}, con categorización de los movimientos bancarios. Además, ayuda a gestionar las \textbf{reuniones de vecinos}.
		\item \textbf{Problema}: el software esta principalmente enfocado en los administradores de fincas y no en los vecinos. Además, no ofrece funcionalidades para la reserva de espacios comunes ni acceso a documentos comunitarios de forma transparente.	
	\end{itemize}
	
	\item  \textbf{Colindar} (\href{https://colindar.com/}{wwww.colindar.com})
	\begin{itemize}
		\item \textbf{Descripción}: aplicación para la gestión de comunidades de vecinos orientada a la gestión de espacios comunes .
		\item \textbf{Funcionalidades}: la principal funcionalidad es al \textbf{gestión de espacios comunes}, permitiendo su reserva y evitando su abuso. Además, ofrece la \textbf{apertura automática de puertas} y \textbf{control de accesos}.
		\item \textbf{Problema}: en este caso, la aplicación solo se enfoca en la gestión de espacios comunes, dejando de lado el resto de tareas administrativas que podemos encontrar en al gestión de una comunidad de vecinos.	
	\end{itemize}
	
	\item  \textbf{Vecinos en Red} (\href{https://www.vecinosenred.es/}{wwww.vecinosenred.es})
	\begin{itemize}
		\item \textbf{Descripción}: aplicación para la gestión de comunidades de vecinos orientada a la gestión de diferentes tareas dentro de la comunidad de vecinos.
		\item \textbf{Funcionalidades}: de las tres analizadas esta es la que ofrece más variedad de funcionalidades, ofreciendo \textbf{gestión de incidencias y documental}, así como \textbf{reserva de espacios comunes}.También proporciona un \textbf{tablón de anuncios}.
		\item \textbf{Problema}: las funcionalidad de esta aplicación son bastante completa, aunque no tiene un sistema de mensajes entre los usuarios y la interfaz es poco intuitiva. Además, para comunidades grandes, puede ser costosa.
	\end{itemize}
\end{itemize}

Como podemos ver, encontramos diferentes aplicaciones para la gestión de comunidades de vecinos en el mercado, pero mientras que algunas \textbf{carecen de funcionalidades básicas} para la gestión de comunidades de vecinos, ya que están orientadas a la gestión de aspecto concretos, otras tienen una \textbf{interfaz engorrosa}. Además, todas ellas \textbf{son de pago}, en algunos casos, con \textbf{precios elevados}.

Nuestra aplicación soluciona todos estos problemas ofreciendo: 

\begin{itemize}
	\item \textbf{Todas las funcionalidades} necesarias para la gestión de una comunidad de vecinos en una misma aplicación, como son la gestión de incidencias, documentos, espacios comunes, un tablón de anuncios y un sistema de mensajería entre los usuarios.
	\item \textbf{Interfaz sencilla y accesible} para todos los usuarios. Se ha puesto especial énfasis en seguir todas las directrices WAI-ARIA para facilitar el acceso a todas las personas independientemente de sus problemas de accesibilidad. 
	\item \textbf{Aplicación gratuita} para que todas las comunidades y vecinos puedan usarla independientemente de su situación financiera.
\end{itemize}
	





\section{Legislación, Obligaciones y  Riesgos Laborales}
En esta sección se van a exponer todas las obligaciones fiscales y legales que deberá cumplir la empresa, así como toda la legislación aplicable en esta materia y relativa a la protección de datos y manejo de cookies.

\subsection{Obligaciones Fiscales y Laborales}
Como hemos comentado, no se va a crear una empresa como tal, sino que el desarrollo va a ser llevado a cabo por un \textbf{trabajador autónomo}, por lo que los tramites de constitución, entre otros, serán más simples.

En primer lugar, comentar que la \textbf{legislación base aplicable} para el trabajo autónomo se establece en la \href{https://www.boe.es/buscar/act.php?id=BOE-A-2007-13409}{Ley 20/2007, de 11 de Julio} del Estatuto del trabajador autónomo. Partiendo de esta legislación base, hay un conjunto de Real Decretos que se aplican a diferentes aspectos del trabajador autónomo, sa saber:

\begin{itemize}
	\item \href{https://www.boe.es/buscar/act.php?id=BOE-A-2015-11724}{Real Decreto Legislativo 8/2015, de 30 de octubre}, por el que se aprueba el texto refundido de la Ley General de la Seguridad Social.
	\item \href{https://www.boe.es/buscar/doc.php?id=BOE-A-2022-12482}{Real Decreto-ley 13/2022, de 26 de julio}, por el que se establece un nuevo sistema de cotización para los trabajadores autónomos.
	\item \href{https://www.boe.es/buscar/doc.php?id=BOE-A-2011-17173}{Real Decreto 1541/2011, de 31 de octubre}, por la que se establece el rey específico de protección por cese de actividad del trabajador autónomo.
	\item \href{https://www.boe.es/diario_boe/txt.php?id=BOE-A-2023-2472}{Orden PCM/74/2023, de 30 de enero}, por el que se desarrollan las normas legales de cotización a la Seguridad Social, desempleo, protección por cese de actividad, Fondo de Garantía Salarial y formación profesional para el ejercicio 2023.
	\item \href{https://www.boe.es/buscar/doc.php?id=BOE-A-2017-9211}{Orden ESS/739/2017, de 26 de julio}, por el que se establecen las bases reguladoras de la concesión de subvenciones a las actividades de promoción del trabajo autónomo.
\end{itemize}

Como vemos, la legislación aplicable es muy extensa, y de esta se derivan un conjunto de \textbf{obligaciones fiscales y legales} que vamos a listar a continuación, dividiéndolas en obligaciones en el inicio de la actividad como trabajador autónomo y en el desarrollo de la actividad.

\begin{itemize}
	\item \textbf{Antes de Iniciar la Actividad}:
	\begin{itemize}
		\item \textbf{Darse de alta} en el \textbf{Censo de Actividad Económicas} de la Agencia Tributaria.
		\item \textbf{Darse de alta} en la \textbf{Seguridad Social} como trabajador autónomo.

	\end{itemize}
	\item \textbf{Durante el Desarrollo de la Actividad}:
		\begin{itemize}
		\item Presentación trimestral de  la \textbf{declaración de IVA}.
		\item Pago de la \textbf{cuota de autónomos}.
		\item Pago del impuesto sobre la \textbf{Renta de las Personas Físicas} (IRPF)	
	    \item \textbf{Llevar un registro} de facturas emitidas y recibidas.
	\end{itemize}
\end{itemize}

Además de estas legislación, cabe mencionar que también es de aplicación \href{https://www.boe.es/diario_boe/txt.php?id=BOE-A-2023-17238}{Resolución de 13 de julio de 2023}, de la Dirección General del Trabajo, que establece el XVIII Convenio Colectivo Estatal de empresas de consultoría, tecnologías de la información y estudios de mercado que, aunque es un convenio aplicado a empresas, hay que tenerlo en consideración a la hora de desarrollar la actividad.

\subsection{Riesgos Laborales}
Respecto a los riesgos laborales, hay un conjunto de riesgos que se aplican a las profesiones relacionadas con la informática que listamos a continuación:

\begin{itemize}
	\item \textbf{Fatiga Visual} producida por el uso de pantallas.
	\item \textbf{Fatiga muscular} provocada por malas posturas y incorrecta ubicación de los equipos informáticos.
	\item \textbf{Entres y ansiedad} por la presión de cumplir plazos.
	\item \textbf{Contacto eléctrico} con los equipos informáticos.
	\item \textbf{Exposición a radiaciones electromagnéticas} producidas por le equipos informáticos.
\end{itemize} 

Para \textbf{mitigar estos riesgos laborales} se van a tomar un \textbf{conjunto de medidas} que nos ayudarán a realizar nuestro trabajo eficientemente y sin estos problemas, siendo las principales:

\begin{itemize}
	\item \textbf{Toma de descansos regulares}: se tomarán descansos regularmente para estirar las piernas y descansas la vista con una duración de 15 min aproximadamente cada 2 horas de trabajo.
	\item \textbf{Adecuación del espacio de trabajo}: se trabajará en un entorno con la iluminación adecuada para mitigar la fatiga visual y con una temperatura adecuada para reducir el estrés térmico.
	\item \textbf{Buen posicionamiento del monitor}: se colocará el monitor a una altura y distancia adecuada evitando la fatiga visual y posturas incómodas.
\end{itemize} 

\subsection{Política de Protección de Datos y Uso de Cookies}
En referencia a la \textbf{política de protección de datos}, es de aplicación a nivel europeo el \href{https://www.boe.es/doue/2016/119/L00001-00088.pdf}{Reglamento (UE) 2016/679}, más conocido como \textbf{RGPD}, y que en España se implementa mediante la \href{https://www.boe.es/buscar/act.php?id=BOE-A-2018-16673}{Ley Orgánica 3/2018, de 5 de diciembre}, de Protección de Datos y garantía de los derechos digitales. Estás dos normativas establecen las restricciones en el tratamiento de datos así como el derecho a eliminación y rectificación de dichos datos, entre otras cosas.

Respecto al \textbf{uso de cookies}, además de la normativa ya comentada, hay que tener en cuenta el \textbf{aparatado 2 del articulo 22 de la LSSI} o \href{https://www.boe.es/buscar/act.php?id=BOE-A-2002-13758}{Ley 34/2002, de 11 de julio}, de servicios de la sociedad de la información y el comercio electrónico y que establece:

\textit{"Los prestadores de servicios podrán utilizar dispositivos de almacenamiento y recuperación de datos en equipos terminales de los destinatarios, a condición de que los mismos hayan dado su consentimiento después de que se les haya facilitado información clara y completa sobre su utilización, en particular, sobre los fines del tratamiento de los datos, con arreglo a lo dispuesto en la Ley Orgánica 15/1999, de 13 de diciembre, de protección de datos de carácter personal. \\ \\ Cuando sea técnicamente posible y eficaz, el consentimiento del destinatario para aceptar el tratamiento de los datos podrá facilitarse mediante el uso de los parámetros adecuados del navegador o de otras aplicaciones. \\ \\ Lo anterior no impedirá el posible almacenamiento o acceso de índole técnica al solo fin de efectuar la transmisión de una comunicación por una red de comunicaciones electrónicas o, en la medida que resulte estrictamente necesario, para la prestación de un servicio de la sociedad de la información expresamente solicitado por el destinatario".}

Para cumplir con esta disposición, la \textbf{Agencia Española de Protección de Datos} ha elaborado una \href{https://www.aepd.es/guias/guia-cookies.pdf}{Guía sobre el uso de la cookies} que será el documento de referencia a la hora de realizar el tratamiento de las cookies en nuestra aplicación.

\section{Conclusiones}
Actualmente podemos encontrar \textbf{varias aplicaciones en el mercado} para la gestión de comunidades de vecinos, pero son aplicación que \textbf{no tienen todas las funcionalidades} que requiere la gestión de una comunidad de vecinos, algunas con \textbf{interfaces pobres} y que no se centran en al facilidad de uso y la accesibilidad.

Nuestra aplicación, \textbf{SolucionesVecinales}, ofrece una alternativa \textbf{gratuita} que incluye todas las funcionalidades en una misma aplicación, para poder realizar toda la gestión de la comunidad de forma centralizada, poniendo especial énfasis en la accesibilidad y facilidad de uso.

Además, nuestra aplicación necesita una \textbf{inversión mínima} para su realización, ya que el equipo se compone de un solo programado, por lo que aunque no se espera un beneficio económico, el \textbf{beneficio publicitario y de imagen de marca} compensará el desarrollo y la puesta en funcionamiento de la aplicación.







%\bibliographystyle{unsrt}

\end{document}