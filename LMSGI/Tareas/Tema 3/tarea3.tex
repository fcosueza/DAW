\input{preambulo.tex}

%----------------------------------------------------------------------------------------
%	TÍTULO Y DATOS DEL ALUMNO
%----------------------------------------------------------------------------------------

\title{
\vspace{10ex}
\normalfont \normalsize
\Huge \textbf{Tarea 3: Aplicación de los Lenguajes de Marcas a la Sindicación de Contenidos}
}
\author{Francisco Javier Sueza Rodríguez}
\date{\normalsize\today}

%----------------------------------------------------------------------------------------
%                                     DOCUMENTO
%----------------------------------------------------------------------------------------
\begin{document}

\maketitle

\thispagestyle{empty}

\vspace{62ex}

\begin{center}
    \begin{tabular}{l l}
        \textbf{Centro}: & IES Aguadulce \\
        \textbf{Ciclo Formativo}: & Desarrollo Aplicaciones Web (Distancia)\\
        \textbf{Asignatura}: & Lenguajes de Marcas y Sistemas de Gestión de la Información\\
        \textbf{Tema}: & Tema 3 -  Aplicación de los Lenguajes de Marcas a la Sindicación de Contenidos\\
    \end{tabular}
\end{center}

\newpage

\tableofcontents

\newpage

\listoffigures

\newpage

\section{Caso Práctico}
Gracias al gran trabajo realizado por nuestra empresa, \textbf{PUMM Technologies}, se nos ha encargado que ampliemos nuestro ámbito de actuación ofreciendo servicios de sindicación de contenidos.

Seguiremos demostrando nuestro saber hacer y nuestros conocimientos en RSS y ATOM. Además vamos a realizar suscripciones a dichos canales utilizando el agregador de noticias Thunderbird. El programa Mozilla Thunderbird es una aplicación de gestión de correo electrónico que, además, nos permite la suscripción a canales RSS y Atom.

Vamos a utilizar el programa Mozilla Thunderbird para agregar canales de contenidos RSS o Atom relacionados con alguna empresa de mensajería/reparto para ver su funcionamiento.
% Bibliography

\newpage
\bibliography{citas}
\bibliographystyle{unsrt}

\end{document}