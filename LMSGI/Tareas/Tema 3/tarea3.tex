\input{preambulo.tex}

%----------------------------------------------------------------------------------------
%	TÍTULO Y DATOS DEL ALUMNO
%----------------------------------------------------------------------------------------

\title{
\vspace{10ex}
\normalfont \normalsize
\Huge \textbf{Tarea 3: Aplicación de los Lenguajes de Marcas a la Sindicación de Contenidos}
}
\author{Francisco Javier Sueza Rodríguez}
\date{\normalsize\today}

%----------------------------------------------------------------------------------------
%                                     DOCUMENTO
%----------------------------------------------------------------------------------------
\begin{document}

\maketitle

\thispagestyle{empty}

\vspace{62ex}

\begin{center}
    \begin{tabular}{l l}
        \textbf{Centro}: & IES Aguadulce \\
        \textbf{Ciclo Formativo}: & Desarrollo Aplicaciones Web (Distancia)\\
        \textbf{Asignatura}: & Lenguajes de Marcas y Sistemas de Gestión de la Información\\
        \textbf{Tema}: & Tema 3 -  Aplicación de los Lenguajes de Marcas a la Sindicación de Contenidos\\
    \end{tabular}
\end{center}

\newpage

\tableofcontents

\newpage

\listoffigures

\newpage

\section{Caso Práctico}
Gracias al gran trabajo realizado por nuestra empresa, \textbf{PUMM Technologies}, se nos ha encargado que ampliemos nuestro ámbito de actuación ofreciendo servicios de sindicación de contenidos.

Seguiremos demostrando nuestro saber hacer y nuestros conocimientos en RSS y ATOM. Además vamos a realizar suscripciones a dichos canales utilizando el agregador de noticias Thunderbird. El programa Mozilla Thunderbird es una aplicación de gestión de correo electrónico que, además, nos permite la suscripción a canales RSS y Atom.

Vamos a utilizar el programa Mozilla Thunderbird para agregar canales de contenidos RSS o Atom relacionados con alguna empresa de mensajería/reparto para ver su funcionamiento.

\section{Enunciado}
Realiza un documento de texto (de Writer o similar, convertirlo a PDF y entregar el PDF) donde documentes la realización de cada uno de los apartados siguientes. Este documento es obligatorio, no se pasará a corregir el resto de la tarea si no se entrega este documento. En dicho documento deberás adjuntar capturas de pantalla, las explicaciones textuales de los distintos pasos necesarios que has dado y las decisiones que has tomado para conseguir realizar los apartados siguientes.

\subsection{Apartado A: RSS}
Crear un canal mediante el estándar RSS 2.0. El canal, además de los elementos obligatorios, tendrá los siguientes: El idioma utilizado (español), el correo del editor (vuestro email), la fecha de publicación y las siguientes categorías: 'paquetería', 'envíos' 'transporte urgente'.

Además tendrá las siguientes entradas:

\begin{itemize}
    \item Entrada sobre una empresa de paquetería que tú elijas con los siguientes elementos:
    \begin{itemize}
        \item Título
        \item Categoría: paquetería
        \item Fecha
        \item Contenido
        \item Enlace
    \end{itemize}

    \item Entrada sobre \href{https://logistica.cdecomunicacion.es/noticias/sectoriales/45956/aliexpress-llegada-iva-paqueteria-internacional}{cómo afectará la llegada del IVA a la paquetería internacional}. Del enlace de esta noticia se extraerán,  además de los elementos obligatorios, los siguientes: el título, la descripción, la \href{https://logistica.cdecomunicacion.es/wp-content/webp-express/webp-images/uploads/2022/12/1624960313-paquetera-iva-aduanas-1-1140x594.jpg.webp}{imagen} y la fecha de publicación.

    \item Entrada sobre los \href{https://logistica.cdecomunicacion.es/wp-content/webp-express/webp-images/uploads/2022/12/1624960313-paquetera-iva-aduanas-1-1140x594.jpg.webp}{problemas en la distribución de la paquetería internacional}. Del enlace de esta noticia se extraerán, además de los elementos obligatorios, los siguientes: el título, el link, la descripción, la \href{https://cdn0.celebritax.com/sites/default/files/styles/watermark_100/public/1642440970-denuncian-atraso-siete-meses-distribucion-paqueteria-internacional-cuba.jpg}{imagen} y la fecha de publicación.
\end{itemize}

La información de las entradas deben ser sobre los datos reales de páginas activas.

Valida el fichero RSS creado usando el servicio disponible en la web \href{https://validator.w3.org/feed/#validate_by_input}{W3C Feed Validation Service}, introduciendo el código del fichero RSS. Realiza una captura de pantalla, a página completa, de la validación que permitan mostrar que se ha realizado correctamente. Debe verse todo el navegador, no recortar la imagen. Es normal que en la validación aparezcan avisos 'warning'. Estas capturas se incluirán en el documento de texto que debes realizar. Incluye esta captura, con su correspondiente explicación textual, en el documento que debes realizar.

\subsection{Aparado B: Atom}
Crear un canal usando el estándar Atom 1.0 con los mismos datos, para el canal y las entradas, utilizados en el apartado A. En este estándar se deberán rellenar todos los elementos obligatorios, todos los elementos recomendados y todos los datos indicados explícitamente en el enunciado del apartado A.

Valida dicho fichero usando el servicio disponible en la web \href{https://validator.w3.org/feed/#validate_by_input}{W3C Feed Validation Service}. Realiza una captura de pantalla, a página completa, de la validación que permitan mostrar que se ha realizado correctamente. Debe verse todo el navegador, no recortar la imagen. Es normal que en la validación aparezcan avisos 'warning'. Estas capturas se incluirán en el documento de texto que debes realizar. Incluye esta captura, con su correspondiente explicación textual, en el documento que debes realizar.

\subsection{Aparatado C: Inclusión del Canal de Noticias en la Página Web}
Modifica la web que has realizado en la tarea 2 incluyendo una carpeta llamada feed que contenga los dos archivos de las actividades A y B (canal.rss y canal.atom). En caso de no haber entregado la tarea 2 se debe utilizar uno de los dos ejemplos que se proporcionaban en dicha tarea 2 como base para crear la funcionalidad el menú.

Añade todo lo necesario para que en el pie de la página web principal tenga dos imágenes (logotipos rss y atom) con enlaces a los canales de noticias. Incluye todo lo necesarios para su funcionamiento.

\subsection{Aparado D: Utilización de Agregador de Contenido}
Descarga e instala el programa de la web oficial \url{https://www.thunderbird.net/es-ES/}, y  agrega un canal de noticias de una empresa de mensajería que escojas. Realiza capturas de todo el proceso, teniendo en cuenta que, en todas las capturas deberá aparecer tu perfil de Moodle. Añade las capturas y las explicaciones de dichas capturas al documento PDF.

Debes buscar dentro de la web de la empresa elegida un archivo de este \href{https://e00-marca.uecdn.es/rss/portada.xml}{estilo}, en este caso es el canal rss de la web deportiva "marca".

\section{Solución}

% Bibliography

\newpage

\bibliography{citas}
\bibliographystyle{unsrt}

\end{document}