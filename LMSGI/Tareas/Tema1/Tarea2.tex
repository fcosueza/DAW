\input{preambulo.tex}

%----------------------------------------------------------------------------------------
%	TÍTULO Y DATOS DEL ALUMNO
%----------------------------------------------------------------------------------------

\title{
\normalfont \normalsize
\huge \textbf{Instalación del ERP Odoo}
}

\author{Francisco Javier Sueza Rodríguez}
\date{\normalsize\today}

%----------------------------------------------------------------------------------------
%                                     DOCUMENTO
%----------------------------------------------------------------------------------------
\begin{document}

\maketitle

\begin{center}
    \begin{tabular}{l l}
        \textbf{Centro}: & IES Aguadulce \\
        \textbf{Ciclo Formativo}: & Desarrollo Aplicaciones Web (Distancia)\\
        \textbf{Asignatura}: & Lenguajes de Marcas y Sistemas de Gestión \\
        \textbf{Tema}: & Tema 1 - Aspectos Básicos de los LM y SGI \\
    \end{tabular}
\end{center}

%\tableofcontents

%\listoffigures

%\listoftables

\vspace{10ex}

\section{Introducción}
En este ejercicio vamos a instalar el sistema de gestión empresaria \textbf{Odoo}. Hemos elegido realizar una instalación mediante una \textbf{máquina virtual}, en concreto, \textbf{Virtual Box}, aunque el proceso de instalación de la MV no se incluye en este documento. Además de instalar el ERP, se realizará la instalación de varios módulos, configurando y añadiendo información a alguno de ellos.

\section{Instalación de Odoo}


% Bibliography

\newpage
\bibliography{citas}
\bibliographystyle{unsrt}

\end{document}