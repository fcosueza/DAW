\input{preambulo.tex}

%----------------------------------------------------------------------------------------
%	TÍTULO Y DATOS DEL ALUMNO
%----------------------------------------------------------------------------------------

\title{
\normalfont \normalsize
\textsc{{\bfseries Curso 2022-2023} \\ Ciclo Superior de Desarrollo de Aplicaciones Web \\ IES Aguadulce} \\ [25pt]
\horrule{0.5pt} \\[0.4cm]
\huge Lenguajes de Marcas y Sistemas de Gestión de la Información\\
\horrule{0.5pt} \\[0.4cm]
}

\author{Francisco Javier Sueza Rodríguez}
\date{\normalsize\today}

%----------------------------------------------------------------------------------------
%                                     DOCUMENTO
%----------------------------------------------------------------------------------------

\makeglossaries
\loadglsentries{glossary.tex}

\begin{document}

\maketitle

\newpage

\tableofcontents

%\listoffigures

%\listoftables

\newpage
\chapter{Lenguajes de Marcas y Sistemas de Gestión de la Información}
En esta unidad vamos a estudiar los aspectos básicos de los lenguajes de marcas y los sistemas de gestión de la información. Por un lado, veremos la evolución de los \textbf{lenguajes de marcas}, desde GML hasta HTML, así como sus elementos y atributos, haciendo especial énfasis en XML. A continuación, veremos en que consisten los \textbf{sistemas de gestión de la información}, en concreto los \textbf{ERP}, sus características, configuración básica, personalización,..etc.

\section{Definición y Clasificación de los Lenguajes de Marcas}
Los <<lenguajes de marcas>> sirven para \textbf{codificar un documentos}. Estos incorporan \textbf{etiquetas} o marcas con \textbf{información adicional} sobre como se estructura el texto o como se presenta. El lenguaje de marcas será el que defina que etiquetas se permiten, donde deben colocarse y que significado tienen.

Todo lenguaje de marcas esta definido en un documento denominado \textbf{\gls{DTD}}, donde se establecen las marcas, los elementos utilizados por dicho lenguaje y sus correspondientes etiquetas y atributos, así como su sintaxis.

Los lenguajes de marcas se pueden clasificar, principalmente, en tres grupos:

\begin{itemize}
    \item \textbf{Orientados a la presentación}: son los utilizados generalmente por los procesadores de texto y definen como debe presentarse el documento, es decir, el formato que tiene.
    \item \textbf{De procedimientos}: orientados también a la presentación, pero en este caso, dentro de un \textbf{marco procedural} que permite la definición de macros, es decir, el programa que representa el documento debe interpretar el código en el mismo orden que aparece. Algunos ejemplos son \textbf{TeX}, \textbf{LaTeX} y \textbf{Postscript}
    \item \textbf{Descriptivos o semánticos}: estos lenguajes no describen la presentación del documento, sino que \textbf{describen la información}, que es lo que se esta representando sin especificar como debe presentarse.
\end{itemize}

Algunos ejemplos de lenguajes de marcado agrupados por su ámbito de uso son los siguientes:

\begin{itemize}
    \item \textbf{Documentación Electrónica}:
    \begin{itemize}
        \item \textbf{RTF} (Rich Text Format): fue desarrollado por Microsoft en 1987 y permite el intercambio de documentos entre los diferentes procesador de texto.
        \item \textbf{TeX}: creado por \href{https://es.wikipedia.org/wiki/Donald_Knuth}{Donald Knuth}, este lenguaje esta especialmente enfocado en la creación de textos científicos. Es considerado la mejor forma de componer formulas matemáticas complejas. \cite{tex}
        \item \textbf{Wikitexto}: permite la creación de páginas wiki en servidores preparados para soportar este lenguaje.
        \item \textbf{DocBook}: permite generar documentos separando la estructura lógica del documento de su formato, permitiendo que estos documentos puedan publicarse en diferentes formatos sin tener que modificar el documento original.
    \end{itemize}
    \item \textbf{Tecnologías de Internet}:
    \begin{itemize}
        \item \textbf{HTML},\textbf{XHTML} (Hypertext Markup Language, eXtensible Hypertext Markup Language): estos lenguajes están orientados a la creación de páginas web.
        \item \textbf{RSS} (Really Simple Sindication): permite la difusión de contenido web mediante la sindicación de contenidos.
    \end{itemize}
    \item Otros lenguajes especializados:
    \begin{itemize}
        \item \textbf{MathML} (Mathematica Markup Language): especializado en expresar los formalismos matemáticos de forma que puedan ser entendidos por diferentes aplicaciones.
        \item \textbf{VoiceXML} (Voice eXtended Markup Language): permite el intercambio de información entre usuarios y una aplicación con capacidad de reconocer el habla.
        \item \textbf{MusicXML}: permite el intercambio de partituras entre diferentes editores de partituras.
    \end{itemize}
\end{itemize}

\section{Evolución de los Lenguajes de Marcas}
A finales de los \textbf{años 60} surgen unos lenguajes informáticos, diferentes de los lenguajes de programación, orientados a la gestión de la información. Con el desarrollo de los editores y procesadores de texto surgen los primeros lenguajes informáticos orientados a la descripción y estructuración de la información: \textbf{los lenguajes de marcas}. Paralelamente también surgen otros lenguajes orientados a la representación, almacenamiento y consultar de grandes cantidades de datos: lenguajes y sistemas de bases de datos.

Los lenguajes de marcas surgieron inicialmente como lenguajes formados por un conjunto de códigos que los procesadores de textos insertaban en los documentos para dirigir el proceso de presentación (impresión) mediante una impresora. Al igual que los lenguajes de programación, estos estaban \textbf{ligados} a las características de los \textbf{procesadores de texto}y las \textbf{impresoras} en los que se usaban y no permitían a los programadores abstraerse de dichas características.

Posteriormente se añadió como medio de presentación a la pantalla y se automatizó el proceso, teniendo ya solo que pulsar una combinación teclas para lograr los resultados deseados en vez de hacerlo a mano. Este marcado estaba orientado exclusivamente a la presentación de la información, aunque posteriormente se le dieron nuevos uso surgiendo con ello el \textbf{formato generalizado}.




% Glossary

\glsaddall
\printglossaries

% Bibliography

\newpage
\addcontentsline{toc}{chapter}{Bibliografía}
\bibliography{citas}
\bibliographystyle{unsrt}

\end{document}