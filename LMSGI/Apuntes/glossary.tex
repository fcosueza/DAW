\newglossaryentry{DTD}{
    name={DTD},
    description={Document Type Definition}
}

\newglossaryentry{GML}{
    name={GML},
    description={Generalized Markup Language}
}

\newglossaryentry{SGML}{
    name={SGML},
    description={Standard Generalized Markup Language}
}

\newglossaryentry{HTML}{
    name={HTML},
    description={Hypertext Markup Language}
}

\newglossaryentry{ASCII}{
    name={ASCII},
    description={American Standard Code for Information Interchange}
}

\newglossaryentry{W3C}{
    name={W3C},
    description={Word Wide Web Consortium}
}

\newglossaryentry{XML}{
    name={XML},
    description={eXtensible Markup Language}
}

\newglossaryentry{metalenguaje}{
    name={metalenguaje},
    description={Lenguaje que permite la definición de otros lenguajes}
}

\newglossaryentry{texto plano}{
    name={texto plano},
    description={Es aquel texto formado solo por datos sin formato, es decir, solo por caractéres.}
}

\newglossaryentry{UTF-8}{
    name={UTF-8},
    description={8-bit Unicode Transformation Format}
}

\newglossaryentry{unicode}{
    name={unicode},
    description={Es un código que permite el tratamiento informático de textos en cualquier lenguaje y disciplina técnica, ya que incluye todos los carácteres conocidos para cualquier lengua. Es compatible con ASCII}
}

\newglossaryentry{URI}{
    name={URI},
    description={Uniform Resource Identifier. Son hypervínculos que dan acceso a un recurso remoto}
}


